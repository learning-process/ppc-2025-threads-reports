\documentclass[14pt,a4paper]{extarticle}
\renewcommand{\normalsize}{\fontsize{14pt}{16.8pt}\selectfont}
\usepackage[T2A]{fontenc}
\usepackage[utf8]{inputenc}
\usepackage[russian]{babel}
\usepackage{geometry}
\usepackage{setspace}
\usepackage{titlesec}
\usepackage{indentfirst}
\usepackage{amsmath}
\usepackage{graphicx}
\usepackage{float}
\usepackage{framed}
\usepackage{listings}
\usepackage{xcolor}
\usepackage{tabularx}
\usepackage{booktabs}
\usepackage{multirow}

\geometry{
  a4paper,
  left=30mm,
  right=15mm,
  top=20mm,
  bottom=20mm
}

\titleformat{\section}[block]
  {\normalfont\fontsize{14}{16}\bfseries\centering}
  {\thesection.}{0.5em}{}
\titleformat{\subsection}[block]
  {\normalfont\fontsize{14}{16}\bfseries\filcenter}
  {\thesubsection.}{0.5em}{}
\titleformat{\subsubsection}[block]
  {\normalfont\fontsize{14}{16}\bfseries\filcenter}
  {\thesubsubsection.}{0.5em}{}

\lstset{
    basicstyle=\ttfamily\small,
    keywordstyle=\color{blue},
    commentstyle=\color{green},
    stringstyle=\color{red},
    frame=single,
    tabsize=4,
    showstringspaces=false,
    breaklines=true
}

\sloppy

\onehalfspacing

\setlength{\parindent}{1.25cm}

\begin{document}

\begin{titlepage}
\begin{center}

\onehalfspacing

\begin{center}
    \textbf{МИНИСТЕРСТВО НАУКИ И ВЫСШЕГО ОБРАЗОВАНИЯ РОССИЙСКОЙ ФЕДЕРАЦИИ} \\
    Федеральное государственное автономное образовательное учреждение высшего образования \\
    «Национальный исследовательский Нижегородский университет им. Н.И. Лобачевского» (ННГУ) \\
    Институт информационных технологий, математики и механики
\end{center}

\vspace{4cm}

\begin{center}
    \textbf{ОТЧЕТ ПО ПРАКТИЧЕСКИМ РАБОТАМ} \vspace{0.5cm} \\
    по курсу «ПАРАЛЛЕЛЬНОЕ ПРОГРАММИРОВАНИЕ» \vspace{0.5cm} \\
    \textit{Задача: Умножение матриц с использованием алгоритма Штрассена}
\end{center}

\vspace{4cm}

\begin{flushright}
    \textbf{ВЫПОЛНИЛ:} \\ 
    Студент
    группы 3822Б1ПР3 \\ 
    Наседкин Е.Д. \\ 
\end{flushright}

\vspace{1cm}

\begin{center}
    Нижний Новгород \\
    2025 \newpage
\end{center}

\end{center}
\end{titlepage}

\tableofcontents
\newpage

\section{Введение}

Умножение матриц является одной из фундаментальных операций в вычислительной математике, используемой в различных областях, таких как обработка сигналов, машинное обучение, компьютерная графика и решение систем линейных уравнений. Стандартный алгоритм умножения матриц имеет вычислительную сложность \(O(n^3)\), что делает его неэффективным для матриц большого размера. Алгоритм Штрассена, предложенный Фолькером Штрассеном в 1969 году, позволяет снизить сложность до \(O(n^{\log_2 7})\), что делает его более эффективным для больших матриц.

Целью данной работы является реализация и исследование алгоритма Штрассена для умножения матриц с использованием различных технологий параллельного программирования.

\section{Постановка задачи}

Целью практической работы является разработка и экспериментальное исследование алгоритма Штрассена для умножения матриц. Задача включает реализацию последовательной версии алгоритма и его параллельных версий с использованием различных технологий параллельного программирования.

В рамках работы необходимо:

\begin{itemize}
    \item Реализовать последовательный алгоритм Штрассена для умножения квадратных матриц.
    \item Разработать параллельные версии алгоритма с использованием следующих технологий:
    \begin{enumerate}
        \item OpenMP;
        \item Intel Threading Building Blocks (TBB);
        \item Стандартная библиотека потоков C++ (STL threads);
        \item MPI с вложенным использованием STL threads.
    \end{enumerate}
    \item Провести тестирование и сравнение различных реализаций по времени выполнения и корректности результатов.
\end{itemize}

Результатом работы должно стать сравнение производительности реализованных версий и выводы об эффективности применения различных технологий параллельного программирования в задаче умножения матриц.

\section{Описание алгоритма}

Алгоритм Штрассена основан на рекурсивном разбиении матриц на четыре подматрицы и выполнении семи умножений вместо восьми, как в стандартном алгоритме. Это достигается за счёт использования специальных комбинаций сложения и вычитания подматриц.

Основные этапы алгоритма:

\begin{enumerate}
    \item Входные матрицы \(A\) и \(B\) размером \(n \times n\) разбиваются на четыре подматрицы размером \(\frac{n}{2} \times \frac{n}{2}\):
    \[
    A = \begin{bmatrix} A_{11} & A_{12} \\ A_{21} & A_{22} \end{bmatrix}, \quad
    B = \begin{bmatrix} B_{11} & B_{12} \\ B_{21} & B_{22} \end{bmatrix}.
    \]
    \item Вычисляются семь промежуточных произведений:
    \[
    \begin{aligned}
        P_1 &= (A_{11} + A_{22}) \cdot (B_{11} + B_{22}), \\
        P_2 &= (A_{21} + A_{22}) \cdot B_{11}, \\
        P_3 &= A_{11} \cdot (B_{12} - B_{22}), \\
        P_4 &= A_{22} \cdot (B_{21} - B_{11}), \\
        P_5 &= (A_{11} + A_{12}) \cdot B_{22}, \\
        P_6 &= (A_{21} - A_{11}) \cdot (B_{11} + B_{12}), \\
        P_7 &= (A_{12} - A_{22}) \cdot (B_{21} + B_{22}).
    \end{aligned}
    \]
    \item Результирующая матрица \(C\) собирается из подматриц:
    \[
    \begin{aligned}
        C_{11} &= P_1 + P_4 - P_5 + P_7, \\
        C_{12} &= P_3 + P_5, \\
        C_{21} &= P_2 + P_4, \\
        C_{22} &= P_1 + P_3 - P_2 + P_6.
    \end{aligned}
    \]
    \item Если размер матрицы не является степенью двойки, матрицы дополняются нулями до ближайшей степени двойки, а после вычислений результирующая матрица обрезается до исходного размера.
    \item Для малых матриц (размером менее или равным 32) используется стандартное умножение матриц для оптимизации производительности.
\end{enumerate}

Алгоритм рекурсивно применяется к подматрицам, пока их размер не станет меньше порогового значения, после чего используется стандартное умножение.

\section{Описание схемы параллельного алгоритма}

Основная идея параллелизации заключается в следующем:

\begin{itemize}
    \item Разбиение матриц на подматрицы и вычисление промежуточных произведений распределяются между потоками или процессами.
    \item Для технологий с общей памятью (OpenMP, TBB, STL threads) параллелизация осуществляется на уровне отдельных умножений подматриц.
    \item Для распределённых систем (MPI) задачи распределяются между процессами, с учётом необходимости передачи данных между узлами.
\end{itemize}

Параллельные версии алгоритма используют модель параллелизма по задачам, где каждая задача соответствует вычислению одного из семи промежуточных произведений.

\section{Описание OpenMP-версии алгоритма}

Технология OpenMP позволяет распараллеливать вычисления с помощью директив компилятора. Основной акцент сделан на распараллеливание операций сложения, вычитания и умножения подматриц.

Ключевые особенности реализации:

\begin{itemize}
    \item Параллелизация осуществляется с использованием директивы \texttt{\#pragma omp parallel for} для операций сложения и вычитания матриц, а также \texttt{\#pragma omp parallel sections} для одновременного вычисления семи промежуточных произведений (\(P_1, \ldots, P_7\)).
    \item Разбиение матриц на подматрицы и их объединение также распараллелены с помощью \texttt{\#pragma omp parallel sections}.
    \item Поскольку вычисления подматриц независимы, синхронизация между потоками минимальна, что снижает накладные расходы.
    \item Для копирования входных и выходных данных используются параллельные циклы, что улучшает производительность на больших матрицах.
\end{itemize}

OpenMP обеспечивает простоту реализации и хорошую масштабируемость на системах с общей памятью, автоматически распределяя задачи между доступными потоками.

\section{Описание TBB-версии алгоритма}

Библиотека Intel oneAPI Threading Building Blocks (TBB) предоставляет высокоуровневые средства для организации параллелизма с автоматической балансировкой нагрузки.

Ключевые особенности реализации:

\begin{itemize}
    \item Параллелизация осуществляется с помощью конструкции \texttt{tbb::parallel\_invoke}, которая распределяет вычисление семи промежуточных произведений между потоками.
    \item TBB автоматически управляет пулом потоков, распределяя задачи для оптимального использования ресурсов.
    \item Разбиение и объединение подматриц выполняется последовательно, так как эти операции менее вычислительно затратны по сравнению с умножением.
\end{itemize}

TBB обеспечивает гибкость и высокую производительность, особенно на многоядерных системах, благодаря динамическому распределению задач.

\section{Описание STL-версии алгоритма}

В данной реализации использовалась стандартная библиотека потоков C++ (STL threads) для создания низкоуровневой многопоточной версии алгоритма. Параллелизация организована вручную, с явным управлением потоками.

Ключевые особенности реализации:

\begin{itemize}
    \item Число потоков определяется с помощью утилиты \texttt{ppc::util::GetPPCNumThreads()}.
    \item Семь промежуточных произведений распределяются между потоками, создаваемыми с помощью \texttt{std::thread}.
    \item Каждый поток выполняет задачу вычисления одного произведения, а результаты объединяются после завершения всех потоков с помощью \texttt{join()}.
    \item Для предотвращения состояний гонки используется явное управление диапазонами задач.
\end{itemize}

STL threads предоставляют полный контроль над потоками, но требуют тщательной настройки для достижения оптимальной производительности.

\section{Описание MPI-версии алгоритма}

В данной версии использовалась библиотека Boost.MPI с вложенным применением STL threads для реализации распределённого параллелизма. Этот подход позволяет масштабировать вычисления на кластерные системы.

Ключевые этапы реализации:

\begin{itemize}
    \item На нулевом процессе выполняется предварительная обработка входных матриц, включая их дополнение до размера, кратного степени двойки.
    \item Входные матрицы и их размеры передаются всем процессам с помощью \texttt{boost::mpi::broadcast}.
    \item Семь промежуточных произведений распределяются между процессами, где каждый процесс выполняет свои задачи с использованием STL threads для дополнительной параллелизации.
    \item Результаты произведений собираются на нулевом процессе с помощью \texttt{world.recv}, после чего формируется результирующая матрица.
    \item Синхронизация между процессами осуществляется с помощью \texttt{world.barrier()}.
\end{itemize}

Гибридный подход MPI + STL threads позволяет эффективно использовать ресурсы кластера, распределяя задачи между узлами и внутри каждого узла.

\section{Анализ производительности вычислений}

Для оценки производительности реализованных версий алгоритма были проведены измерения времени выполнения на матрицах размером 512x512. Тестирование включало измерение времени выполнения полной цепочки обработки (\texttt{PipelineRun}) и изолированного ядра вычислений (\texttt{TaskRun}). Все параллельные реализации тестировались с использованием 2 потоков или 2 процессов (для MPI + STL).

\subsection{Сравнение производительности}

Производительность оценивалась по следующим метрикам:
\begin{itemize}
    \item \texttt{PipelineRun} — время выполнения полной цепочки обработки (в секундах).
    \item \texttt{TaskRun} — время выполнения основного вычислительного ядра (в секундах).
    \item \textbf{Speedup} — коэффициент ускорения, рассчитанный как отношение времени \texttt{PipelineRun} последовательной версии к времени \texttt{PipelineRun} параллельной версии:
    \[
    \text{Speedup} = \frac{T_{\text{seq}}}{T_{\text{parallel}}}
    \]
\end{itemize}

\renewcommand{\arraystretch}{1.3}
\begin{table}[H]
\centering
\footnotesize
\begin{tabularx}{\textwidth}{|l|X|X|X|}
\hline
\textbf{Реализация} & \textbf{PipelineRun (с)} & \textbf{TaskRun (с)} & \textbf{Speedup} \\
\hline
TBB & 0.3213 & 0.3177 & 1.88 \\
\hline
OpenMP & 0.3411 & 0.3459 & 1.77 \\
\hline
Последовательная & 0.6040 & 0.5984 & 1.00 \\
\hline
STL threads & 0.5509 & 0.5433 & 1.10 \\
\hline
MPI + STL & 0.9746 & 0.8881 & 0.62 \\
\hline
\end{tabularx}
\caption{Сравнение производительности реализаций алгоритма Штрассена}
\end{table}

\subsection{Анализ результатов}

Результаты тестирования показывают, что:
\begin{itemize}
    \item TBB-реализация обеспечила наилучшую производительность с временем \texttt{PipelineRun} 0.3213 с и \texttt{TaskRun} 0.3177 с, что соответствует ускорению 1.88. Это обусловлено эффективным управлением потоками и динамической балансировкой нагрузки.
    \item OpenMP-реализация близка по производительности к TBB (ускорение 1.77), благодаря простоте параллелизации и минимальной синхронизации.
    \item Последовательная версия, как ожидалось, показала базовое время выполнения, служащее эталоном для расчёта ускорения.
    \item STL threads дала скромное ускорение (1.10), что связано с накладными расходами на ручное управление потоками.
    \item MPI + STL-реализация оказалась наименее эффективной (ускорение 0.62) из-за высоких затрат на коммуникацию между процессами, что делает её неподходящей для матриц размером 512x512.
\end{itemize}

Полученные данные подчёркивают преимущества технологий с общей памятью (TBB, OpenMP) для данной задачи.

\section{Заключение}

В ходе выполнения работы был реализован алгоритм Штрассена для умножения матриц в последовательной и параллельных версиях с использованием технологий OpenMP, TBB, STL threads и MPI. Каждая реализация была адаптирована для эффективного использования вычислительных ресурсов, будь то многоядерные процессоры или распределённые системы.

Результаты тестирования подтвердили, что параллельные реализации на основе TBB и OpenMP обеспечивают значительное ускорение по сравнению с последовательной версией, особенно для матриц размером 512x512. TBB показала наилучшую производительность благодаря эффективному управлению потоками. Реализация на STL threads оказалась менее эффективной из-за ручного управления потоками, а MPI + STL threads продемонстрировала низкую производительность из-за накладных расходов на коммуникацию.

Работа позволила углубить понимание методов параллельного программирования и их применения к вычислительно интенсивным задачам.

\section{Список литературы}

\begin{enumerate}
    \item Штрассен В. \textit{Гауссово исключение не оптимально} // Numerische Mathematik, 1969.
    \item Статья: \textit{Что такое OpenMP?}
    \item Сысоев А.В., Мееров И.Б., Сиднев А.А. \textit{Средства разработки параллельных программ для систем с общей памятью. Библиотека Intel Threading Building Blocks}. — Нижний Новгород, 2007.
    \item Статья: \textit{Технологии параллельного программирования. Message Passing Interface (MPI)}.
\end{enumerate}

\section{Приложение}

В данном разделе приведены ключевые фрагменты реализации алгоритма Штрассена для умножения матриц с использованием различных технологий параллельного программирования. Представлены только основные вычислительные фрагменты (функции \texttt{StrassenMultiply} или эквивалентные).

\subsection{Последовательная версия}
\begin{lstlisting}[language=C++]
std::vector<double> StrassenSequential::StrassenMultiply(
    const std::vector<double>& a, const std::vector<double>& b, int size) {
  if (size <= 32) {
    return StandardMultiply(a, b, size);
  }

  int half_size = size / 2;
  std::vector<double> a11(half_size * half_size);
  std::vector<double> a12(half_size * half_size);
  std::vector<double> a21(half_size * half_size);
  std::vector<double> a22(half_size * half_size);
  std::vector<double> b11(half_size * half_size);
  std::vector<double> b12(half_size * half_size);
  std::vector<double> b21(half_size * half_size);
  std::vector<double> b22(half_size * half_size);

  SplitMatrix(a, a11, 0, 0, size);
  SplitMatrix(a, a12, 0, half_size, size);
  SplitMatrix(a, a21, half_size, 0, size);
  SplitMatrix(a, a22, half_size, half_size, size);
  SplitMatrix(b, b11, 0, 0, size);
  SplitMatrix(b, b12, 0, half_size, size);
  SplitMatrix(b, b21, half_size, 0, size);
  SplitMatrix(b, b22, half_size, half_size, size);

  std::vector<double> p1 =
      StrassenMultiply(AddMatrices(a11, a22, half_size), AddMatrices(b11, b22, half_size), half_size);
  std::vector<double> p2 = StrassenMultiply(AddMatrices(a21, a22, half_size), b11, half_size);
  std::vector<double> p3 = StrassenMultiply(a11, SubtractMatrices(b12, b22, half_size), half_size);
  std::vector<double> p4 = StrassenMultiply(a22, SubtractMatrices(b21, b11, half_size), half_size);
  std::vector<double> p5 = StrassenMultiply(AddMatrices(a11, a12, half_size), b22, half_size);
  std::vector<double> p6 =
      StrassenMultiply(SubtractMatrices(a21, a11, half_size), AddMatrices(b11, b12, half_size), half_size);
  std::vector<double> p7 =
      StrassenMultiply(SubtractMatrices(a12, a22, half_size), AddMatrices(b21, b22, half_size), half_size);

  std::vector<double> c11 = AddMatrices(SubtractMatrices(AddMatrices(p1, p4, half_size), p5, half_size), p7, half_size);
  std::vector<double> c12 = AddMatrices(p3, p5, half_size);
  std::vector<double> c21 = AddMatrices(p2, p4, half_size);
  std::vector<double> c22 = AddMatrices(SubtractMatrices(AddMatrices(p1, p3, half_size), p2, half_size), p6, half_size);

  std::vector<double> result(size * size);
  MergeMatrix(result, c11, 0, 0, size);
  MergeMatrix(result, c12, 0, half_size, size);
  MergeMatrix(result, c21, half_size, 0, size);
  MergeMatrix(result, c22, half_size, half_size, size);

  return result;
}
\end{lstlisting}

\subsection{OpenMP-версия}
\begin{lstlisting}[language=C++]
std::vector<double> StrassenOmp::StrassenMultiply(const std::vector<double>& a, const std::vector<double>& b, int size) {
  if (size <= 32) {
    return StandardMultiply(a, b, size);
  }

  int half_size = size / 2;
  std::vector<double> a11(half_size * half_size);
  std::vector<double> a12(half_size * half_size);
  std::vector<double> a21(half_size * half_size);
  std::vector<double> a22(half_size * half_size);
  std::vector<double> b11(half_size * half_size);
  std::vector<double> b12(half_size * half_size);
  std::vector<double> b21(half_size * half_size);
  std::vector<double> b22(half_size * half_size);

#pragma omp parallel sections
  {
#pragma omp section
    SplitMatrix(a, a11, 0, 0, size);
#pragma omp section
    SplitMatrix(a, a12, 0, half_size, size);
#pragma omp section
    SplitMatrix(a, a21, half_size, 0, size);
#pragma omp section
    SplitMatrix(a, a22, half_size, half_size, size));
#pragma omp section
    SplitMatrix(b, b11, 0, 0, size);
#pragma omp section
    SplitMatrix(b, b12, 0, half_size, size);
#pragma omp section
    SplitMatrix(b, b21, half_size, 0, size);
#pragma omp section
    SplitMatrix(b, b22, half_size, half_size, size);
  }

  std::vector<double> p1;
  std::vector<double> p2;
  std::vector<double> p3;
  std::vector<double> p4;
  std::vector<double> p5;
  std::vector<double> p6;
  std::vector<double> p7;

#pragma omp parallel sections
  {
#pragma omp section
    p1 = StrassenMultiply(AddMatrices(a11, a22, half_size), AddMatrices(b11, b22, half_size), half_size);
#pragma omp section
    p2 = StrassenMultiply(AddMatrices(a21, a22, half_size), b11, half_size);
#pragma omp section
    p3 = StrassenMultiply(a11, SubtractMatrices(b12, b22, half_size), half_size);
#pragma omp section
    p4 = StrassenMultiply(a22, SubtractMatrices(b21, b11, half_size), half_size);
#pragma omp section
    p5 = StrassenMultiply(AddMatrices(a11, a12, half_size), b22, half_size);
#pragma omp section
    p6 = StrassenMultiply(SubtractMatrices(a21, a11, half_size), AddMatrices(b11, b12, half_size), half_size);
#pragma omp section
    p7 = StrassenMultiply(SubtractMatrices(a12, a22, half_size), AddMatrices(b21, b22, half_size), half_size);
  }

  std::vector<double> c11 = AddMatrices(SubtractMatrices(AddMatrices(p1, p4, half_size), p5, half_size), p7, half_size);
  std::vector<double> c12 = AddMatrices(p3, p5, half_size);
  std::vector<double> c21 = AddMatrices(p2, p4, half_size);
  std::vector<double> c22 = AddMatrices(SubtractMatrices(AddMatrices(p1, p3, half_size), p2, half_size), p6, half_size);

  std::vector<double> result(size * size);
#pragma omp parallel sections
  {
#pragma omp section
    MergeMatrix(result, c11, 0, 0, size);
#pragma omp section
    MergeMatrix(result, c12, 0, half_size, size);
#pragma omp section
    MergeMatrix(result, c21, half_size, 0, size);
#pragma omp section
    MergeMatrix(result, c22, half_size, half_size, size);
  }

  return result;
}
\end{lstlisting}

\subsection{TBB-версия}
\begin{lstlisting}[language=C++]
std::vector<double> StrassenTbb::StrassenMultiply(const std::vector<double>& a, const std::vector<double>& b, int size) {
  if (size <= 32) {
    return StandardMultiply(a, b, size);
  }

  int half_size = size / 2;
  int half_size_squared = half_size * half_size;

  std::vector<double> a11(half_size_squared);
  std::vector<double> a12(half_size_squared);
  std::vector<double> a21(half_size_squared);
  std::vector<double> a22(half_size_squared);
  std::vector<double> b11(half_size_squared);
  std::vector<double> b12(half_size_squared);
  std::vector<double> b21(half_size_squared);
  std::vector<double> b22(half_size_squared);

  SplitMatrix(a, a11, 0, 0, size);
  SplitMatrix(a, a12, 0, half_size, size);
  SplitMatrix(a, a21, half_size, 0, size);
  SplitMatrix(a, a22, half_size, half_size, size);
  SplitMatrix(b, b11, 0, 0, size);
  SplitMatrix(b, b12, 0, half_size, size);
  SplitMatrix(b, b21, half_size, 0, size);
  SplitMatrix(b, b22, half_size, half_size, size);

  std::vector<double> p1(half_size_squared);
  std::vector<double> p2(half_size_squared);
  std::vector<double> p3(half_size_squared);
  std::vector<double> p4(half_size_squared);
  std::vector<double> p5(half_size_squared);
  std::vector<double> p6(half_size_squared);
  std::vector<double> p7(half_size_squared);

  tbb::parallel_invoke(
      [&]() { p1 = StrassenMultiply(AddMatrices(a11, a22, half_size), AddMatrices(b11, b22, half_size), half_size); },
      [&]() { p2 = StrassenMultiply(AddMatrices(a21, a22, half_size), b11, half_size); },
      [&]() { p3 = StrassenMultiply(a11, SubtractMatrices(b12, b22, half_size), half_size); },
      [&]() { p4 = StrassenMultiply(a22, SubtractMatrices(b21, b11, half_size), half_size); },
      [&]() { p5 = StrassenMultiply(AddMatrices(a11, a12, half_size), b22, half_size); },
      [&]() {
        p6 = StrassenMultiply(SubtractMatrices(a21, a11, half_size), AddMatrices(b11, b12, half_size), half_size);
      },
      [&]() {
        p7 = StrassenMultiply(SubtractMatrices(a12, a22, half_size), AddMatrices(b21, b22, half_size), half_size);
      });

  std::vector<double> c11 = AddMatrices(SubtractMatrices(AddMatrices(p1, p4, half_size), p5, half_size), p7, half_size);
  std::vector<double> c12 = AddMatrices(p3, p5, half_size);
  std::vector<double> c21 = AddMatrices(p2, p4, half_size);
  std::vector<double> c22 = AddMatrices(SubtractMatrices(AddMatrices(p1, p3, half_size), p2, half_size), p6, half_size);

  std::vector<double> result(size * size);
  MergeMatrix(result, c11, 0, 0, size);
  MergeMatrix(result, c12, 0, half_size, size);
  MergeMatrix(result, c21, half_size, 0, size);
  MergeMatrix(result, c22, half_size, half_size, size);

  return result;
}
\end{lstlisting}

\subsection{STL (std::thread) версия}
\begin{lstlisting}[language=C++]
std::vector<double> StrassenStl::StrassenMultiply(const std::vector<double>& a, const std::vector<double>& b, int size,
                                                  int num_threads) {
  if (size <= 32) {
    return StandardMultiply(a, b, size);
  }

  int half_size = size / 2;
  int half_size_squared = half_size * half_size;

  std::vector<double> a11(half_size_squared);
  std::vector<double> a12(half_size_squared);
  std::vector<double> a21(half_size_squared);
  std::vector<double> a22(half_size_squared);
  std::vector<double> b11(half_size_squared);
  std::vector<double> b12(half_size_squared);
  std::vector<double> b21(half_size_squared);
  std::vector<double> b22(half_size_squared);

  SplitMatrix(a, a11, 0, 0, size);
  SplitMatrix(a, a12, 0, half_size, size);
  SplitMatrix(a, a21, half_size, 0, size);
  SplitMatrix(a, a22, half_size, half_size, size);
  SplitMatrix(b, b11, 0, 0, size);
  SplitMatrix(b, b12, 0, half_size, size);
  SplitMatrix(b, b21, half_size, 0, size);
  SplitMatrix(b, b22, half_size, half_size, size);

  std::vector<double> p1(half_size_squared);
  std::vector<double> p2(half_size_squared);
  std::vector<double> p3(half_size_squared);
  std::vector<double> p4(half_size_squared);
  std::vector<double> p5(half_size_squared);
  std::vector<double> p6(half_size_squared);
  std::vector<double> p7(half_size_squared);

  std::vector<std::function<void()>> tasks;
  tasks.reserve(7);
  tasks.emplace_back([&]() {
    p1 = StrassenMultiply(AddMatrices(a11, a22, half_size), AddMatrices(b11, b22, half_size), half_size, num_threads);
  });
  tasks.emplace_back([&]() { p2 = StrassenMultiply(AddMatrices(a21, a22, half_size), b11, half_size, num_threads); });
  tasks.emplace_back(
      [&]() { p3 = StrassenMultiply(a11, SubtractMatrices(b12, b22, half_size), half_size, num_threads); });
  tasks.emplace_back(
      [&]() { p4 = StrassenMultiply(a22, SubtractMatrices(b21, b11, half_size), half_size, num_threads); });
  tasks.emplace_back([&]() { p5 = StrassenMultiply(AddMatrices(a11, a12, half_size), b22, half_size, num_threads); });
  tasks.emplace_back([&]() {
    p6 = StrassenMultiply(SubtractMatrices(a21, a11, half_size), AddMatrices(b11, b12, half_size), half_size,
                          num_threads);
  });
  tasks.emplace_back([&]() {
    p7 = StrassenMultiply(SubtractMatrices(a12, a22, half_size), AddMatrices(b21, b22, half_size), half_size,
                          num_threads);
  });

  std::vector<std::thread> threads;
  threads.reserve(std::min(num_threads, static_cast<int>(tasks.size())));
  size_t task_index = 0;

  for (int i = 0; i < std::min(num_threads, static_cast<int>(tasks.size())); ++i) {
    if (task_index < tasks.size()) {
      threads.emplace_back(tasks[task_index]);
      ++task_index;
    }
  }

  while (task_index < tasks.size()) {
    tasks[task_index]();
    ++task_index;
  }

  for (auto& thread : threads) {
    if (thread.joinable()) {
      thread.join();
    }
  }

  std::vector<double> c11 = AddMatrices(SubtractMatrices(AddMatrices(p1, p4, half_size), p5, half_size), p7, half_size);
  std::vector<double> c12 = AddMatrices(p3, p5, half_size);
  std::vector<double> c21 = AddMatrices(p2, p4, half_size);
  std::vector<double> c22 = AddMatrices(SubtractMatrices(AddMatrices(p1, p3, half_size), p2, half_size), p6, half_size);

  std::vector<double> result(size * size);
  MergeMatrix(result, c11, 0, 0, size);
  MergeMatrix(result, c12, 0, half_size, size);
  MergeMatrix(result, c21, half_size, 0, size);
  MergeMatrix(result, c22, half_size, half_size, size);

  return result;
}
\end{lstlisting}

\subsection{MPI + STL threads версия}
\begin{lstlisting}[language=C++]
void StrassenAll::StrassenWorker(int prod_idx, const std::vector<double> &a, const std::vector<double> &b, int size,
                                 std::vector<double> &result, std::mutex &mtx) {
  std::vector<double> local_result;
  if (size <= 32) {
    local_result = StandardMultiply(a, b, size);
  } else {
    int half_size = size / 2;
    std::vector<double> a11(half_size * half_size);
    std::vector<double> a12(half_size * half_size);
    std::vector<double> a21(half_size * half_size);
    std::vector<double> a22(half_size * half_size);
    std::vector<double> b11(half_size * half_size);
    std::vector<double> b12(half_size * half_size);
    std::vector<double> b21(half_size * half_size);
    std::vector<double> b22(half_size * half_size);

    SplitMatrix(a, a11, 0, 0, size);
    SplitMatrix(a, a12, 0, half_size, size);
    SplitMatrix(a, a21, half_size, 0, size);
    SplitMatrix(a, a22, half_size, half_size, size);
    SplitMatrix(b, b11, 0, 0, size);
    SplitMatrix(b, b12, 0, half_size, size);
    SplitMatrix(b, b21, half_size, 0, size);
    SplitMatrix(b, b22, half_size, half_size, size);

    std::vector<std::vector<double>> inputs_a(7);
    std::vector<std::vector<double>> inputs_b(7);
    switch (prod_idx) {
      case 0:
        inputs_a[0] = AddMatrices(a11, a22, half_size);
        inputs_b[0] = AddMatrices(b11, b22, half_size);
        break;
      case 1:
        inputs_a[1] = AddMatrices(a21, a22, half_size);
        inputs_b[1] = b11;
        break;
      case 2:
        inputs_a[2] = a11;
        inputs_b[2] = SubtractMatrices(b12, b22, half_size);
        break;
      case 3:
        inputs_a[3] = a22;
        inputs_b[3] = SubtractMatrices(b21, b11, half_size);
        break;
      case 4:
        inputs_a[4] = AddMatrices(a11, a12, half_size);
        inputs_b[4] = b22;
        break;
      case 5:
        inputs_a[5] = SubtractMatrices(a21, a11, half_size);
        inputs_b[5] = AddMatrices(b11, b12, half_size);
        break;
      case 6:
        inputs_a[6] = SubtractMatrices(a12, a22, half_size);
        inputs_b[6] = AddMatrices(b21, b22, half_size);
        break;
      default:
        break;
    }

    local_result.resize(half_size * half_size);
    if (size <= 32) {
      local_result = StandardMultiply(inputs_a[prod_idx], inputs_b[prod_idx], half_size);
    } else {
      std::vector<std::thread> threads;
      size_t num_threads = ppc::util::GetPPCNumThreads();
      threads.reserve(num_threads);
      std::mutex local_mtx;
      for (size_t t = 0; t < num_threads; ++t) {
        threads.emplace_back([&inputs_a, &inputs_b, prod_idx, half_size, &local_result, &local_mtx]() {
          auto temp = StrassenMultiply(inputs_a[prod_idx], inputs_b[prod_idx], half_size);
          std::lock_guard<std::mutex> lock(local_mtx);
          local_result = temp;
        });
      }
      for (auto &t : threads) {
        t.join();
      }
    }
  }

  std::lock_guard<std::mutex> lock(mtx);
  result = local_result;
}
\end{lstlisting}

\end{document}