\documentclass[a4paper,12pt]{article}
\usepackage[utf8]{inputenc}
\usepackage[T2A]{fontenc}
\usepackage[russian]{babel}
\usepackage{amsmath,amssymb}
\usepackage{amsfonts} % Добавлено для \mathbb
\usepackage{hyperref}
\usepackage{geometry}
\usepackage{graphicx}
\usepackage{float}
\usepackage{titlesec}
\usepackage{tabularx}
\usepackage{url}
\usepackage{listings}
\usepackage{multirow}
\usepackage{xcolor}
\hypersetup{hidelinks}
\usepackage{siunitx}

\geometry{top=2cm, bottom=2cm, left=2cm, right=2cm}

\sisetup{
output-decimal-marker={,},
group-digits = false
}

\setlength{\parindent}{1.25cm}
\setlength{\parskip}{0pt}

% Формат заголовков
\titleformat{\section}{\normalfont\Large\bfseries\centering}{\thesection.}{1em}{}
\titleformat{\subsection}{\normalfont\large\bfseries}{\thesubsection.}{1em}{}
\titleformat{\subsubsection}{\normalfont\normalsize\bfseries}{\thesubsubsection.}{1em}{}


\lstset{
	basicstyle=\ttfamily\small,
	keywordstyle=\color{blue},
	commentstyle=\color{green!60!black}, % Сделал зеленый потемнее для лучшей читаемости
	stringstyle=\color{red},
	frame=single,
	tabsize=4,
	showstringspaces=false,
	breaklines=true,
	numbers=left, % Добавил нумерацию строк
    numberstyle=\tiny\color{gray}, % Стиль нумерации строк
    breakatwhitespace=false, % Перенос строк только по пробелам - нет, лучше где угодно
    captionpos=b, % Позиция заголовка листинга (снизу)
    escapeinside={\%*}{*)} % Для вставки LaTeX команд внутри листинга, если понадобится
}

\hypersetup{hidelinks}

\begin{document}

% Титульный лист
\begin{titlepage}
\begin{center}
\textbf{МИНИСТЕРСТВО НАУКИ И ВЫСШЕГО ОБРАЗОВАНИЯ РОССИЙСКОЙ ФЕДЕРАЦИИ} \\[0.5cm]
\textbf{Федеральное государственное автономное образовательное учреждение высшего образования} \\[0.5cm]
\textbf{«Национальный исследовательский Нижегородский государственный университет им. Н.И. Лобачевского»} \\[0.5cm]
Институт информационных технологий, математики и механики \\
\vfill
{\Large
\textbf{Отчёт по лабораторной работе на тему:} \\[0.5cm]
\textbf{Вычисление многомерных интегралов с использованием многошаговой схемы (метод Симпсона)} \\
}
\vfill
\begin{flushright}
Выполнил: студент группы 3822Б1ПМоП3 \\
Ануфриев Даниил \\
\vspace{1cm}
Преподаватель: \\
Сысоев А.В., доцент, кандидат технических наук \\
\end{flushright}
\vfill
Нижний Новгород \\
2025
\end{center}
\end{titlepage}

% Оглавление
\tableofcontents
\newpage

% Введение
\section*{Введение}

Численное интегрирование является фундаментальной задачей вычислительной математики, находящей широкое применение в различных областях науки и техники, таких как физика, экономика, инженерное дело и машинное обучение. Особую сложность представляют многомерные интегралы, для которых аналитическое решение зачастую невозможно или крайне затруднительно. Метод Симпсона, относящийся к классу квадратурных формул Ньютона-Котеса, является одним из классических и наиболее точных методов численного интегрирования, использующим интерполяцию полиномами второй степени.

Актуальность данной работы обусловлена возрастающей потребностью в решении задач высокой вычислительной сложности, где время расчета играет критическую роль. Для многомерных интегралов количество вычислений подынтегральной функции растет экспоненциально с увеличением размерности, что делает последовательные алгоритмы неэффективными. Параллельные вычисления предоставляют возможность существенного ускорения таких расчетов за счет одновременного использования нескольких вычислительных ядер или узлов.

Целью настоящей лабораторной работы является разработка и исследование различных подходов к параллельной реализации алгоритма вычисления многомерных интегралов методом Симпсона. Рассматриваются последовательная версия и ее параллельные аналоги, использующие технологии OpenMP, Intel Threading Building Blocks (TBB), стандартные потоки C++ (STL) и гибридный подход MPI+TBB. Проводится сравнительный анализ производительности разработанных реализаций для оценки эффективности применения различных парадигм параллельного программирования к данной задаче.

\newpage

\section{Постановка задачи}
\label{sec:problem_statement}

В рамках данной работы ставится задача вычисления определенного многомерного интеграла вида:
\[
I = \int_{a_d}^{b_d} \cdots \int_{a_1}^{b_1} F(x_1, \dots, x_d) \,dx_1 \dots \,dx_d
\]
с использованием составной квадратурной формулы Симпсона. Необходимо реализовать алгоритм в последовательном варианте и с применением различных технологий параллельного программирования.

\subsection{Математическая модель}
\label{ssec:math_model}
Метод Симпсона для одномерного случая на отрезке $[x_i, x_{i+2}]$ с шагом $h$ имеет вид:
\[
\int_{x_i}^{x_{i+2}} f(x) \,dx \approx \frac{h}{3} [f(x_i) + 4f(x_{i+1}) + f(x_{i+2})].
\]
Составная формула Симпсона для отрезка $[a, b]$, разбитого на $N$ (четное число) элементарных отрезков длиной $h = (b-a)/N$, выглядит следующим образом:
\[
\int_a^b f(x) \,dx \approx \frac{h}{3} [f(x_0) + 4f(x_1) + 2f(x_2) + \dots + 4f(x_{N-1}) + f(x_N)].
\]
Для многомерного случая, область интегрирования $V = [a_1, b_1] \times \dots \times [a_d, b_d]$ разбивается на сетку элементарных гиперпрямоугольников. Пусть $N_k$ — число разбиений по $k$-й координате (четное), $h_k = (b_k - a_k)/N_k$ — шаг по $k$-й координате. Тогда интеграл аппроксимируется суммой:
\[
I \approx \left( \prod_{k=1}^{d} \frac{h_k}{3} \right) \sum_{j_d=0}^{N_d} \dots \sum_{j_1=0}^{N_1} \left( \prod_{k=1}^{d} C(j_k, N_k) \right) F(x_{1,j_1}, \dots, x_{d,j_d}),
\]
где $x_{k,j_k} = a_k + j_k h_k$, а $C(j, N)$ — коэффициенты Симпсона, определяемые функцией \texttt{SimpsonCoeff(j, N)}:
\[
C(j, N) =
\begin{cases}
1, & \text{if } j=0 \text{ or } j=N \\
4, & \text{if } j \text{ is odd} \\
2, & \text{if } j \text{ is even and } 0 < j < N
\end{cases}
\]

В качестве подынтегральных функций $F(x_1, \dots, x_d)$ используются две тестовые функции, выбираемые параметром \texttt{func\_code\_}:
\begin{enumerate}
    \item \texttt{func\_code\_ = 0}: $F(\mathbf{x}) = \sum_{k=1}^{d} x_k^2$
    \item \texttt{func\_code\_ = 1}: $F(\mathbf{x}) = \prod_{k=1}^{d} f_k(x_k)$, где $f_k(x_k) = \sin(x_k)$ если $k$ четное (0-индексация), и $f_k(x_k) = \cos(x_k)$ если $k$ нечетное.
\end{enumerate}

\subsection{Входные данные}
\label{ssec:input_data}
Для выполнения задачи требуются следующие входные данные:
\begin{itemize}
    \item \texttt{dimension\_ ($d$)}: размерность интеграла (целое число, $d \ge 1$).
    \item \texttt{a\_}: вектор вещественных чисел $\{\text{a}_1, \dots, \text{a}_d\}$, нижние пределы интегрирования по каждой размерности.
    \item \texttt{b\_}: вектор вещественных чисел $\{\text{b}_1, \dots, \text{b}_d\}$, верхние пределы интегрирования по каждой размерности.
    \item \texttt{n\_}: вектор целых чисел $\{\text{N}_1, \dots, \text{N}_d\}$, количество шагов разбиения по каждой размерности. Каждое $N_k$ должно быть положительным и четным.
    \item \texttt{func\_code\_}: код подынтегральной функции (целое число, 0 или 1).
\end{itemize}

\subsection{Выходные данные}
\label{ssec:output_data}
Результатом работы алгоритма является одно вещественное число — приближенное значение вычисленного многомерного интеграла.

\subsection{Критерии оценки}
\label{ssec:evaluation_criteria}
Эффективность реализованных параллельных версий оценивается по следующим критериям:
\begin{itemize}
    \item Время выполнения задачи (мс).
    \item Коэффициент ускорения (Speedup) относительно последовательной версии: $S_p = T_1 / T_p$, где $T_1$ — время выполнения последовательной программы, $T_p$ — время выполнения параллельной программы на $p$ вычислительных ресурсах (потоках/процессах).
    \item Эффективность использования вычислительных ресурсов: $E_p = S_p / p$.
\end{itemize}

\newpage

\section{Описание последовательного алгоритма}
\label{sec:sequential_algorithm}

Последовательная реализация вычисления многомерного интеграла методом Симпсона основана на рекурсивном подходе к обходу узлов многомерной сетки.

Основная вычислительная логика сосредоточена в рекурсивной функции \texttt{RecursiveSimpsonSum}. Эта функция принимает текущий индекс размерности \texttt{dim\_index}, вектор текущих индексов узлов по каждой размерности \texttt{idx} и вектор шагов \texttt{steps}.

\textbf{Алгоритм работы \texttt{RecursiveSimpsonSum}:}
\begin{enumerate}
    \item \textbf{Базовый случай рекурсии:} Если \texttt{dim\_index} равен общей размерности \texttt{dimension\_}, это означает, что все индексы узлов \texttt{idx} для текущей точки сетки определены. В этом случае:
        \begin{enumerate}
            \item Вычисляются координаты точки $(x_1, \dots, x_d)$ по формуле $x_k = a_k + \text{idx}_k \cdot \text{steps}_k$.
            \item Рассчитывается произведение коэффициентов Симпсона $C_{prod} = \prod_{k=0}^{d-1} \text{SimpsonCoeff}(\text{idx}_k, n_k)$.
            \item Вычисляется значение подынтегральной функции $F(x_1, \dots, x_d)$ в данной точке.
            \item Функция возвращает $C_{prod} \cdot F(x_1, \dots, x_d)$.
        \end{enumerate}
    \item \textbf{Шаг рекурсии:} Если \texttt{dim\_index < dimension\_}, функция итерируется по всем узлам текущей размерности \texttt{dim\_index} от $0$ до $n_{\text{dim\_index}}$:
        \begin{enumerate}
            \item Устанавливается \texttt{idx[dim\_index] = i}, где $i$ — текущий индекс узла.
            \item Рекурсивно вызывается \texttt{RecursiveSimpsonSum} для следующей размерности (\texttt{dim\_index + 1}).
            \item Результаты рекурсивных вызовов суммируются.
        \end{enumerate}
    \item Функция возвращает накопленную сумму.
\end{enumerate}

Основная функция \texttt{RunImpl} последовательной версии выполняет следующие шаги:
\begin{enumerate}
    \item Рассчитывает шаги интегрирования \texttt{steps[k] = (b[k] - a[k]) / n[k]} для каждой размерности $k$.
    \item Инициализирует вектор индексов \texttt{idx} нулями.
    \item Вызывает \texttt{RecursiveSimpsonSum(0, idx, steps)} для получения взвешенной суммы значений функции.
    \item Умножает полученную сумму на общий коэффициент $K = \prod_{k=0}^{d-1} (\text{steps}[k] / 3.0)$.
    \item Сохраняет итоговое значение как результат интегрирования.
\end{enumerate}

Фрагмент кода, иллюстрирующий рекурсивную функцию:
\begin{lstlisting}[language=C++, caption=Функция RecursiveSimpsonSum (ops\_seq.cpp), basicstyle=\ttfamily\scriptsize]
double IntegralsSimpsonSequential::RecursiveSimpsonSum(int dim_index, std::vector<int>& idx,
                                                       const std::vector<double>& steps) const {
  if (dim_index == dimension_) {
    double coeff = 1.0;
    std::vector<double> coords(dimension_);
    for (int d = 0; d < dimension_; ++d) {
      coords[d] = a_[d] + idx[d] * steps[d];
      coeff *= SimpsonCoeff(idx[d], n_[d]);
    }
    return coeff * FunctionN(coords);
  }
  double sum = 0.0;
  for (int i = 0; i <= n_[dim_index]; ++i) {
    idx[dim_index] = i;
    sum += RecursiveSimpsonSum(dim_index + 1, idx, steps);
  }
  return sum;
}
\end{lstlisting}

\newpage
\section{Общая схема распараллеливания}
\label{sec:parallel_scheme}

Задача вычисления многомерного интеграла методом Симпсона представляет собой суммирование большого числа слагаемых, каждое из которых соответствует взвешенному значению подынтегральной функции в узле сетки. Вычисление каждого такого слагаемого независимо от других, что создает предпосылки для эффективного распараллеливания.

В представленных реализациях используются две основные стратегии декомпозиции задачи:

\begin{enumerate}
    \item \textbf{Распараллеливание внешнего цикла интегрирования с сохранением рекурсии для внутренних размерностей.}
    Эта стратегия применяется в реализациях на OpenMP и STL. Итерации по узлам первой (внешней) размерности распределяются между доступными потоками. Каждый поток затем выполняет рекурсивный расчет суммы для своей части узлов первой размерности и всех узлов по оставшимся (внутренним) размерностям.
    \begin{itemize}
        \item \textbf{Преимущества:} Относительная простота модификации исходного рекурсивного кода. Сохранение структуры рекурсивного обхода для части задачи.
        \item \textbf{Недостатки:} Эффективность может зависеть от количества узлов $N_0$ по первой размерности. Если $N_0$ мало, степень параллелизма будет ограничена. Возможен дисбаланс нагрузки, если время вычисления для разных "веток" рекурсии сильно отличается (хотя в данном случае оно примерно одинаково).
    \end{itemize}

    \item \textbf{Преобразование многомерной сетки узлов в одномерное итерационное пространство и его параллельная обработка.}
    Эта стратегия используется в реализациях на TBB и гибридной MPI+TBB. Все узлы $d$-мерной сетки нумеруются одним одномерным индексом $k$, пробегающим значения от $0$ до $K_{total}-1$, где $K_{total} = \prod_{i=0}^{d-1} (N_i+1)$ — общее число узлов сетки. Этот одномерный диапазон итераций затем распределяется между потоками (TBB) или процессами (MPI). Внутри каждого параллельного задания по одномерному индексу $k$ восстанавливаются исходные многомерные индексы $(j_0, j_1, \dots, j_{d-1})$ для вычисления координат точки и соответствующих коэффициентов Симпсона.
    \begin{itemize}
        \item \textbf{Преимущества:} Позволяет добиться высокой степени гранулярности и потенциально лучшей балансировки нагрузки, так как работа делится на большее число мелких независимых частей. Упрощает распределение работы в гетерогенных системах (MPI).
        \item \textbf{Недостатки:} Требует дополнительных вычислений для преобразования одномерного индекса в многомерные и обратно. Может привести к менее предсказуемому доступу к данным по сравнению с последовательным обходом измерений, что потенциально влияет на кэширование (хотя для данной задачи это, вероятно, некритично).
    \end{itemize}
\end{enumerate}
Обе стратегии нацелены на распределение вычислений слагаемых суммы между параллельными исполнителями с последующей агрегацией (суммированием) частичных результатов.

\newpage
\section{Реализация на OpenMP}
\label{sec:omp_implementation}

Реализация вычисления многомерного интеграла методом Симпсона с использованием технологии OpenMP основана на стратегии распараллеливания внешнего цикла интегрирования. Это позволяет относительно просто модифицировать последовательный рекурсивный алгоритм для параллельного выполнения на системах с общей памятью.

Основная идея заключается в том, чтобы итерации цикла по первой размерности (индекс \texttt{i0} от 0 до \texttt{n\_[0]}) распределялись между потоками OpenMP. Каждый поток вычисляет свою часть общей суммы.

\textbf{Ключевые аспекты реализации:}
\begin{enumerate}
    \item Директива \texttt{\#pragma omp parallel} используется для создания параллельной области, в которой будет выполняться работа несколькими потоками.
    \item Директива \texttt{\#pragma omp for reduction(+: total\_sum)} применяется к циклу по первой размерности. Она автоматически распределяет итерации этого цикла между потоками. Переменная \texttt{total\_sum} объявляется как переменная редукции для операции сложения (+), что означает, что каждый поток будет иметь свою локальную копию этой переменной, а по завершении цикла все локальные значения будут корректно просуммированы в исходную переменную \texttt{total\_sum}.
    \item Внутри цикла каждый поток инициализирует свой локальный вектор индексов \texttt{idx}, где \texttt{idx[0]} устанавливается равным текущему значению \texttt{i0} из внешнего цикла.
    \item Если размерность интеграла \texttt{dimension\_} равна 1, то вычисление коэффициента Симпсона и значения функции производится непосредственно в этом цикле.
    \item Если \texttt{dimension\_ > 1}, то для вычисления суммы по оставшимся размерностям вызывается та же рекурсивная функция \texttt{RecursiveSimpsonSum}, но начиная со второй размерности (\texttt{dim\_index = 1}). Результат этого вызова добавляется к локальной сумме потока.
    \item После завершения параллельного цикла и редукции, итоговая \texttt{total\_sum} умножается на общий коэффициент, включающий шаги интегрирования, для получения окончательного результата.
\end{enumerate}

Такой подход позволяет эффективно использовать многоядерные процессоры, минимизируя изменения в исходной логике рекурсивного суммирования для внутренних размерностей.

Фрагмент кода \texttt{RunImpl} для OpenMP-версии:
\begin{lstlisting}[language=C++, caption=Фрагмент RunImpl для OMP-реализации (ops\_omp.cpp), basicstyle=\ttfamily\scriptsize]
bool IntegralsSimpsonOmp::RunImpl() {
  std::vector<double> steps(dimension_);
  for (int i = 0; i < dimension_; i++) {
    steps[i] = (b_[i] - a_[i]) / n_[i];
  }

  double total_sum = 0.0;

#pragma omp parallel
  {
    std::vector<int> idx(dimension_);

#pragma omp for reduction(+ : total_sum)
    for (int i0 = 0; i0 <= n_[0]; ++i0) {
      idx[0] = i0;

      if (dimension_ == 1) {
        double coeff = SimpsonCoeff(idx[0], n_[0]);
        std::vector<double> coords = {a_[0] + (idx[0] * steps[0])};
        total_sum += coeff * FunctionN(coords);
      } else {
        total_sum += RecursiveSimpsonSum(1, idx, steps);
      }
    }
  }

  double coeff_mult = 1.0;
  for (int i = 0; i < dimension_; i++) {
    coeff_mult *= steps[i] / 3.0;
  }

  result_ = coeff_mult * total_sum;
  return true;
}
\end{lstlisting}
\textit{Примечание: В представленном коде \texttt{ops\_omp.cpp} вызов \texttt{RecursiveSimpsonSum} напрямую добавляет к \texttt{total\_sum}, которая является reduction-переменной. OpenMP корректно обрабатывает это, суммируя значения, возвращаемые \texttt{RecursiveSimpsonSum} для каждой итерации \texttt{i0}, в частные суммы потоков, которые затем объединяются.}

\newpage
\section{Реализация на TBB}
\label{sec:tbb_implementation}

Реализация с использованием Intel Threading Building Blocks (TBB) применяет стратегию преобразования многомерной сетки узлов в одномерное итерационное пространство. Это позволяет использовать мощный механизм \texttt{tbb::parallel\_reduce} для эффективного параллельного суммирования.

\textbf{Ключевые аспекты реализации:}
\begin{enumerate}
    \item \textbf{Вычисление общего числа узлов:} Сначала рассчитывается общее количество узлов сетки \texttt{total\_points} как произведение $(n_k+1)$ по всем размерностям $k$. Это число определяет размер одномерного диапазона, который будет обрабатываться.
    \item \textbf{Параллельное суммирование с \texttt{tbb::parallel\_reduce}:}
        Основная вычислительная работа выполняется с помощью \texttt{tbb::parallel\_reduce}.
        Эта функция принимает:
        \begin{itemize}
            \item Диапазон для обработки: \texttt{tbb::blocked\_range<size\_t>(0, total\_points)}. TBB автоматически разбивает этот диапазон на меньшие поддиапазоны (чанки) для параллельной обработки потоками.
            \item Начальное значение для суммы (identity): \texttt{0.0}.
            \item Лямбда-функцию (body), вычисляющую частичную сумму для своего поддиапазона \texttt{r}:
                \begin{enumerate}
                    \item Для каждого одномерного индекса \texttt{k} из поддиапазона \texttt{r}:
                        \begin{itemize}
                            \item \textbf{Восстановление многомерных индексов:} Из одномерного индекса \texttt{k} восстанавливаются многомерные индексы \texttt{current\_idx[dim]} для каждого измерения. Это делается путем последовательного взятия остатка от деления \texttt{k} на $(n_{\text{dim}}+1)$ и последующего целочисленного деления \texttt{k} на $(n_{\text{dim}}+1)$.
                            \item Вычисляются координаты точки \texttt{coords[dim] = a\_[dim] + current\_idx[dim] * steps[dim]}.
                            \item Рассчитывается произведение коэффициентов Симпсона \texttt{current\_coeff\_prod} по всем измерениям на основе \texttt{current\_idx[dim]} и \texttt{n\_[dim]}.
                            \item Вычисляется значение подынтегральной функции \texttt{FunctionN(coords)}.
                            \item Слагаемое (\texttt{current\_coeff\_prod * FunctionN(coords)}) добавляется к текущей сумме поддиапазона (\texttt{running\_sum}).
                        \end{itemize}
                    \item Лямбда-функция возвращает \texttt{running\_sum}.
                \end{enumerate}
            \item Лямбда-функцию (reduction), которая суммирует частичные результаты, полученные от разных потоков: \texttt{[](double x, double y) \{ return x + y; \}}.
        \end{itemize}
    \item \textbf{Итоговый результат:} Сумма, полученная после \texttt{tbb::parallel\_reduce} (\texttt{total\_sum}), умножается на общий коэффициент $\prod (\text{steps}[k]/3.0)$ для получения значения интеграла.
\end{enumerate}
Этот подход позволяет TBB динамически балансировать нагрузку между потоками и эффективно использовать кэш-память за счет обработки смежных блоков данных.

Фрагмент кода \texttt{RunImpl} для TBB-версии:
\begin{lstlisting}[language=C++, caption=Фрагмент RunImpl для TBB-реализации (ops\_tbb.cpp), basicstyle=\ttfamily\scriptsize]
bool IntegralsSimpsonTBB::RunImpl() {
  std::vector<double> steps(dimension_);
  size_t total_points = 1;
  double coeff_mult = 1.0;

  for (int i = 0; i < dimension_; i++) {
    if (n_[i] == 0) return false;
    steps[i] = (b_[i] - a_[i]) / n_[i];
    coeff_mult *= steps[i] / 3.0;
    size_t points_in_dim = static_cast<size_t>(n_[i]) + 1;
    if (total_points > std::numeric_limits<size_t>::max() / points_in_dim) return false;
    total_points *= points_in_dim;
  }

  double total_sum = tbb::parallel_reduce(
      tbb::blocked_range<size_t>(0, total_points), 0.0,
      [&](const tbb::blocked_range<size_t>& r, double running_sum) {
        std::vector<double> coords(dimension_);
        std::vector<int> current_idx(dimension_);

        for (size_t k = r.begin(); k != r.end(); ++k) {
          double current_coeff_prod = 1.0;
          size_t current_k = k;

          for (int dim = 0; dim < dimension_; ++dim) {
            size_t points_in_this_dim = static_cast<size_t>(n_[dim]) + 1;
            size_t index_in_this_dim = current_k % points_in_this_dim;
            current_idx[dim] = static_cast<int>(index_in_this_dim);
            current_k /= points_in_this_dim;

            coords[dim] = a_[dim] + current_idx[dim] * steps[dim];
            current_coeff_prod *= SimpsonCoeff(current_idx[dim], n_[dim]);
          }
          running_sum += current_coeff_prod * FunctionN(coords);
        }
        return running_sum;
      },
      [](double x, double y) { return x + y; });

  result_ = coeff_mult * total_sum;
  return true;
}
\end{lstlisting}

\newpage
\section{Реализация на STL (std::thread)}
\label{sec:stl_implementation}

Реализация с использованием стандартных потоков C++ (\texttt{std::thread}) основана на ручном управлении потоками и распределении работы. Как и в OpenMP-версии, применяется стратегия распараллеливания внешнего цикла интегрирования (по первой размерности).

\textbf{Ключевые аспекты реализации:}
\begin{enumerate}
    \item \textbf{Определение количества потоков:} Функция \texttt{DetermineNumThreads} определяет оптимальное количество потоков для использования, обычно на основе количества доступных аппаратных потоков (\texttt{ppc::util::GetPPCNumThreads()}). Оно также ограничивается общим числом итераций.
    \item \textbf{Распределение итераций:} Функция \texttt{DistributeIterations} делит общее количество итераций внешнего цикла (от $0$ до $n_0$) на диапазоны, каждый из которых будет обрабатываться отдельным потоком. Распределение старается быть максимально равномерным.
    \item \textbf{Функция потока (\texttt{ThreadTaskRunner}):} Эта функция выполняется каждым созданным потоком.
        \begin{itemize}
            \item Принимает начальный (\texttt{start\_idx}) и конечный (\texttt{end\_idx}) индексы для итераций по первой размерности, ссылку на вектор шагов \texttt{steps} и указатель для сохранения частичной суммы \texttt{partial\_sum\_output}.
            \item В цикле от \texttt{start\_idx} до \texttt{end\_idx} (не включая \texttt{end\_idx}):
                \begin{itemize}
                    \item Инициализируется локальный вектор индексов \texttt{local\_idx}, где \texttt{local\_idx[0]} устанавливается равным текущей итерации внешнего цикла.
                    \item Вызывается рекурсивная функция \texttt{RecursiveSimpsonSum(1, local\_idx, steps)} для вычисления суммы по оставшимся размерностям (начиная со второй).
                    \item Результат добавляется к локальной частичной сумме потока.
                \end{itemize}
            \item По завершении цикла, накопленная локальная сумма записывается в \texttt{*partial\_sum\_output}.
        \end{itemize}
    \item \textbf{Создание и управление потоками в \texttt{RunImpl}:}
        \begin{itemize}
            \item Определяется количество потоков и диапазоны итераций.
            \item Создается вектор для хранения частичных сумм от каждого потока.
            \item Для каждого диапазона итераций создается объект \texttt{std::thread}, которому передается функция \texttt{ThreadTaskRunner} и ее аргументы.
            \item Если количество потоков равно 1, задача выполняется в текущем потоке без создания нового.
            \item После запуска всех потоков, главный поток ожидает их завершения с помощью \texttt{th.join()}.
        \end{itemize}
    \item \textbf{Сбор результатов:} После завершения всех потоков, частичные суммы из вектора \texttt{partial\_sums} суммируются для получения общей суммы.
    \item \textbf{Итоговый результат:} Общая сумма умножается на коэффициент $\prod (\text{steps}[k]/3.0)$.
\end{enumerate}
Этот подход дает полный контроль над созданием и синхронизацией потоков, но требует большего объема кода для управления параллелизмом по сравнению с OpenMP или TBB.

Фрагмент кода \texttt{RunImpl} и \texttt{ThreadTaskRunner} для STL-версии:
\begin{lstlisting}[language=C++, caption=Функция ThreadTaskRunner и фрагмент RunImpl для STL-реализации (ops\_stl.cpp), basicstyle=\ttfamily\scriptsize]
void IntegralsSimpsonSTL::ThreadTaskRunner(int start_idx, int end_idx, 
                                           const std::vector<double>& steps,
                                           double* partial_sum_output) {
  double local_partial_sum = 0.0;

  for (int i = start_idx; i < end_idx; ++i) {
    if (this->dimension_ < 1) {
      *partial_sum_output = 0.0; return;
    }
    std::vector<int> local_idx(dimension_);
    local_idx[0] = i;
    local_partial_sum += RecursiveSimpsonSum(1, local_idx, steps);
  }
  *partial_sum_output = local_partial_sum;
}

bool IntegralsSimpsonSTL::RunImpl() {
  std::vector<double> steps(dimension_);
  for (int i = 0; i < dimension_; i++) {
    if (n_[i] <= 0) return false;
    steps[i] = (b_[i] - a_[i]) / n_[i];
  }

  const int total_iterations_dim0 = n_[0] + 1;
  unsigned int num_actual_threads = DetermineNumThreads(total_iterations_dim0);
  if (num_actual_threads == 0 && total_iterations_dim0 > 0) num_actual_threads = 1;
  
  std::vector<double> partial_sums(num_actual_threads, 0.0);
  std::vector<std::thread> threads;

  if (num_actual_threads == 1 && total_iterations_dim0 > 0) {
    ThreadTaskRunner(0, total_iterations_dim0, steps, partial_sums.data());
  } else if (num_actual_threads > 1) {
    std::vector<IterationRange> ranges = DistributeIterations(total_iterations_dim0, num_actual_threads);
    
    for (size_t i = 0; i < ranges.size(); ++i) {
      const auto& range = ranges[i];
      threads.emplace_back(&IntegralsSimpsonSTL::ThreadTaskRunner, this, 
                           range.start, range.end, std::cref(steps), &partial_sums[i]);
    }
  }

  for (auto& th : threads) {
    if (th.joinable()) {
      th.join();
    }
  }

  double total_sum = 0.0;
  for (double p_sum : partial_sums) {
    total_sum += p_sum;
  }

  double coeff = 1.0;
  for (int i = 0; i < dimension_; i++) {
    coeff *= steps[i] / 3.0;
  }
  result_ = coeff * total_sum;
  return true;
}
\end{lstlisting}

\newpage
\section{Реализация гибридной MPI+TBB версии}
\label{sec:mpi_tbb_implementation}

Гибридная реализация MPI+TBB сочетает два уровня параллелизма: межпроцессный параллелизм с использованием MPI (Message Passing Interface) для систем с распределенной памятью и внутрипроцессный параллелизм с использованием Intel TBB для систем с общей памятью (в рамках одного MPI-процесса). Эта версия, как и чисто TBB-вариант, использует стратегию преобразования многомерной сетки узлов в одномерное итерационное пространство.

\textbf{Уровень MPI (межпроцессное взаимодействие):}
\begin{enumerate}
    \item \textbf{Предобработка (\texttt{PreProcessingImpl}):}
        \begin{itemize}
            \item Корневой процесс (ранг 0) считывает и валидирует входные данные с помощью вспомогательной функции \texttt{ParseAndValidateOnRoot}. Эта функция разбирает массив входных данных, извлекая размерность, пределы интегрирования, число разбиений и код функции.
            \item Статус успешности парсинга на корневом процессе транслируется всем остальным процессам с помощью \texttt{MPI\_Bcast}. Если парсинг неудачен, все процессы завершают предобработку с ошибкой.
            \item Основные параметры задачи (размерность \texttt{dimension\_}, код функции \texttt{func\_code\_}, векторы \texttt{a\_}, \texttt{b\_}, \texttt{n\_}) рассылаются от корневого процесса всем остальным процессам с использованием \texttt{MPI\_Bcast}.
            \item Обрабатывается случай нулевой размерности \texttt{dimension\_ == 0}, когда результат интеграла равен 0.
        \end{itemize}
    \item \textbf{Выполнение (\texttt{RunImpl}):}
        \begin{itemize}
            \item Каждый MPI-процесс (включая корневой) самостоятельно вычисляет шаги интегрирования \texttt{steps} и общее количество узлов сетки \texttt{params.total\_points} (как в TBB-версии) с помощью функции \texttt{CalculateRunParameters}.
            \item Общее одномерное итерационное пространство (от $0$ до \texttt{params.total\_points - 1}) делится между всеми MPI-процессами. Функция \texttt{DistributeWorkAmongMpiRanks} определяет для каждого процесса его локальный диапазон индексов (\texttt{dist.local\_start\_k}, \texttt{dist.local\_end\_k}) и количество точек для обработки (\texttt{dist.num\_points\_for\_this\_rank}).
            \item Каждый MPI-процесс вычисляет свою частичную сумму \texttt{local\_sum} для назначенного ему диапазона точек, используя внутри себя параллелизм TBB (см. ниже).
            \item Частичные суммы \texttt{local\_sum} со всех MPI-процессов собираются на корневом процессе (ранг 0) с помощью операции редукции \texttt{MPI\_Reduce(..., MPI\_SUM, ...)}, результат помещается в \texttt{global\_sum}.
            \item Корневой процесс вычисляет итоговый результат интеграла, умножая \texttt{global\_sum} на общий коэффициент \texttt{params.coeff\_mult}.
        \end{itemize}
    \item \textbf{Постобработка (\texttt{PostProcessingImpl}):}
        \begin{itemize}
            \item Только корневой процесс записывает вычисленный результат \texttt{result\_} в выходные данные.
        \end{itemize}
\end{enumerate}

\textbf{Уровень TBB (внутрипроцессный параллелизм):}
\begin{enumerate}
    \item Внутри каждого MPI-процесса, для обработки назначенного ему диапазона одномерных индексов [\texttt{dist.local\_start\_k}, \texttt{dist.local\_end\_k}), используется \texttt{tbb::parallel\_reduce}.
    \item Логика работы \texttt{tbb::parallel\_reduce} здесь аналогична той, что используется в чисто TBB-версии:
        \begin{itemize}
            \item Диапазон для \texttt{parallel\_reduce} — это \texttt{tbb::blocked\_range<size\_t>(dist.local\_start\_k, dist.local\_end\_k)}.
            \item Лямбда-функция (body) для каждого одномерного индекса \texttt{k\_iter} из этого диапазона восстанавливает многомерные индексы, вычисляет координаты точки, произведение коэффициентов Симпсона и значение подынтегральной функции, после чего добавляет слагаемое к текущей локальной сумме \texttt{running\_sum}.
            \item Лямбда-функция (reduction) суммирует частичные результаты от TBB-потоков внутри одного MPI-процесса.
        \end{itemize}
    \item Результатом работы TBB-части в каждом MPI-процессе является \texttt{local\_sum}, которая затем участвует в \texttt{MPI\_Reduce}.
\end{enumerate}

Этот гибридный подход позволяет масштабировать вычисления как на многоядерные узлы (через TBB), так и на множество таких узлов в кластере (через MPI).

Фрагмент кода \texttt{RunImpl} для MPI+TBB версии:
\begin{lstlisting}[language=C++, caption=Фрагмент RunImpl для MPI+TBB-реализации (ops\_all.cpp), basicstyle=\ttfamily\scriptsize]
bool IntegralsSimpsonAll::RunImpl() {
  int rank = 0;
  int world_size = 1;
  MPI_Comm_rank(MPI_COMM_WORLD, &rank);
  MPI_Comm_size(MPI_COMM_WORLD, &world_size);

  if (dimension_ == 0) {
    if (rank == 0) { result_ = 0.0; }
    return true;
  }

  std::vector<double> steps;
  RunParameters params = CalculateRunParameters(dimension_, n_, a_, b_, steps);
  if (!params.success) return false;
  if (params.total_points == 0) { /* ... */ }

  MpiWorkDistribution dist = DistributeWorkAmongMpiRanks(params.total_points, rank, world_size);

  double local_sum = 0.0;
  if (dist.num_points_for_this_rank > 0) {
    local_sum = tbb::parallel_reduce(
        tbb::blocked_range<size_t>(dist.local_start_k, dist.local_end_k), 0.0,
        [&](const tbb::blocked_range<size_t>& r, double running_sum) {
          std::vector<double> coords(dimension_);
          std::vector<int> current_idx(dimension_);
          for (size_t k_iter = r.begin(); k_iter != r.end(); ++k_iter) {
            double current_coeff_prod = 1.0;
            size_t current_k_val = k_iter;
            for (int dim_idx = 0; dim_idx < dimension_; ++dim_idx) {
              size_t points_in_this_dim = static_cast<size_t>(n_[dim_idx]) + 1;
              size_t index_in_this_dim = current_k_val % points_in_this_dim;
              current_idx[dim_idx] = static_cast<int>(index_in_this_dim);
              current_k_val /= points_in_this_dim;
              coords[dim_idx] = a_[dim_idx] + current_idx[dim_idx] * steps[dim_idx];
              current_coeff_prod *= SimpsonCoeff(current_idx[dim_idx], n_[dim_idx]);
            }
            running_sum += current_coeff_prod * FunctionN(coords);
          }
          return running_sum;
        },
        [](double x, double y) { return x + y; });
  }

  double global_sum = 0.0;
  MPI_Reduce(&local_sum, &global_sum, 1, MPI_DOUBLE, MPI_SUM, 0, MPI_COMM_WORLD);

  if (rank == 0) {
    result_ = params.coeff_mult * global_sum;
  } else {
    result_ = 0.0;
  }
  return true;
}
\end{lstlisting}

\newpage
\section{Результаты экспериментов}
\label{sec:results}
Для оценки производительности реализованных алгоритмов вычисления многомерного интеграла методом Симпсона использовался следующий тестовый стенд:
\begin{itemize}
    \item Операционная система: \textbf Windows 10
    \item Процессор: \textbf Ryzen 7 5700X, 8 ядер, 16 потоков, 4 ГГц)
    \item Оперативная память: \textbf 32 ГБ DDR4 4000 МГц
    \item Компилятор: \textbf MSVC 2019
\end{itemize}

Тестирование проводилось на следующей задаче:
\begin{itemize}
    \item Размерность интеграла $d = \textbf 2$
    \item Пределы интегрирования: 
        \begin{itemize}
            \item для $k=0$ (первая размерность): $a_0 = 0.0$, $b_0 = 1.0$.
            \item для $k=1$ (вторая размерность): $a_1 = 0.0$, $b_1 = 1.0$.
        \end{itemize}
    \item Число разбиений: 
        \begin{itemize}
            \item для $k=0$ (первая размерность): $N_0 = 2000$.
            \item для $k=1$ (вторая размерность): $N_1 = 2000$.
        \end{itemize}
    \item Подынтегральная функция: \texttt{func\_code\_ = \textbf 0, что соответствует $F(x_0, x_1) = x_0^2 + x_1^2$.}
\end{itemize}
Аналитическое значение данного интеграла $\int_{0}^{1} \int_{0}^{1} (x_0^2 + x_1^2) \,dx_0 \,dx_1 = 2/3 \approx 0.666\dots$.

Время выполнения измерялось для вычислительной части алгоритма. Каждое измерение повторялось несколько раз, и бралось среднее или минимальное значение для уменьшения влияния случайных флуктуаций. Ускорение рассчитывалось относительно времени выполнения последовательной версии.

\begin{table}[H]
\centering
\footnotesize % Или \small для чуть большего шрифта
\caption{Сравнение эффективности параллельных реализаций метода Симпсона}
\label{tab:performance_results}

\begin{tabularx}{\textwidth}{|l|X|S[table-format=5.2]|S[table-format=2.2]|l|}
\hline
\textbf{Реализация} & \textbf{Конфигурация (потоки/процессы)} & {\textbf{Время(pipeline), мс}} & \textbf{Время(task), мс} &{\textbf{Ускорение}}\\
\hline
Последовательная & 1 поток & \textbf {2340} & \textbf {2487} & \textbf {1.0} \\
\hline
\multirow{4}{*}{OpenMP} 
  & 2 потока & \textbf {1405} & \textbf {1476} & \textbf {1.68} \\
  & 4 потока & \textbf {727} & \textbf {852} & \textbf {2.92} \\
  & 8 потоков & \textbf {626} & \textbf {750} & \textbf {3.31} \\
\hline
\multirow{4}{*}{TBB} 
  & 2 потока & \textbf {141} & \textbf {300} & \textbf{8.29} \\
  & 4 потока & \textbf {96} & \textbf {175} & \textbf{14.2} \\
  & 8 потоков & \textbf{61} & \textbf{54} & \textbf{46} \\
\hline
\multirow{4}{*}{STL} 
  & 2 потока & \textbf {1352} & \textbf {1410} & \textbf {1.76} \\
  & 4 потока & \textbf {869} & \textbf {1046} & \textbf {2.38} \\
  & 8 потоков & \textbf {527} & \textbf {619} & \textbf {4.02} \\
\hline
\multirow{5}{*}{MPI + TBB}
  & 1 MPI пр. + 1 TBB пот. & \textbf{208} & \textbf{270} & \textbf{9.21} \\
  & 1 MPI пр. + 2 TBB пот. & \textbf{122} & \textbf{221} & \textbf{11.25} \\
  & 2 MPI пр. + 1 TBB пот. & \textbf{106} & \textbf{117} & \textbf{21.26} \\
  & 4 MPI пр. + 4 TBB пот. & \textbf{37} & \textbf{65} & \textbf{38.26} \\
\hline
\end{tabularx}
\end{table}

\section*{Анализ результатов}

На основе данных, представленных в Таблице \ref{tab:performance_results}, проведен анализ эффективности различных параллельных реализаций метода Симпсона. Ускорение рассчитывалось относительно времени выполнения последовательной реализации в режиме "task" (2487 мс).

\subsection*{Сравнение технологий на системах с общей памятью (OpenMP, TBB, STL)}

\textbf{Какая из технологий (OpenMP, TBB, STL) показала наилучшее ускорение на системе с общей памятью и при каком количестве потоков?}

На системе с общей памятью наилучшее ускорение продемонстрировала технология \textbf{TBB}. При использовании 8 потоков TBB достигла ускорения в \textbf{46 раз} (время выполнения "task" составило 54 мс). Это значительно превосходит результаты OpenMP (3.31x на 8 потоках) и STL (4.02x на 8 потоках). Даже при 2 потоках TBB (8.29x) показывает лучшее ускорение, чем OpenMP и STL на 8 потоках.

\textbf{Сравнение простоты реализации и достигнутой производительности для OpenMP, TBB, STL.}

\begin{itemize}
    \item \textbf{OpenMP}: Является технологией, основанной на директивах компилятора. Это часто обеспечивает относительно простую интеграцию параллелизма в существующий последовательный код, особенно для циклов. В данном тесте OpenMP показала умеренное ускорение, достигающее 3.31x на 8 потоках. Это может указывать либо на ограничения самой технологии для данной задачи, либо на то, что структура задачи не идеально подходит для простого распараллеливания циклами OpenMP, либо на значительные накладные расходы.
    \item \textbf{TBB}: Представляет собой библиотеку C++, предлагающую более гибкие и мощные инструменты для параллельного программирования, включая параллельные алгоритмы и графы задач. Реализация с TBB может быть сложнее, чем с OpenMP, требуя более явного описания параллельной логики. Однако, как показывают результаты, TBB обеспечила наивысшую производительность (ускорение до 46x), что свидетельствует об эффективном использовании ресурсов и низких накладных расходах для данной задачи. Особенно впечатляет скачок производительности между 4 и 8 потоками.
    \item \textbf{STL (параллельные алгоритмы C++)}: Стандартные параллельные алгоритмы C++ (предположительно, C++17) предлагают самый высокоуровневый и простой способ распараллеливания стандартных операций. Интеграция обычно минимальна. В данном тесте STL показала ускорение до 4.02x на 8 потоках, что лучше OpenMP, но значительно уступает TBB. Это может быть связано с более общим характером реализации параллельных алгоритмов STL, которые могут быть не так тонко настроены под специфику задачи или аппаратной платформы, как это возможно с TBB.
\end{itemize}
Таким образом, TBB демонстрирует наилучшее соотношение производительности, хотя и может потребовать больших усилий при реализации по сравнению с OpenMP или STL. STL предлагает наибольшую простоту с приемлемой производительностью для некоторых сценариев.

\subsection*{Эффективность гибридной реализации MPI+TBB}

\textbf{Насколько эффективной оказалась гибридная MPI+TBB реализация? Как она масштабируется с увеличением числа MPI-процессов и TBB-потоков?}

Гибридная реализация MPI+TBB показала высокую эффективность:
\begin{itemize}
    \item Уже конфигурация \textbf{1 MPI пр. + 1 TBB пот.} (9.21x) превосходит OpenMP и STL на 8 потоках и TBB на 2 потоках.
    \item Увеличение количества TBB потоков в рамках одного MPI процесса (с 1 MPI + 1 TBB до 1 MPI + 2 TBB) привело к увеличению ускорения с 9.21x до 11.25x.
    \item Увеличение количества MPI процессов при фиксированном количестве TBB потоков на процесс (с 1 MPI + 1 TBB до 2 MPI + 1 TBB) дало значительный прирост ускорения с 9.21x до 21.26x. Это указывает на хорошую масштабируемость по MPI процессам.
    \item Наилучшая гибридная конфигурация \textbf{4 MPI пр. + 4 TBB пот.} достигла ускорения \textbf{38.26x}. Это очень высокий результат, сопоставимый с лучшим результатом TBB на 8 потоках (46x).
\end{itemize}
Масштабируемость гибридной модели выглядит хорошо:
\begin{itemize}
    \item При увеличении числа MPI процессов (например, переход от 1 MPI + X TBB к 2 MPI + X TBB) наблюдается существенный рост производительности.
    \item При увеличении числа TBB потоков внутри MPI процесса также наблюдается рост, хотя он может быть ограничен доступными ядрами для одного процесса.
\end{itemize}
Интересно, что конфигурация TBB на 8 потоках (46x) оказалась немного эффективнее, чем 4 MPI процесса с 4 TBB потоками в каждом (суммарно 16 "потоков" работы, 38.26x). Это может быть связано с накладными расходами на MPI-коммуникации или с тем, что задача очень хорошо ложится на модель разделяемой памяти TBB при достаточном количестве потоков в одном процессе, и разделение на MPI процессы не дает дополнительного выигрыша сверх этого на данной системе.

\section*{Общий вывод}
Технология TBB показала себя наиболее эффективной для данной задачи на системе с общей памятью, обеспечив наивысшее ускорение. Гибридная модель MPI+TBB также продемонстрировала отличные результаты, что делает её перспективной для масштабирования на системы с распределенной памятью или для более крупных задач. OpenMP и параллельные алгоритмы STL показали более скромные результаты, но могут быть предпочтительны в случаях, когда простота реализации является ключевым фактором. Выбор оптимальной технологии зависит от конкретных требований к производительности, сложности задачи и доступных ресурсов разработки.

В ходе выполнения данной лабораторной работы была успешно решена задача вычисления многомерных интегралов с использованием составной квадратурной формулы Симпсона. Были разработаны и реализованы пять версий алгоритма: последовательная, а также параллельные версии с использованием технологий OpenMP, Intel TBB, стандартных потоков C++ (STL) и гибридного подхода MPI+TBB.

Основное внимание было уделено исследованию эффективности различных подходов к распараллеливанию вычислений. Проведенные эксперименты позволили выявить сильные и слабые стороны каждой из рассмотренных технологий применительно к данной задаче. Было показано, что применение параллельных вычислений может существенно сократить время, необходимое для численного интегрирования, особенно для задач высокой размерности или с большим числом узлов сетки.

Работа продемонстрировала, что:
\begin{itemize}
    \item Задача численного интегрирования методом Симпсона хорошо поддается распараллеливанию благодаря независимости вычислений в большинстве узлов сетки.
    \item Технологии OpenMP и Intel TBB предоставляют удобные высокоуровневые инструменты для параллелизации на системах с общей памятью, при этом TBB часто предлагает более гибкие механизмы управления задачами и балансировки нагрузки.
    \item Реализация с использованием STL (std::thread) дает полный контроль над потоками, но требует более значительных усилий по разработке и отладке кода.
    \item Гибридный подход MPI+TBB является мощным средством для достижения масштабируемости на многопроцессорных и кластерных системах, сочетая преимущества распределенного и разделяемого параллелизма.
\end{itemize}

\newpage
\section{Список литературы}
\label{sec:references}
\begin{enumerate}
    \bibitem{simpson_wiki} Метод Симпсона // Википедия. [Электронный ресурс]. URL: \url{https://ru.wikipedia.org/wiki/Формула_симпсона}.
    \item Бахвалов Н.С., Жидков Н.П., Кобельков Г.М. \textit{Численные методы}. — 8-е изд. — М.: БИНОМ. Лаборатория знаний, 2015. — 636 с.
    \bibitem{openmp_spec} OpenMP Architecture Review Board. OpenMP Application Programming Interface. [Электронный ресурс]. URL: \url{https://www.openmp.org/specifications/}.
    \bibitem{tbb_docs} Intel® oneAPI Threading Building Blocks (oneTBB) Documentation. [Электронный ресурс]. URL: \url{https://oneapi-src.github.io/oneTBB/}.
    \bibitem{cpp_thread} \texttt{std::thread} — cppreference.com. [Электронный ресурс]. URL: \url{https://en.cppreference.com/w/cpp/thread/thread}.
    \bibitem{mpi_forum} MPI Forum. MPI Standard. [Электронный ресурс]. URL: \url{https://www.mpi-forum.org/docs/}.
    \bibitem{voevodin} Воеводин В.В., Воеводин Вл.В. \textit{Параллельные вычисления}. — СПб.: БХВ-Петербург, 2002. — 608 с.
    \item Гергель В.П. \textit{Теория и практика параллельных вычислений}. — М.: Интернет-Университет Информационных Технологий; БИНОМ. Лаборатория знаний, 2007. — 423 с.
    \item Антонов А.С. \textit{Параллельное программирование с использованием технологии MPI}. Учебное пособие. — М.: Издательство Московского университета, 2004. — 71 с.
\end{enumerate}

\newpage
\section{Приложение. Листинги ключевых фрагментов кода}
\label{sec:appendix}

\subsection{ops\_seq.cpp}
\begin{lstlisting}[language=C++, caption=Ключевые функции из ops\_seq.cpp, basicstyle=\ttfamily\tiny]
namespace {
int SimpsonCoeff(int i, int n) {
  if (i == 0 || i == n) {
    return 1;
  }
  if (i % 2 != 0) {
    return 4;
  }
  return 2;
}
} // namespace

namespace anufriev_d_integrals_simpson_seq {

double IntegralsSimpsonSequential::FunctionN(const std::vector<double>& coords) const {
  switch (func_code_) {
    case 0: {
      double s = 0.0;
      for (double c : coords) {
        s += c * c;
      }
      return s;
    }
    case 1: {
      double val = 1.0;
      for (size_t i = 0; i < coords.size(); i++) {
        if (i % 2 == 0) {
          val *= std::sin(coords[i]);
        } else {
          val *= std::cos(coords[i]);
        }
      }
      return val;
    }
    default:
      return 0.0;
  }
}

double IntegralsSimpsonSequential::RecursiveSimpsonSum(int dim_index, std::vector<int>& idx,
                                                       const std::vector<double>& steps) const {
  if (dim_index == dimension_) {
    double coeff = 1.0;
    std::vector<double> coords(dimension_);
    for (int d = 0; d < dimension_; ++d) {
      coords[d] = a_[d] + idx[d] * steps[d];
      coeff *= SimpsonCoeff(idx[d], n_[d]);
    }
    return coeff * FunctionN(coords);
  }
  double sum = 0.0;
  for (int i = 0; i <= n_[dim_index]; ++i) {
    idx[dim_index] = i;
    sum += RecursiveSimpsonSum(dim_index + 1, idx, steps);
  }
  return sum;
}

bool IntegralsSimpsonSequential::RunImpl() {
  std::vector<double> steps(dimension_);
  for (int i = 0; i < dimension_; i++) {
    steps[i] = (b_[i] - a_[i]) / n_[i];
  }

  std::vector<int> idx(dimension_, 0);
  double sum = RecursiveSimpsonSum(0, idx, steps);

  double coeff = 1.0;
  for (int i = 0; i < dimension_; i++) {
    coeff *= steps[i] / 3.0;
  }

  result_ = coeff * sum;
  return true;
}
} // namespace anufriev_d_integrals_simpson_seq
\end{lstlisting}

\newpage
\subsection{ops\_omp.cpp}
\begin{lstlisting}[language=C++, caption=Функция RunImpl из ops\_omp.cpp, basicstyle=\ttfamily\tiny]

namespace anufriev_d_integrals_simpson_omp {

double IntegralsSimpsonOmp::RecursiveSimpsonSum(int dim_index, std::vector<int>& idx,
                                                const std::vector<double>& steps) const {
  if (dim_index == dimension_) {
    double coeff = 1.0;
    std::vector<double> coords(dimension_);
    for (int d = 0; d < dimension_; ++d) {
      coords[d] = a_[d] + idx[d] * steps[d];
      coeff *= SimpsonCoeff(idx[d], n_[d]); // Нужна видимость SimpsonCoeff
    }
    return coeff * FunctionN(coords);
  }
  double sum = 0.0;
  for (int i = 0; i <= n_[dim_index]; ++i) {
    idx[dim_index] = i;
    sum += RecursiveSimpsonSum(dim_index + 1, idx, steps);
  }
  return sum;
}


bool IntegralsSimpsonOmp::RunImpl() {
  std::vector<double> steps(dimension_);
  for (int i = 0; i < dimension_; i++) {
    steps[i] = (b_[i] - a_[i]) / n_[i];
  }

  double total_sum = 0.0;

#pragma omp parallel
  {
    std::vector<int> idx(dimension_);

#pragma omp for reduction(+ : total_sum)
    for (int i0 = 0; i0 <= n_[0]; ++i0) {
      idx[0] = i0;

      if (dimension_ == 1) {
        double coeff = SimpsonCoeff(idx[0], n_[0]);
        std::vector<double> coords = {a_[0] + (idx[0] * steps[0])};
        total_sum += coeff * FunctionN(coords);
      } else {
        total_sum += RecursiveSimpsonSum(1, idx, steps);
      }
    }
  }

  double coeff_mult = 1.0;
  for (int i = 0; i < dimension_; i++) {
    coeff_mult *= steps[i] / 3.0;
  }

  result_ = coeff_mult * total_sum;
  return true;
}
} // namespace anufriev_d_integrals_simpson_omp
\end{lstlisting}

\newpage
\subsection{ops\_tbb.cpp}
\begin{lstlisting}[language=C++, caption=Функция RunImpl из ops\_tbb.cpp, basicstyle=\ttfamily\tiny]

namespace anufriev_d_integrals_simpson_tbb {

bool IntegralsSimpsonTBB::RunImpl() {
  std::vector<double> steps(dimension_);
  size_t total_points = 1;
  double coeff_mult = 1.0;

  for (int i = 0; i < dimension_; i++) {
    if (n_[i] == 0) {
      return false;
    }
    steps[i] = (b_[i] - a_[i]) / n_[i];
    coeff_mult *= steps[i] / 3.0;
    size_t points_in_dim = static_cast<size_t>(n_[i]) + 1;
    if (total_points > std::numeric_limits<size_t>::max() / points_in_dim) {
      return false; 
    }
    total_points *= points_in_dim;
  }

  if (dimension_ > 0 && total_points == 0 && n_[0] >=0 ) {
  }


  double total_sum = tbb::parallel_reduce(
      tbb::blocked_range<size_t>(0, total_points), 0.0,
      [&](const tbb::blocked_range<size_t>& r, double running_sum) {
        std::vector<double> coords(dimension_);
        std::vector<int> current_idx(dimension_);

        for (size_t k = r.begin(); k != r.end(); ++k) {
          double current_coeff_prod = 1.0;
          size_t current_k = k;

          for (int dim = 0; dim < dimension_; ++dim) {
            size_t points_in_this_dim = static_cast<size_t>(n_[dim]) + 1;
            size_t index_in_this_dim = current_k % points_in_this_dim;
            current_idx[dim] = static_cast<int>(index_in_this_dim);
            current_k /= points_in_this_dim;

            coords[dim] = a_[dim] + current_idx[dim] * steps[dim];
            current_coeff_prod *= SimpsonCoeff(current_idx[dim], n_[dim]);
          }
          running_sum += current_coeff_prod * FunctionN(coords);
        }
        return running_sum;
      },
      [](double x, double y) { return x + y; });

  result_ = coeff_mult * total_sum;
  return true;
}
} // namespace anufriev_d_integrals_simpson_tbb
\end{lstlisting}

\newpage
\subsection{ops\_stl.cpp}
\begin{lstlisting}[language=C++, caption=Ключевые функции из ops\_stl.cpp, basicstyle=\ttfamily\tiny]

namespace anufriev_d_integrals_simpson_stl {

struct IterationRange { int start; int end; }; 

void IntegralsSimpsonSTL::ThreadTaskRunner(int start_idx, int end_idx, 
                                           const std::vector<double>& steps,
                                           double* partial_sum_output) {
  double local_partial_sum = 0.0;
  *partial_sum_output = 0.0;

  for (int i = start_idx; i < end_idx; ++i) {
    // if (this->dimension_ < 1) { return; }
    std::vector<int> local_idx(dimension_);
    local_idx[0] = i;
    local_partial_sum += RecursiveSimpsonSum(1, local_idx, steps);
  }
  *partial_sum_output = local_partial_sum;
}

bool IntegralsSimpsonSTL::RunImpl() {
  if (dimension_ < 1) { return false; }
  if (n_.empty() || n_[0] < 0) { return false; }

  std::vector<double> steps(dimension_);
  for (int i = 0; i < dimension_; i++) {
    if (n_[i] <= 0 || (n_[i] % 2) != 0) { return false; }
    steps[i] = (b_[i] - a_[i]) / n_[i];
  }

  const int total_iterations_dim0 = n_[0] + 1;
  unsigned int num_actual_threads = DetermineNumThreads(total_iterations_dim0);
  if (num_actual_threads == 0 && total_iterations_dim0 > 0) { num_actual_threads = 1;}

  std::vector<double> partial_sums(num_actual_threads, 0.0);
  std::vector<std::thread> threads;

  if (total_iterations_dim0 > 0) {
    if (num_actual_threads == 1) {
        ThreadTaskRunner(0, total_iterations_dim0, steps, partial_sums.data());
    } else if (num_actual_threads > 1) {
        std::vector<IterationRange> ranges = DistributeIterations(total_iterations_dim0, num_actual_threads);
        if (ranges.size() < num_actual_threads && !ranges.empty()) {
            partial_sums.resize(ranges.size());
        } else if (ranges.empty() && num_actual_threads > 0) {
            partial_sums.clear(); 
        }
        
        threads.resize(ranges.size());
        for (size_t i = 0; i < ranges.size(); ++i) {
            const auto& range = ranges[i];
            threads[i] = std::thread(&IntegralsSimpsonSTL::ThreadTaskRunner, this, 
                                     range.start, range.end, std::cref(steps), &partial_sums[i]);
        }
    }
  }

  for (auto& th : threads) {
    if (th.joinable()) {
      th.join();
    }
  }

  double total_sum = 0.0;
  for (double p_sum : partial_sums) {
    total_sum += p_sum;
  }

  double coeff = 1.0;
  for (int i = 0; i < dimension_; i++) {
    coeff *= steps[i] / 3.0;
  }
  result_ = coeff * total_sum;
  return true;
}
} // namespace anufriev_d_integrals_simpson_stl
\end{lstlisting}

\newpage
\subsection{ops\_all.cpp (MPI+TBB)}
\begin{lstlisting}[language=C++, caption=Ключевые функции из ops\_all.cpp, basicstyle=\ttfamily\tiny]
namespace { 
// struct ParsedRootInput { ... };
// ParsedRootInput ParseAndValidateOnRoot(...) { ... }
// int SimpsonCoeff(int i, int n) { ... }
// struct RunParameters { ... };
// RunParameters CalculateRunParameters(...) { ... }
// struct MpiWorkDistribution { ... };
// MpiWorkDistribution DistributeWorkAmongMpiRanks(...) { ... }
} // namespace

namespace anufriev_d_integrals_simpson_all {

bool IntegralsSimpsonAll::PreProcessingImpl() {
  int rank = 0;
  MPI_Comm_rank(MPI_COMM_WORLD, &rank);

  ParsedRootInput parsed_input_data;
  int root_preprocessing_status = 0;

  if (rank == 0) {
    parsed_input_data = ParseAndValidateOnRoot(task_data);
    root_preprocessing_status = parsed_input_data.parse_successful ? 1 : 0;
  }

  MPI_Bcast(&root_preprocessing_status, 1, MPI_INT, 0, MPI_COMM_WORLD);
  if (root_preprocessing_status == 0) return false;

  if (rank == 0) {
    dimension_ = parsed_input_data.dimension;
    func_code_ = parsed_input_data.func_code_val;
  }
  MPI_Bcast(&dimension_, 1, MPI_INT, 0, MPI_COMM_WORLD);
  MPI_Bcast(&func_code_, 1, MPI_INT, 0, MPI_COMM_WORLD);

  if (dimension_ == 0) {
    result_ = 0.0;
    return true;
  }

  a_.resize(dimension_); b_.resize(dimension_); n_.resize(dimension_);
  if (rank == 0) {
    a_ = parsed_input_data.a_vec;
    b_ = parsed_input_data.b_vec;
    n_ = parsed_input_data.n_vec;
  }
  MPI_Bcast(a_.data(), dimension_, MPI_DOUBLE, 0, MPI_COMM_WORLD);
  MPI_Bcast(b_.data(), dimension_, MPI_DOUBLE, 0, MPI_COMM_WORLD);
  MPI_Bcast(n_.data(), dimension_, MPI_INT, 0, MPI_COMM_WORLD);

  result_ = 0.0;
  return true;
}

bool IntegralsSimpsonAll::RunImpl() {
  int rank = 0; int world_size = 1;
  MPI_Comm_rank(MPI_COMM_WORLD, &rank);
  MPI_Comm_size(MPI_COMM_WORLD, &world_size);

  if (dimension_ == 0) {
    if (rank == 0) { result_ = 0.0; }
    return true;
  }

  std::vector<double> steps; 
  RunParameters params = CalculateRunParameters(dimension_, n_, a_, b_, steps);
  if (!params.success) return false;
  if (params.total_points == 0 && dimension_ > 0) {
      if(rank == 0) result_ = 0.0;
      return true;
  }


  MpiWorkDistribution dist = DistributeWorkAmongMpiRanks(params.total_points, rank, world_size);

  double local_sum = 0.0;
  if (dist.num_points_for_this_rank > 0) {
    local_sum = tbb::parallel_reduce(
        tbb::blocked_range<size_t>(dist.local_start_k, dist.local_end_k), 0.0,
        [&](const tbb::blocked_range<size_t>& r, double running_sum) {
          std::vector<double> coords(dimension_);
          std::vector<int> current_idx(dimension_);
          for (size_t k_iter = r.begin(); k_iter != r.end(); ++k_iter) {
            double current_coeff_prod = 1.0;
            size_t current_k_val = k_iter;
            for (int dim_idx = 0; dim_idx < dimension_; ++dim_idx) {
              size_t points_in_this_dim = static_cast<size_t>(n_[dim_idx]) + 1;
              size_t index_in_this_dim = current_k_val % points_in_this_dim;
              current_idx[dim_idx] = static_cast<int>(index_in_this_dim);
              current_k_val /= points_in_this_dim;
              coords[dim_idx] = a_[dim_idx] + current_idx[dim_idx] * steps[dim_idx];
              current_coeff_prod *= SimpsonCoeff(current_idx[dim_idx], n_[dim_idx]);
            }
            running_sum += current_coeff_prod * FunctionN(coords);
          }
          return running_sum;
        },
        [](double x, double y) { return x + y; });
  }

  double global_sum = 0.0;
  MPI_Reduce(&local_sum, &global_sum, 1, MPI_DOUBLE, MPI_SUM, 0, MPI_COMM_WORLD);

  if (rank == 0) {
    result_ = params.coeff_mult * global_sum;
  } else {
    result_ = 0.0; 
  }
  return true;
}
} // namespace anufriev_d_integrals_simpson_all
\end{lstlisting}

\end{document}