\documentclass[12pt]{article}
\usepackage[utf8]{inputenc}
\usepackage[russian]{babel}
\usepackage[T1]{fontenc}
\usepackage{amsmath}
\usepackage{geometry}
\usepackage{multirow}
\usepackage{titlesec}
\usepackage{hyperref}
\usepackage{tocloft}
\usepackage{xcolor} 
\usepackage{listings}
\usepackage{float}

\geometry{a4paper, left=25mm, right=15mm, top=20mm, bottom=20mm}

\hypersetup{
    colorlinks=true,
    linkcolor=black,
    urlcolor=black,
    pdftitle={Отчет по проекту},
    pdfauthor={Сорокин Андрей},
    pdfsubject={Умножение разреженных матриц. Элементы типа double. Формат хранения матрицы – столбцовый (CCS).}
}

\lstset{
  language=C++,
  basicstyle=\ttfamily\small,
  keywordstyle=\bfseries\color{blue},
  commentstyle=\itshape\color{green!50!black},
  stringstyle=\color{red!60!black},
  numbers=left,
  numberstyle=\tiny,
  stepnumber=1,
  numbersep=8pt,
  backgroundcolor=\color{gray!10},
  showspaces=false,
  showstringspaces=false,
  breaklines=true,
  frame=single,
  tabsize=2,
  captionpos=b
}

\renewcommand{\cftsecaftersnum}{.}
\renewcommand{\cftsecleader}{\cftdotfill{\cftdotsep}}
\titleformat{\section}{\large\bfseries}{\thesection.}{1em}{}

\begin{document}
\begin{titlepage}
    \centering
    \large
    Министерство науки и высшего образования Российской Федерации\\[0.5cm]
    Федеральное государственное автономное образовательное учреждение высшего образования\\[0.5cm]
    \textbf{«Национальный исследовательский Нижегородский государственный университет им. Н.И. Лобачевского»}\\
    (ННГУ)\\[1cm]
    Институт информационных технологий, математики и механики\\[0.5cm]
    Направление подготовки: \textbf{«Программная инженерия»}\\[2cm]

    \vfill
    {\LARGE \textbf{ОТЧЕТ}}\\[0.5cm]
    {\Large по задаче}\\[0.5cm]
    {\LARGE \textbf{«Умножение разреженных матриц. Элементы типа double. Формат хранения матрицы – столбцовый (CCS)»}}\\[0.5cm]
    {\Large \textbf{Вариант №5}}\\[2.5cm]

    \hfill\parbox{0.5\textwidth}{
        \textbf{Выполнил:} \\
        студент группы 3822Б1ПР2 \\
        \textbf{Сорокин Андрей}
    }\\[0.5cm]

    \hfill\parbox{0.5\textwidth}{
        \textbf{Преподаватели:} \\
        Нестеров А.Ю.
        Оболенский А.А.
        
    }\\[2cm]

    Нижний Новгород\\
    2025
\end{titlepage}


\thispagestyle{empty}
\clearpage
\pagenumbering{arabic} 
\setcounter{page}{2} 
\tableofcontents
\clearpage
\setcounter{page}{3} 
\section{Введение}

\hspace*{1.25em}Умножение разреженных матриц является одной из фундаментальных операций в вычислительной математике и имеет широкий спектр практических применений. Особую актуальность данная операция приобретает при работе с большими массивами данных, где большинство элементов матрицы равны нулю. Формат CCS (Compressed Column Storage) представляет собой эффективный метод хранения разреженных матриц, предложенный в начале 1970-х годов и получивший широкое распространение благодаря своей компактности и удобству выполнения матричных операций.

В современных реалиях операции с разреженными матрицами находят применение в различных областях, включая численное моделирование, машинное обучение, анализ социальных сетей, компьютерную графику и решение систем линейных уравнений. Особенно важную роль эти операции играют в методе конечных элементов, при моделировании физических процессов и в задачах оптимизации. Использование типа данных double обеспечивает необходимую точность вычислений, что критично для многих научных и инженерных приложений.

Несмотря на эффективность формата CCS, операция умножения разреженных матриц остается вычислительно сложной задачей, особенно при работе с матрицами большой размерности. В связи с этим особую актуальность приобретает разработка параллельных алгоритмов, способных эффективно использовать возможности современных многоядерных и распределенных вычислительных систем.

В рамках данной работы представлено исследование различных подходов к параллельной реализации алгоритма умножения разреженных матриц с использованием современных технологий параллельного программирования: OpenMP, Intel Threading Building Blocks (TBB), стандартной библиотеки потоков C++ и гибридной модели MPI + OpenMP. Каждый из этих подходов имеет свои особенности и преимущества, которые будут подробно рассмотрены в последующих разделах работы. Особое внимание будет уделено анализу эффективности различных параллельных реализаций и их сравнительной характеристике.

\section{Постановка задачи}

\hspace*{1.25em}Цель практической работы — разработка и исследование различных параллельных реализаций алгоритма умножения разреженных матриц в формате CCS с элементами типа double.

Для достижения поставленной цели необходимо выполнить следующие задачи:

\begin{itemize}
\item Реализовать последовательный вариант алгоритма умножения разреженных матриц в формате CCS;
\item Разработать параллельные версии с использованием следующих технологий:
\begin{itemize}
\item OpenMP — для эффективного использования систем с общей памятью;
\item Intel TBB — для реализации потокового параллелизма с динамической балансировкой нагрузки;
\item std::thread — для создания версии с явным управлением потоками;
\item MPI + OpenMP — для построения гибридной модели, сочетающей межпроцессорное взаимодействие с многопоточностью.
\end{itemize}
\item Провести тестирование и проверку корректности всех реализаций алгоритма;
\item Сравнить производительность последовательной и параллельных версий.
\begin{itemize}
\item Время выполнения на различных размерах матриц;
\item Масштабируемость при увеличении числа процессоров/потоков;
\end{itemize}
\end{itemize}

Ожидаемые результаты включают в себя определение наиболее эффективных подходов к распараллеливанию алгоритма умножения разреженных матриц, выявление факторов, влияющих на производительность различных реализаций.

\section{Описание алгоритма}

Пусть даны две матрицы: A размера M×K и B размера K×N. Результирующая матрица C будет иметь размер M×N.
Классическая формула умножения матриц:
\begin{lstlisting}
C[i][j] = Σ(k=0 to K-1) A[i][k] × B[k][j]
\end{lstlisting}
где:
\begin{enumerate}
\item i изменяется от 0 до M-1 (строки матрицы A)
\item j изменяется от 0 до N-1 (столбцы матрицы B)
\item k изменяется от 0 до K-1 (столбцы матрицы A / строки матрицы B)
\end{enumerate}
Каждый элемент результирующей матрицы C[i][j] получается как скалярное произведение i-й строки матрицы A на j-й столбец матрицы B.

Модификация для разреженных матриц в формате CCS:

При работе с разреженными матрицами классический алгоритм становится неэффективным, так как большинство операций производится с нулевыми элементами. Поэтому в представленной реализации используются следующие ключевые модификации:

Вместо работы со всеми элементами обрабатываются только ненулевые значения, хранящиеся в формате CCS.

Для каждого столбца результирующей матрицы:
\begin{enumerate}
\item Рассматриваются только ненулевые элементы соответствующего столбца матрицы B
\item Для каждого такого элемента берутся ненулевые элементы соответствующего столбца матрицы A
\item Результаты накапливаются во временном массиве
\end{enumerate}

Формат CCS позволяет эффективно получать доступ к элементам по столбцам, что учитывается в алгоритме при организации циклов обработки данных.

Представленная в коде реализация следует этим принципам, оптимизируя классический алгоритм для работы с разреженными структурами данных. Основное отличие от классического подхода заключается в том, что вместо перебора всех возможных комбинаций индексов обрабатываются только существующие ненулевые элементы, что существенно сокращает количество необходимых операций при работе с разреженными матрицами.

Такой подход особенно эффективен, когда большая часть элементов матриц равна нулю, что часто встречается в практических задачах, например, при решении систем линейных уравнений, в задачах оптимизации или при моделировании физических процессов.

\section{Описание параллельных реализаций алгоритма}

\subsection{OpenMP}

Алгоритм разделен на два основных параллельных этапа:

\begin{enumerate}
\item Первый проход - анализ структуры:
\begin{lstlisting}
#pragma omp parallel for
for (int j = 0; j < n; ++j) {
  std::vector<bool> temp_used(m, false);
  for (int t = b_col_ptr[j]; t < b_col_ptr[j + 1]; ++t) {
    int row_b = b_row_indices[t];
    for (int i = a_col_ptr[row_b]; i < a_col_ptr[row_b + 1]; ++i) {
      int row_a = a_row_indices[i];
      if (!temp_used[row_a]) {
        temp_used[row_a] = true;
        nnz_per_column[j]++;
      }
    }
  }
}
\end{lstlisting}
На этом этапе каждый поток:
\begin{enumerate}
\item Работает со своим столбцом результирующей матрицы
\item Использует локальный массив temp used для отслеживания уникальных строк
\item Подсчитывает количество ненулевых элементов в своём столбце
\item Не требует синхронизации с другими потоками
\end{enumerate}

\item Второй проход - вычисление значений:
\begin{lstlisting}
#pragma omp parallel for
for (int j = 0; j < n; ++j) {
  std::vector<double> temp_values(m, 0.0);
  std::vector<bool> temp_used(m, false);

  for (int t = b_col_ptr[j]; t < b_col_ptr[j + 1]; ++t) {
    int row_b = b_row_indices[t];
    double val_b = b_values[t];
    for (int i = a_col_ptr[row_b]; i < a_col_ptr[row_b + 1]; ++i) {
      int row_a = a_row_indices[i];
      temp_values[row_a] += a_values[i] * val_b;
      temp_used[row_a] = true;
    }
  }
\end{lstlisting}
\end{enumerate}

Особенности реализации:
\begin{enumerate}
\item Каждый поток имеет собственные буферные массивы
\item Отсутствие блокировок и критических секций
\item Эффективное использование кэша благодаря локальности данных
\item Результаты сразу записываются в финальные массивы без промежуточного хранения
\end{enumerate}

\subsection{Intel TBB}
Алгоритм разделен на два основных параллельных этапа:
\begin{enumerate}
\item Первый проход - анализ структуры:
\begin{lstlisting}
tbb::parallel_for(tbb::blocked_range<int>(0, n), & {
for (int j = range.begin(); j < range.end(); ++j) {
std::vector<bool> temp_used(m, false);
for (int t = b_col_ptr[j]; t < b_col_ptr[j + 1]; ++t) {
const int row_b = b_row_indices[t];
for (int i = a_col_ptr[row_b]; i < a_col_ptr[row_b + 1]; ++i) {
const int row_a = a_row_indices[i];
if (!temp_used[row_a]) {
temp_used[row_a] = true;
col_sizes[j]++;
}
}
}
}
});
\end{lstlisting}
На этом этапе каждый поток:
\begin{enumerate}
\item Обрабатывает блок столбцов результирующей матрицы
\item Использует локальный массив temp used для отслеживания уникальных строк
\item Подсчитывает количество ненулевых элементов для каждого столбца в своём блоке
\item Автоматически балансирует нагрузку через механизм work stealing
\end{enumerate}
\item Второй проход - вычисление значений:
\begin{lstlisting}
tbb::parallel_for(tbb::blocked_range<int>(0, n), & {
for (int j = range.begin(); j < range.end(); ++j) {
std::vector<double> temp_values(m, 0.0);
std::vector<bool> temp_used(m, false);
for (int t = b_col_ptr[j]; t < b_col_ptr[j + 1]; ++t) {
const int row_b = b_row_indices[t];
const double val_b = b_values[t];
for (int i = a_col_ptr[row_b]; i < a_col_ptr[row_b + 1]; ++i) {
const int row_a = a_row_indices[i];
temp_values[row_a] += a_values[i] * val_b;
temp_used[row_a] = true;
}
}
}
});
\end{lstlisting}
\end{enumerate}
Особенности реализации:
\begin{enumerate}
\item Блочное разделение работы с динамической балансировкой нагрузки
\item Каждый поток работает с независимым набором данных
\item Эффективное использование кэша благодаря обработке смежных столбцов
\item Отсутствие необходимости в явной синхронизации между потоками
\end{enumerate}
\subsection{STL Threads}
Алгоритм разделен на два основных параллельных этапа:
\begin{enumerate}
\item Первый проход - анализ структуры:
\begin{lstlisting}
std::atomic<int> next_j(0);
auto first_pass_worker = & {
int j = 0;
while ((j = next_j.fetch_add(1, std::memory_order_relaxed)) < n) {
std::vector<char> temp_used(m, 0);
for (int t = b_col_ptr[j]; t < b_col_ptr[j + 1]; ++t) {
const int row_b = b_row_indices[t];
for (int i = a_col_ptr[row_b]; i < a_col_ptr[row_b + 1]; ++i) {
const int row_a = a_row_indices[i];
if (temp_used[row_a] == 0) {
temp_used[row_a] = 1;
col_sizes[j]++;
}
}
}
}
};
\end{lstlisting}
На этом этапе каждый поток:
\begin{enumerate}
\item Динамически получает следующий столбец через атомарный счетчик
\item Использует оптимизированный массив char вместо bool
\item Подсчитывает количество ненулевых элементов в полученном столбце
\item Работает независимо до исчерпания всех столбцов
\end{enumerate}
\item Второй проход - вычисление значений:
\begin{lstlisting}
next_j.store(0);
auto second_pass_worker = & {
int j = 0;
while ((j = next_j.fetch_add(1, std::memory_order_relaxed)) < n) {
std::vector<double> temp_values(m, 0.0);
std::vector<char> temp_used(m, 0);
// Вычисление значений для столбца j
}
};
\end{lstlisting}
\end{enumerate}
Особенности реализации:
\begin{enumerate}
\item Динамическое распределение работы через атомарные операции
\item Оптимизация памяти через использование типа char
\item Эффективная балансировка нагрузки между потоками
\item Минимальные накладные расходы на синхронизацию
\end{enumerate}
\subsection{MPI + OpenMP}
Алгоритм сочетает распределенные вычисления с локальным параллелизмом:
\begin{enumerate}
\item Первый проход - анализ структуры:
\begin{lstlisting}
int local_cols = end_col - start_col;
std::vector<int> local_nnz(local_cols, 0);
#pragma omp parallel for
for (int j_local = 0; j_local < local_cols; ++j_local) {
int j = start_col + j_local;
std::vector<bool> temp_used(m, false);
for (int t = b_col_ptr[j]; t < b_col_ptr[j + 1]; ++t) {
int row_b = b_row_indices[t];
for (int i = a_col_ptr[row_b]; i < a_col_ptr[row_b + 1]; ++i) {
int row_a = a_row_indices[i];
if (!temp_used[row_a]) {
temp_used[row_a] = true;
local_nnz[j_local]++;
}
}
}
}
\end{lstlisting}
На этом этапе:
\begin{enumerate}
\item Матрица разделяется на вертикальные полосы между процессами
\item Каждый процесс параллельно обрабатывает свои столбцы через OpenMP
\item Локальные результаты собираются через MPI-операции
\item Обеспечивается двухуровневый параллелизм
\end{enumerate}
\item Второй проход - вычисление значений:
\begin{lstlisting}
#pragma omp parallel for
for (int j = 0; j < local_cols; ++j) {
std::vector<double> temp_values(m, 0.0);
std::vector<bool> temp_used(m, false);
// Вычисление значений для локального столбца
}
\end{lstlisting}
\end{enumerate}
Особенности реализации:
\begin{enumerate}
\item Эффективное распределение данных между процессами
\item Использование общей памяти внутри узла через OpenMP
\item Минимизация межпроцессных коммуникаций
\item Масштабируемость на кластерных системах
\end{enumerate}

\subsection{Сравнительный анализ реализаций}
В ходе работы были реализованы и исследованы четыре различных подхода к параллельному умножению разреженных матриц в формате CCS:
\begin{enumerate}
\item OpenMP реализация отличается:
\begin{itemize}
\item Простотой реализации и поддержки кода
\item Эффективным использованием систем с общей памятью
\item Минимальными накладными расходами на синхронизацию
\end{itemize}
\item Intel TBB реализация обеспечивает:
\begin{itemize}
\item Автоматическую балансировку нагрузки через механизм work stealing
\item Эффективную работу с неравномерным распределением данных
\item Гибкое управление размером блоков обработки
\end{itemize}
\item STL Threads реализация предоставляет:
\begin{itemize}
\item Точный контроль над распределением работы между потоками
\item Оптимизацию использования памяти
\item Эффективную динамическую балансировку через атомарные операции
\end{itemize}
\item Гибридная MPI + OpenMP реализация обеспечивает:
\begin{itemize}
\item Масштабируемость на распределенных системах
\item Эффективное использование иерархии памяти
\item Возможность обработки матриц большого размера
\end{itemize}
\end{enumerate}
Каждая реализация имеет свою оптимальную область применения:
\begin{itemize}
\item OpenMP лучше всего подходит для систем с общей памятью и равномерным распределением ненулевых элементов
\item Intel TBB эффективна для задач с неравномерной вычислительной нагрузкой
\item STL Threads оптимальна при необходимости тонкой настройки параллельной обработки
\item MPI + OpenMP незаменима для больших задач, требующих распределенных вычислений
\end{itemize}
Все реализации следуют общему принципу двухпроходного алгоритма:
\begin{enumerate}
\item Первый проход для анализа структуры и определения размеров результата
\item Второй проход для вычисления значений элементов
\end{enumerate}
Такой подход позволяет эффективно организовать параллельные вычисления, минимизировать накладные расходы на синхронизацию и оптимально использовать доступные вычислительные ресурсы.

\section{Результаты экспериментов}

\hspace*{1.25em}Для оценки производительности реализованных версий алгоритма умножения разреженных матриц были проведены замеры времени выполнения в конфигурациях с различным числом потоков и процессов. В таблице ниже представлены усреднённые значения времени выполнения (в секундах) для двух режимов тестирования — \texttt{PipelineRun} и \texttt{TaskRun}.

\subsection{Таблица производительности}

\renewcommand{\arraystretch}{1.4}
\begin{table}[H]
\centering
\footnotesize
\begin{tabular}{|c|c|c|c|}
\hline
\textbf{Версия} & \textbf{Конфигурация} & \textbf{PipelineRun (с)} & \textbf{TaskRun (с)} \\
\hline
\textbf{Последовательная} & 1 поток & \centering 2.6001 & 2.5978 \\
\hline
\multirow{3}{*}{OpenMP} 
  & 2 потока & 1.3471 & 1.3437 \\
  & 3 потока & 0.8977 & 0.8818 \\
  & 4 потоков & 0.7204 & 0.7045 \\
\hline
\multirow{3}{*}{TBB} 
  & 2 потока & 1.1095 & 1.1339 \\
  & 3 потока & 0.7622 & 0.7620 \\
  & 4 потоков & 0.5968 & 0.5875 \\
\hline
\multirow{3}{*}{STL (std::thread)} 
  & 2 потока & 0.8588 & 0.8009 \\
  & 3 потока & 1.0681 & 0.9115 \\
  & 4 потоков & 0.9629 & 0.9080 \\
\hline
\multirow{3}{*}{MPI and OMP} 
  & 2 и 2 & 1.4504 & 1.4602 \\
  & 3 и 3 & 1.3268 & 1.2527 \\
  & 4 и 4 & 1.3045 & 1.3868 \\
\hline
\end{tabular}
\label{tab:parallel_perf}
\end{table}
\subsection{Анализ результатов тестирования}
На основе полученных экспериментальных данных можно сделать следующие выводы:
\begin{enumerate}
\item \textbf{Последовательная реализация} показывает стабильное время выполнения около 2.6 секунд, что служит базовым значением для оценки эффективности параллельных реализаций.
\item \textbf{OpenMP реализация} демонстрирует:
\begin{itemize}
\item Хорошую масштабируемость с увеличением числа потоков
\item Почти линейное ускорение: в 1.93 раза на 2 потоках, в 2.89 раза на 3 потоках и в 3.61 раза на 4 потоках
\item Стабильные результаты между запусками, что говорит о надежности реализации
\end{itemize}
\item \textbf{Intel TBB реализация} показывает:
\begin{itemize}
\item Наилучшую производительность среди всех реализаций
\item Отличную масштабируемость: ускорение в 2.34 раза на 2 потоках, в 3.41 раза на 3 потоках и в 4.36 раза на 4 потоках
\item Стабильные результаты между тестовыми запусками
\end{itemize}
\item \textbf{STL Threads реализация} демонстрирует:
\begin{itemize}
\item Хорошую производительность на всех потоках
\item Неожиданное падение производительности при увеличении числа потоков
\item Нестабильность результатов между запусками, что может указывать на проблемы с балансировкой нагрузки
\end{itemize}
\item \textbf{Гибридная реализация MPI + OpenMP} показывает:
\begin{itemize}
\item Меньшую эффективность по сравнению с другими параллельными реализациями
\item Слабую масштабируемость при увеличении числа процессов и потоков
\item Дополнительные накладные расходы на межпроцессное взаимодействие
\end{itemize}
\end{enumerate}
\textbf{Общие выводы:}
\begin{itemize}
\item Intel TBB демонстрирует наилучшую производительность и масштабируемость
\item OpenMP обеспечивает хороший баланс между простотой реализации и производительностью
\item STL Threads показывает относительно хорошие результаты
\item Гибридный подход MPI + OpenMP оказался менее эффективным для данной задачи
\end{itemize}
Для данной задачи умножения разреженных матриц оптимальным выбором является использование Intel TBB или OpenMP, в зависимости от требований к простоте реализации и максимальной производительности.
\section{Заключение}

В данной работе была реализована и проанализирована модифицированная версия алгоритма умножения разреженных матриц в формате CCS с элементами типа double. Базовая (последовательная) реализация была расширена четырьмя параллельными версиями с использованием современных технологий: OpenMP, Intel TBB, стандартной библиотеки потоков C++ и гибридной модели MPI + OpenMP.
\begin{enumerate}
\item \textbf{Реализация алгоритмов:}
\begin{itemize}
\item Разработан базовый алгоритм умножения разреженных матриц в формате CCS
\item Созданы четыре параллельные версии с использованием различных технологий
\item Обеспечена корректность работы всех реализаций через систему тестов
\end{itemize}
\item \textbf{Экспериментальные результаты:}
\begin{itemize}
\item Intel TBB показала наилучшую производительность с ускорением в 4.36 раза на 4 потоках
\item OpenMP продемонстрировала стабильную масштабируемость с ускорением в 3.61 раза
\item STL Threads выявила особенности при масштабировании на большее число потоков
\item Гибридный подход MPI + OpenMP показал дополнительные накладные расходы на коммуникации
\end{itemize}
\item \textbf{Достигнутые результаты:}
\begin{itemize}
\item Реализованы и протестированы все запланированные версии алгоритма
\item Проведен детальный анализ производительности каждой реализации
\item Выявлены сильные и слабые стороны различных подходов к параллелизации
\item Определены оптимальные сценарии использования каждой технологии
\end{itemize}
\end{enumerate}
Результаты работы демонстрируют эффективность применения параллельных вычислений для задачи умножения разреженных матриц и предоставляют основу для дальнейших исследований в области оптимизации подобных алгоритмов.
\newpage
\section{Список литературы}
\begin{enumerate}
\item Статья: \textit{Разреженные матрицы и форматы их хранения}
\item Мещеряков И.Н. \textit{Набор и вёрстка в системе \LaTeX}
\item Сысоев А.В., Мееров И.Б., Сиднев А.А. \textit{Средства разработки параллельных программ для систем с общей памятью. Библиотека Intel Threading Building Blocks}. — Нижний Новгород, 2007.
\item Статья: \textit{Технологии параллельного программирования. Message Passing Interface (MPI)}
\item Шилдт Г. \textit{Язык программирования C++. Полное руководство}. — М.: Вильямс, 2020. — 1200 с.
\item Reinders J. \textit{Intel Threading Building Blocks: Outfitting C++ for Multi-core Processor Parallelism}. — O'Reilly Media, 2007.
\item OpenMP Architecture Review Board. \textit{OpenMP Application Program Interface Version 5.1}. — 2020. \url{https://www.openmp.org}
\item Gropp W., Lusk E., Skjellum A. \textit{Using MPI: Portable Parallel Programming with the Message Passing Interface}. — MIT Press, 2014.
\item Saad Y. \textit{Iterative Methods for Sparse Linear Systems}. — SIAM, 2003.
\item Habr. \textit{Параллельное программирование с использованием OpenMP}. — \url{https://habr.com/ru/companies/intel/articles/82486/}
\end{enumerate}
\appendix
\newpage
\section*{Приложение}
\addcontentsline{toc}{section}{Приложение}

Полные методы включающие в себя параллелизм.

\subsection*{Seq}
\begin{lstlisting}[language=C++]
namespace sorokin_a_multiplication_sparse_matrices_double_ccs_seq {
void MultiplyCCS(const std::vector<double> &a_values, const std::vector<int> &a_row_indices, int m,
                 const std::vector<int> &a_col_ptr, const std::vector<double> &b_values,
                 const std::vector<int> &b_row_indices, int k, const std::vector<int> &b_col_ptr,
                 std::vector<double> &c_values, std::vector<int> &c_row_indices, int n, std::vector<int> &c_col_ptr) {
  if (static_cast<int>(a_values.size()) > m * k || static_cast<int>(b_values.size()) > k * n) {
    throw std::invalid_argument("Invalid val pointer size");
  }
  c_values.clear();
  c_row_indices.clear();
  c_col_ptr.assign(n + 1, 0);

  std::vector<double> temp_values(m, 0.0);
  std::vector<bool> temp_used(m, false);

  for (int j = 0; j < n; ++j) {
    for (int t = b_col_ptr[j]; t < b_col_ptr[j + 1]; ++t) {
      int row_b = b_row_indices[t];
      double val_b = b_values[t];

      for (int i = a_col_ptr[row_b]; i < a_col_ptr[row_b + 1]; ++i) {
        int row_a = a_row_indices[i];
        temp_values[row_a] += a_values[i] * val_b;
        temp_used[row_a] = true;
      }
    }

    c_col_ptr[j] = static_cast<int>(c_values.size());

    for (int i = 0; i < m; ++i) {
      if (temp_used[i]) {
        c_values.emplace_back(temp_values[i]);
        c_row_indices.emplace_back(i);
        temp_values[i] = 0.0;
        temp_used[i] = false;
      }
    }
  }
  c_col_ptr[n] = static_cast<int>(c_values.size());
}
}  // namespace sorokin_a_multiplication_sparse_matrices_double_ccs_seq
\end{lstlisting}

\subsection*{OpenMP}
\begin{lstlisting}[language=C++]
namespace sorokin_a_multiplication_sparse_matrices_double_ccs_omp {
void MultiplyCCS(const std::vector<double> &a_values, const std::vector<int> &a_row_indices, int m,
                 const std::vector<int> &a_col_ptr, const std::vector<double> &b_values,
                 const std::vector<int> &b_row_indices, int k, const std::vector<int> &b_col_ptr,
                 std::vector<double> &c_values, std::vector<int> &c_row_indices, int n, std::vector<int> &c_col_ptr) {
  if (static_cast<int>(a_values.size()) > m * k || static_cast<int>(b_values.size()) > k * n) {
    throw std::invalid_argument("Invalid val pointer size");
  }

  c_col_ptr.resize(n + 1);
  std::vector<int> nnz_per_column(n, 0);

#pragma omp parallel for
  for (int j = 0; j < n; ++j) {
    std::vector<bool> temp_used(m, false);
    for (int t = b_col_ptr[j]; t < b_col_ptr[j + 1]; ++t) {
      int row_b = b_row_indices[t];
      for (int i = a_col_ptr[row_b]; i < a_col_ptr[row_b + 1]; ++i) {
        int row_a = a_row_indices[i];
        if (!temp_used[row_a]) {
          temp_used[row_a] = true;
          nnz_per_column[j]++;
        }
      }
    }
  }

  c_col_ptr[0] = 0;
  for (int j = 0; j < n; ++j) {
    c_col_ptr[j + 1] = c_col_ptr[j] + nnz_per_column[j];
  }

  int total_nnz = c_col_ptr[n];
  c_values.resize(total_nnz);
  c_row_indices.resize(total_nnz);
#pragma omp parallel for
  for (int j = 0; j < n; ++j) {
    std::vector<double> temp_values(m, 0.0);
    std::vector<bool> temp_used(m, false);

    for (int t = b_col_ptr[j]; t < b_col_ptr[j + 1]; ++t) {
      int row_b = b_row_indices[t];
      double val_b = b_values[t];
      for (int i = a_col_ptr[row_b]; i < a_col_ptr[row_b + 1]; ++i) {
        int row_a = a_row_indices[i];
        temp_values[row_a] += a_values[i] * val_b;
        temp_used[row_a] = true;
      }
    }
    int pos = c_col_ptr[j];
    for (int i = 0; i < m; ++i) {
      if (temp_used[i]) {
        c_row_indices[pos] = i;
        c_values[pos] = temp_values[i];
        pos++;
      }
    }
  }
}
}  // namespace sorokin_a_multiplication_sparse_matrices_double_ccs_omp
\end{lstlisting}

\subsection*{TBB}
\begin{lstlisting}[language=C++]
namespace sorokin_a_multiplication_sparse_matrices_double_ccs_tbb {
void ProcessFirstPass(int j, int m, const std::vector<int>& a_col_ptr, const std::vector<int>& a_row_indices,
                      const std::vector<int>& b_col_ptr, const std::vector<int>& b_row_indices,
                      std::vector<int>& col_sizes) {
  std::vector<bool> temp_used(m, false);
  for (int t = b_col_ptr[j]; t < b_col_ptr[j + 1]; ++t) {
    const int row_b = b_row_indices[t];
    for (int i = a_col_ptr[row_b]; i < a_col_ptr[row_b + 1]; ++i) {
      const int row_a = a_row_indices[i];
      if (!temp_used[row_a]) {
        temp_used[row_a] = true;
        col_sizes[j]++;
      }
    }
  }
}

void ProcessSecondPass(int j, int m, const std::vector<int>& a_col_ptr, const std::vector<int>& a_row_indices,
                       const std::vector<double>& a_values, const std::vector<int>& b_col_ptr,
                       const std::vector<int>& b_row_indices, const std::vector<double>& b_values,
                       const std::vector<int>& c_col_ptr, std::vector<int>& c_row_indices,
                       std::vector<double>& c_values) {
  std::vector<double> temp_values(m, 0.0);
  std::vector<bool> temp_used(m, false);

  for (int t = b_col_ptr[j]; t < b_col_ptr[j + 1]; ++t) {
    const int row_b = b_row_indices[t];
    const double val_b = b_values[t];
    for (int i = a_col_ptr[row_b]; i < a_col_ptr[row_b + 1]; ++i) {
      const int row_a = a_row_indices[i];
      temp_values[row_a] += a_values[i] * val_b;
      temp_used[row_a] = true;
    }
  }

  int pos = c_col_ptr[j];
  for (int i = 0; i < m; ++i) {
    if (temp_used[i]) {
      c_row_indices[pos] = i;
      c_values[pos] = temp_values[i];
      pos++;
    }
  }
}

void MultiplyCCS(const std::vector<double>& a_values, const std::vector<int>& a_row_indices, int m,
                 const std::vector<int>& a_col_ptr, const std::vector<double>& b_values,
                 const std::vector<int>& b_row_indices, int k, const std::vector<int>& b_col_ptr,
                 std::vector<double>& c_values, std::vector<int>& c_row_indices, int n, std::vector<int>& c_col_ptr) {
  c_col_ptr.resize(n + 1);
  c_col_ptr[0] = 0;
  std::vector<int> col_sizes(n, 0);

  tbb::parallel_for(tbb::blocked_range<int>(0, n), [&](const tbb::blocked_range<int>& range) {
    for (int j = range.begin(); j < range.end(); ++j) {
      ProcessFirstPass(j, m, a_col_ptr, a_row_indices, b_col_ptr, b_row_indices, col_sizes);
    }
  });

  for (int j = 0; j < n; ++j) {
    c_col_ptr[j + 1] = c_col_ptr[j] + col_sizes[j];
  }

  const int total_non_zero = c_col_ptr[n];
  c_values.resize(total_non_zero);
  c_row_indices.resize(total_non_zero);

  tbb::parallel_for(tbb::blocked_range<int>(0, n), [&](const tbb::blocked_range<int>& range) {
    for (int j = range.begin(); j < range.end(); ++j) {
      ProcessSecondPass(j, m, a_col_ptr, a_row_indices, a_values, b_col_ptr, b_row_indices, b_values, c_col_ptr,
                        c_row_indices, c_values);
    }
  });
}
}  // namespace sorokin_a_multiplication_sparse_matrices_double_ccs_tbb

\end{lstlisting}

\subsection*{STL}
\begin{lstlisting}[language=C++]
namespace sorokin_a_multiplication_sparse_matrices_double_ccs_stl {

void ProcessFirstPass(int j, int m, const std::vector<int>& a_col_ptr, const std::vector<int>& a_row_indices,
                      const std::vector<int>& b_col_ptr, const std::vector<int>& b_row_indices,
                      std::vector<int>& col_sizes) {
  std::vector<char> temp_used(m, 0);
  for (int t = b_col_ptr[j]; t < b_col_ptr[j + 1]; ++t) {
    const int row_b = b_row_indices[t];
    for (int i = a_col_ptr[row_b]; i < a_col_ptr[row_b + 1]; ++i) {
      const int row_a = a_row_indices[i];
      if (temp_used[row_a] == 0) {
        temp_used[row_a] = 1;
        col_sizes[j]++;
      }
    }
  }
}

void ProcessSecondPass(int j, int m, const std::vector<int>& a_col_ptr, const std::vector<int>& a_row_indices,
                       const std::vector<double>& a_values, const std::vector<int>& b_col_ptr,
                       const std::vector<int>& b_row_indices, const std::vector<double>& b_values,
                       const std::vector<int>& c_col_ptr, std::vector<int>& c_row_indices,
                       std::vector<double>& c_values) {
  std::vector<double> temp_values(m, 0.0);
  std::vector<char> temp_used(m, 0);

  for (int t = b_col_ptr[j]; t < b_col_ptr[j + 1]; ++t) {
    const int row_b = b_row_indices[t];
    const double val_b = b_values[t];
    for (int i = a_col_ptr[row_b]; i < a_col_ptr[row_b + 1]; ++i) {
      const int row_a = a_row_indices[i];
      temp_values[row_a] += a_values[i] * val_b;
      temp_used[row_a] = 1;
    }
  }

  int pos = c_col_ptr[j];
  for (int i = 0; i < m; ++i) {
    if (temp_used[i] != 0) {
      c_row_indices[pos] = i;
      c_values[pos] = temp_values[i];
      pos++;
    }
  }
}

void MultiplyCCS(const std::vector<double>& a_values, const std::vector<int>& a_row_indices, int m,
                 const std::vector<int>& a_col_ptr, const std::vector<double>& b_values,
                 const std::vector<int>& b_row_indices, int k, const std::vector<int>& b_col_ptr,
                 std::vector<double>& c_values, std::vector<int>& c_row_indices, int n, std::vector<int>& c_col_ptr) {
  c_col_ptr.resize(n + 1);
  c_col_ptr[0] = 0;
  std::vector<int> col_sizes(n, 0);

  int num_threads = ppc::util::GetPPCNumThreads();
  num_threads = std::min(num_threads, n);

  std::vector<std::thread> threads;
  std::atomic<int> next_j(0);

  auto first_pass_worker = [&]() {
    int j = 0;
    while ((j = next_j.fetch_add(1, std::memory_order_relaxed)) < n) {
      ProcessFirstPass(j, m, a_col_ptr, a_row_indices, b_col_ptr, b_row_indices, col_sizes);
    }
  };

  threads.reserve(num_threads);
  for (int t = 0; t < num_threads; ++t) {
    threads.emplace_back(first_pass_worker);
  }

  for (auto& thread : threads) {
    thread.join();
  }

  for (int j = 0; j < n; ++j) {
    c_col_ptr[j + 1] = c_col_ptr[j] + col_sizes[j];
  }

  const int total_non_zero = c_col_ptr[n];
  c_values.resize(total_non_zero);
  c_row_indices.resize(total_non_zero);

  threads.clear();
  next_j.store(0);

  auto second_pass_worker = [&]() {
    int j = 0;
    while ((j = next_j.fetch_add(1, std::memory_order_relaxed)) < n) {
      ProcessSecondPass(j, m, a_col_ptr, a_row_indices, a_values, b_col_ptr, b_row_indices, b_values, c_col_ptr,
                        c_row_indices, c_values);
    }
  };

  for (int t = 0; t < num_threads; ++t) {
    threads.emplace_back(second_pass_worker);
  }

  for (auto& thread : threads) {
    thread.join();
  }
}

}  // namespace sorokin_a_multiplication_sparse_matrices_double_ccs_stl
\end{lstlisting}

\subsection*{MPI + OMP}
\begin{lstlisting}[language=C++]
namespace {

void ComputeLocalNNZ(int start_col, int local_cols, const std::vector<int>& b_col_ptr,
                     const std::vector<int>& b_row_indices, const std::vector<int>& a_col_ptr,
                     const std::vector<int>& a_row_indices, int m, std::vector<int>& local_nnz) {
  for (int j_local = 0; j_local < local_cols; ++j_local) {
    int j = start_col + j_local;
    std::vector<bool> temp_used(m, false);
    for (int t = b_col_ptr[j]; t < b_col_ptr[j + 1]; ++t) {
      int row_b = b_row_indices[t];
      for (int i = a_col_ptr[row_b]; i < a_col_ptr[row_b + 1]; ++i) {
        int row_a = a_row_indices[i];
        if (!temp_used[row_a]) {
          temp_used[row_a] = true;
          local_nnz[j_local]++;
        }
      }
    }
  }
}

void GatherNNZData(boost::mpi::communicator& world, const std::vector<int>& local_nnz, int local_cols, int start_col,
                   std::vector<int>& nnz_per_column, std::vector<int>& recv_counts, std::vector<int>& displs) {
  int rank = world.rank();
  int size = world.size();

  if (rank == 0) {
    std::vector<int> all_start_cols(size);
    std::vector<int> all_local_cols(size);
    boost::mpi::gather(world, start_col, all_start_cols, 0);
    boost::mpi::gather(world, local_cols, all_local_cols, 0);

    for (int p = 0; p < size; ++p) {
      recv_counts[p] = all_local_cols[p];
      displs[p] = all_start_cols[p];
    }
  } else {
    boost::mpi::gather(world, start_col, 0);
    boost::mpi::gather(world, local_cols, 0);
  }

  if (rank == 0) {
    boost::mpi::gatherv(world.split(0), local_nnz.data(), local_cols, nnz_per_column.data(), recv_counts, displs, 0);
  } else {
    boost::mpi::gatherv(world.split(0), local_nnz.data(), local_cols, 0);
  }
}

void BuildCColPtr(boost::mpi::communicator& world, const std::vector<int>& nnz_per_column,
                  std::vector<int>& c_col_ptr) {
  size_t n = nnz_per_column.size();
  if (world.rank() == 0) {
    c_col_ptr.resize(n + 1);
    c_col_ptr[0] = 0;
    std::partial_sum(nnz_per_column.begin(), nnz_per_column.end(), c_col_ptr.begin() + 1);
  }
  boost::mpi::broadcast(world, c_col_ptr, 0);
}

void ComputeCValuesAndIndices(const std::vector<double>& a_values, const std::vector<int>& a_row_indices,
                              const std::vector<int>& a_col_ptr, const std::vector<double>& b_values,
                              const std::vector<int>& b_row_indices, const std::vector<int>& b_col_ptr, int m, int n,
                              std::vector<double>& c_values, std::vector<int>& c_row_indices,
                              const std::vector<int>& c_col_ptr) {
  int total_nnz = c_col_ptr[n];
  c_values.resize(total_nnz);
  c_row_indices.resize(total_nnz);

#pragma omp parallel for
  for (int j = 0; j < n; ++j) {
    std::vector<double> temp_values(m, 0.0);
    std::vector<bool> temp_used(m, false);

    for (int t = b_col_ptr[j]; t < b_col_ptr[j + 1]; ++t) {
      int row_b = b_row_indices[t];
      double val_b = b_values[t];
      for (int i = a_col_ptr[row_b]; i < a_col_ptr[row_b + 1]; ++i) {
        int row_a = a_row_indices[i];
        temp_values[row_a] += a_values[i] * val_b;
        temp_used[row_a] = true;
      }
    }

    int pos = c_col_ptr[j];
    int count = 0;
    for (int i = 0; i < m; ++i) {
      if (temp_used[i]) {
        c_row_indices[pos + count] = i;
        c_values[pos + count] = temp_values[i];
        count++;
      }
    }
  }
}

}  // namespace

namespace sorokin_a_multiplication_sparse_matrices_double_ccs_all {

void MultiplyCCS(boost::mpi::communicator& world, const std::vector<double>& a_values,
                 const std::vector<int>& a_row_indices, int m, const std::vector<int>& a_col_ptr,
                 const std::vector<double>& b_values, const std::vector<int>& b_row_indices, int k,
                 const std::vector<int>& b_col_ptr, std::vector<double>& c_values, std::vector<int>& c_row_indices,
                 int n, std::vector<int>& c_col_ptr) {
  int rank = world.rank();
  int size = world.size();

  int cols_per_process = n / size;
  int remainder = n % size;
  int start_col = (rank * cols_per_process) + std::min(rank, remainder);
  int end_col = start_col + cols_per_process + (rank < remainder ? 1 : 0);
  int local_cols = end_col - start_col;

  std::vector<int> local_nnz(local_cols, 0);
  ComputeLocalNNZ(start_col, local_cols, b_col_ptr, b_row_indices, a_col_ptr, a_row_indices, m, local_nnz);

  std::vector<int> nnz_per_column(n, 0);
  std::vector<int> recv_counts(size);
  std::vector<int> displs(size);
  GatherNNZData(world, local_nnz, local_cols, start_col, nnz_per_column, recv_counts, displs);

  BuildCColPtr(world, nnz_per_column, c_col_ptr);

  if (rank == 0) {
    ComputeCValuesAndIndices(a_values, a_row_indices, a_col_ptr, b_values, b_row_indices, b_col_ptr, m, n, c_values,
                             c_row_indices, c_col_ptr);
  } else {
    c_col_ptr.clear();
    c_values.clear();
    c_row_indices.clear();
  }

  world.barrier();
}

}  // namespace sorokin_a_multiplication_sparse_matrices_double_ccs_all
\end{lstlisting}
\end{document}