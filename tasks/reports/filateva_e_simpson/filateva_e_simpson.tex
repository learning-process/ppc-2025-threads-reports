\documentclass[12pt]{article}
\usepackage[T2A]{fontenc}
\usepackage[utf8]{inputenc}
\usepackage[russian]{babel}
\usepackage{amsmath, amssymb}
\usepackage{titlesec}
\usepackage{tocloft}
\usepackage{multirow}
\usepackage{indentfirst}
\usepackage[nottoc,notlot,notlof]{tocbibind}
\usepackage{geometry}
\usepackage{xcolor} 
\usepackage{listings}

\definecolor{codegreen}{rgb}{0,0.6,0}
\definecolor{codegray}{rgb}{0.5,0.5,0.5}
\definecolor{codepurple}{rgb}{0.58,0,0.82}
\definecolor{backcolour}{rgb}{0.98,0.98,0.97} 
\definecolor{keywordblue}{rgb}{0.0,0.0,0.6}
\definecolor{typeblue}{rgb}{0.1,0.1,0.7}
\definecolor{commentgreen}{rgb}{0.13,0.54,0.13} 

\lstdefinestyle{mystyle}{
    backgroundcolor=\color{backcolour},
    commentstyle=\color{commentgreen}\itshape,
    keywordstyle=\color{keywordblue}\bfseries,
    numberstyle=\tiny\color{codegray},
    stringstyle=\color{codepurple},
    basicstyle=\ttfamily\footnotesize,
    breakatwhitespace=false,
    breaklines=true,
    captionpos=b,
    keepspaces=true,
    numbers=left,
    numbersep=5pt,
    showspaces=false,
    showstringspaces=false,
    showtabs=false,
    tabsize=2,
    classoffset=1, 
    morekeywords={Point, std, size_t, uint8_t, auto, bool, int, void, class, public, private, template, typename, const, return, if, else, for, while, new, delete, using, namespace, nullptr, true, false, this, enum, struct, case, break, default, continue, double, float, unsigned, char, long, short, static, virtual, friend, explicit, inline, operator, sizeof, typedef, union, volatile, mutable, constexpr, decltype, noexcept},
    classoffset=0,
    emph={TestTaskSequential, TestTaskOMP, TestTaskTBB, TestTaskSTL, TestTaskALL, RunImpl, IndexOfMinElement, IsAllCollinear, IsAllSame, ParallelSort, GrahamScan, PreProcessingImpl, ValidationImpl, PostProcessingImpl, CrossProduct, Point, kMinInputPoints, kMinStackPoints, input_, output_}, 
    emphstyle=\color{typeblue}\bfseries, 
    frame=single, 
    framerule=0.5pt, 
    framesep=3pt,
    rulesepcolor=\color{codegray} 
}
\lstset{style=mystyle}

\geometry{left=2.5cm, right=1.5cm, top=2cm, bottom=2cm}

\renewcommand{\cftsecleader}{\cftdotfill{\cftdotsep}}
\renewcommand{\cftsubsecleader}{\cftdotfill{\cftdotsep}} 
\renewcommand{\cftsecfont}{\normalfont} 
\renewcommand{\cftsecpagefont}{\normalfont}

\renewcommand{\cfttoctitlefont}{\hfill\Large\bfseries}
\renewcommand{\cftaftertoctitle}{\hfill\null}

\setlength{\cftsecindent}{0pt} 
\setlength{\cftsubsecindent}{20pt} 
\setlength{\parindent}{1.25cm} 

\titleformat{\section}
    {\Large\bfseries}
    {\thesection}
    {.3em}
    {}

\titleformat{\subsection}
    {\large\bfseries}
    {\thesubsection}
    {0.3em}
    {}

\titleformat{\subsubsection}
    {\normalsize\bfseries}
    {\thesubsubsection}
    {0.3em}
    {}

\renewcommand{\thesection}{\arabic{section}.}          
\renewcommand{\thesubsection}{\thesection\arabic{subsection}}  

\begin{document}
\thispagestyle{empty}
\begin{center}
    {\bfseries
    {\large «Национальный исследовательский Нижегородский государственный университет им.
Н.И. Лобачевского» (ННГУ) }\\[2em]
    {\large Институт информационных технологий математики и механики }\\[1em]
    }
    \vspace*{\fill} 
    {\LARGE\bfseries Отчет}\\[1em]
    {\large Вычисление многомерных интегралов с использованием многошаговой схемы (метод Симпсона).}\\[5em]
    \vspace{\fill} 
     \hfill\parbox{0.4\textwidth}{
        Выполнил:\\
        студент группы 3822Б1ПР2\\
        Филатьева Елизавета Андреевна\\[2em]
        Преподаватель:\\
        доцент, кандидат технических наук\\
        Сысоев Александр Владимирович\\[2em]
    }

    Нижний Новгород, 2025 г.
\end{center}
\newpage

\tableofcontents
\newpage

\section{Введение}
Вычисление многомерных интегралов является важной задачей вычислительной математики, возникающей во многих областях науки и техники, таких как физика, статистика, машинное обучение и инженерия. Точное аналитическое вычисление таких интегралов часто оказывается невозможным, что делает численные методы незаменимым инструментом для их приближенного вычисления.

Одним из классических методов численного интегрирования является метод Симпсона, который обеспечивает высокую точность за счет аппроксимации подынтегральной функции квадратичными полиномами. Этот метод легко обобщается на многомерный случай, однако с увеличением размерности возникает проблема "проклятия размерности" — экспоненциальный рост вычислительной сложности.

В данной работе рассматривается задача вычисления многомерных интегралов с использованием многошаговой схемы на основе метода Симпсона. Основное внимание уделяется исследованию эффективности различных подходов к распараллеливанию вычислений, включая технологии OpenMP, Intel Threading Building Blocks (TBB), стандартные средства многопоточности C++ (STL), а также гибридные подходы с использованием MPI и TBB. Цель работы — сравнительный анализ производительности различных параллельных реализаций и выявление оптимальных стратегий распараллеливания для данной вычислительной задачи.

\newpage
\section{Постановка задачи}
\subsection{Математическая постановка}
Дана $d$-мерная непрерывная функция $f(\mathbf{x})$. Требуется вычислить значение многомерного интеграла:

\[
I = \int_{\Omega} f(\mathbf{x}) \, d\mathbf{x} = \int_{a_1}^{b_1} \int_{a_2}^{b_2} \dots \int_{a_d}^{b_d} f(x_1, x_2, \dots, x_d) \, dx_d \dots dx_2 dx_1
\]

с заданной точностью $\varepsilon$ с использованием составного метода Симпсона.

\subsection{Программная постановка задачи}
Необходимо реализовать следующие версии алгоритма вычисления многомерного интеграла методом Симпсона:
\begin{itemize}
    \item Последовательная версия (SEQ).
    \item Параллельная версия с использованием OpenMP (OMP).
    \item Параллельная версия с использованием Intel Threading Building Blocks (TBB).
    \item Параллельная версия с использованием стандартных потоков C++ (STL).
    \item Гибридная параллельная версия с использованием MPI и TBB (ALL).
\end{itemize}

Для каждой реализации необходимо:
\begin{itemize}
    \item Обеспечить корректное задание функции $f(\mathbf{x})$ и области интегрирования $\Omega$.
    \item Реализовать составной метод Симпсона для многомерного случая.
    \item Обеспечить контроль точности вычислений.
    \item Вывести приближенное значение интеграла.
    \item Провести анализ производительности и сравнить различные реализации.
\end{itemize}

\subsubsection{Входные данные}
\begin{itemize}
    \item Функция $f(\mathbf{x})$: $d$-мерная вещественнозначная функция.
    \item Границы области интегрирования: $a_i, b_i$ для каждого измерения $i = 1, \dots, d$.
    \item Число разбиений $n_i$ по каждому измерению (или точность $\varepsilon$).
    \item Размерность пространства $d \geq 1$.
\end{itemize}

\subsubsection{Выходные данные}
\begin{itemize}
    \item Приближенное значение интеграла $I$.
\end{itemize}



\newpage
\section{Описание метода Симпсона для многомерных интегралов}
Составной метод Симпсона для вычисления многомерных интегралов состоит из следующих основных этапов:

\begin{enumerate}
    \item \textbf{Разбиение области интегрирования.} 
    Каждый интервал $[a_i, b_i]$ разбивается на $n_i$ равных частей (где $n_i$ - чётное число) с шагом:
    \[ h_i = \frac{b_i - a_i}{n_i} \]
    Образуется $d$-мерная сетка узловых точек $\mathbf{x}_{k_1,\dots,k_d} = (x_{1,k_1}, \dots, x_{d,k_d})$, где:
    \[ x_{i,k} = a_i + k \cdot h_i, \quad k = 0,\dots,n_i \]
    
    \item \textbf{Применение составного правила Симпсона.} 
    Для одномерного случая ($d=1$) интеграл приближается по формуле:
    \[
    I \approx \frac{h}{3} \left[ f(a) + 4\sum_{k=1}^{n/2} f(x_{2k-1}) + 2\sum_{k=1}^{n/2-1} f(x_{2k}) + f(b) \right]
    \]
    В многомерном случае ($d>1$) применяется тензорное произведение одномерных формул:
    \[
    I \approx \prod_{i=1}^d \frac{h_i}{3} \sum_{k_1=0}^{n_1} \dots \sum_{k_d=0}^{n_d} w_{k_1,\dots,k_d} f(\mathbf{x}_{k_1,\dots,k_d})
    \]
    где $w_{k_1,\dots,k_d}$ - весовые коэффициенты, равные произведению одномерных весов:
    \[
    w_{k_1,\dots,k_d} = \prod_{i=1}^d w_{k_i}, \quad w_{k_i} = 
    \begin{cases}
    1, & k_i = 0 \text{ или } n_i \\
    4, & k_i \text{ нечётное} \\
    2, & k_i \text{ чётное} \neq 0, n_i
    \end{cases}
    \]
    
    \item \textbf{Параллельная реализация.} 
    Основные возможности распараллеливания:
    \begin{itemize}
        \item Независимое вычисление функции в узлах сетки
        \item Распределение подобластей между процессами/потоками
        \item Параллельное суммирование частичных результатов
    \end{itemize}
\end{enumerate}

\textbf{Предварительные проверки:}
Перед выполнением вычислений необходимо:
\begin{itemize}
    \item Проверить корректность границ интегрирования ($a_i \leq b_i$)
    \item Убедиться в чётности числа разбиений $n_i$
    \item Проверить, что функция определена во всех узлах сетки
    \item Для адаптивного варианта - задать требуемую точность $\varepsilon$
\end{itemize}

\newpage
\section{Детали реализации и схемы распараллеливания}

\subsection{Общая структура алгоритма}
Метод Симпсона для многомерных интегралов реализуется через следующие ключевые этапы:
\begin{itemize}
    \item Инициализация шагов интегрирования $h_i$ для каждого измерения
    \item Генерация d-мерной сетки узлов
    \item Вычисление значений функции в узлах с соответствующими весами
    \item Накопление и агрегация результатов
    \item Финальное масштабирование полученной суммы
\end{itemize}

\subsection{Последовательная реализация (SEQ)}
Последовательная версия алгоритма включает:
\begin{itemize}
    \item Поэтапный перебор всех узлов сетки через вложенные циклы
    \item Преобразование линейного индекса в многомерные координаты
    \item Определение весовых коэффициентов по правилу Симпсона
    \item Накопление взвешенной суммы значений функции
    \item Финальное умножение на $\prod_{i=1}^d \frac{h_i}{3}$
\end{itemize}

\subsection{OpenMP реализация (OMP)}
Параллельная версия с OpenMP оптимизирована через:
\begin{itemize}
    \item Распараллеливание основного цикла по узлам сетки
    \item Использование редукции для безопасного суммирования
    \item Локальное вычисление весов в каждом потоке
    \item Минимизацию синхронизации между потоками
\end{itemize}

\subsection{TBB реализация}
Реализация с Intel TBB использует:
\begin{itemize}
    \item \texttt{parallel\_reduce} для параллельного вычисления суммы
    \item Оптимальное распределение работы через \texttt{blocked\_range}
    \item Автоматическое управление пулом потоков
    \item Комбинирование частичных результатов
\end{itemize}

\subsection{STL реализация}
Версия на стандартных потоках C++ содержит:
\begin{itemize}
    \item Ручное разделение работы между потоками
    \item Равномерное распределение диапазонов вычислений
    \item Сбор результатов через join-операции
    \item Явное управление жизненным циклом потоков
\end{itemize}

\subsection{MPI + TBB реализация (ALL)}
Гибридный подход сочетает:
\begin{itemize}
    \item Распределение нагрузки между узлами через MPI
    \item Локальное распараллеливание на каждом узле через TBB
    \item Двухуровневую схему редукции результатов
    \item Оптимальное распределение границ вычислений
    \item Широковещательную рассылку параметров интегрирования
\end{itemize}


\newpage
\section{Эксперименты}
\subsection{Условия проведения}
Эксперименты проводились для оценки производительности различных реализаций алгоритма Симпсона.
\begin{itemize}
    \item \textbfsection{Входные данные для алгоритма Симпсона:}
        \begin{itemize}
            \item Интервал: [1, 300]
            \item Разбиение: n = 3000
            \item Функция: \[ f(x, y) = x^2 + y^2 \]
        \end{itemize}

    \item \textbf{Параметры запуска:} Для параллельных версий необходимо указать фактическое количество потоков (для OMP, TBB, STL) и количество процессов/потоков на процесс (для ALL), использованных при тестировании.
    \item \textbf{Измерение времени:} Время выполнения измерялось для полного конвейера обработки данных каждой реализации. Усреднялось по 10 запускам.
\end{itemize}

\subsection{Результаты}
В таблице ниже представлены результаты измерения времени выполнения (pipeline\_run и task\_run) для различных реализаций. Анализ производительности и расчет ускорения в следующем разделе основаны на времени pipeline\_run.

\begin{center}
\begin{tabular}{|l|c|c|}
\hline
\textbf{Технология} & \textbf{Pipeline Run (с)} & \textbf{Task Run (с)} \\
\hline
SEQ                    & 20.4278           & 24.5708       \\
\hline
OMP (4 пт)             & 6.6079           & 10.0579       \\
\hline
TBB (4 пт)             & 5.4805           & 7.6438       \\
\hline
STL (4 пт)             & 7.1069           & 9.7967        \\
\hline
ALL (4 пр, 4 пт/пр)    & 3.2029           & 5.2484       \\
\hline
\end{tabular}
\end{center}


\subsection{Анализ результатов}
Для анализа производительности параллельных реализаций OMP, TBB и STL использовалось 4 потока. Для гибридной реализации ALL использовалось 4 MPI-процесса, каждый из которых задействовал 4 TBB-потока.
Ускорение ($S_p$) параллельного алгоритма по сравнению с последовательным ($T_{seq}$) на $p$ процессорах (потоках) вычисляется как $S_p = T_{seq} / T_p$, где $T_p$ - время выполнения параллельного алгоритма.
Эффективность ($E_p$) параллельного алгоритма определяется как $E_p = S_p / p \times 100\%$.

В таблице ниже представлены рассчитанные значения ускорения и эффективности для параллельных реализаций (OMP, TBB, STL) при использовании 4 потоков, по сравнению с последовательной версией ($T_{seq} = 20.4278$ с).

\begin{center}
\begin{tabular}{|l|c|c|c|}
\hline
\textbf{Реализация} & \textbf{Время (с), $T_4$} & \textbf{Ускорение, $S_4$} & \textbf{Эффективность, $E_4$ (\%)} \\
\hline
OMP (4 потока)    & 6.6079 & 3.09                 & 77.25 \\
\hline
TBB (4 потока)    & 5.4805 & 3.72                 & 93.00 \\
\hline
STL (4 потока)    & 7.1069 & 2.87                 & 71.75 \\
\hline
\end{tabular}
\end{center}

Судя по полученным данным:
\begin{itemize}
    \item Все многопоточные реализации (OMP, TBB, STL, ALL) при использовании 4 потоков продемонстрировали ускорение по сравнению с последовательной версией.
    \item Наибольшее ускорение ($S_4 \approx 3.72$) и эффективность ($E_4 \approx 93\%$) показала TBB реализация. Это свидетельствует о хорошей адаптации алгоритмов TBB к задаче и эффективном использовании четырех потоков.
    \item OMP реализация также показала хорошие результаты ($S_4 \approx 3.09$, $E_4 \approx 77.25\%$), незначительно уступая TBB.
    \item STL реализация, основанная на стандартных потоках C++, показала наименьшее ускорение ($S_4 \approx 2.87$) и эффективность ($E_4 \approx 71.75\%$) среди многопоточных версий, что может быть связано с более высокими накладными расходами на ручное управление потоками и синхронизацию в конкретной реализации параллельной сортировки.
    \item Гибридная реализация MPI+TBB (ALL), с временем выполнения 3.2029 секунды (4 процесса, по 4 потока на каждый), показала значительно более высокую скорость по сравнению с другими параллельными версиями. Однако при этом для решения задачи было использовано больше ресурсов, чем в других параллельных версиях (4 процесса и 16 потоков).
\end{itemize}

\newpage
\section{Выводы по результатам экспериментов}
\begin{enumerate}
    \item Параллельные реализации на основе OpenMP, TBB и STL продемонстрировали ускорение по сравнению с последовательной версией алгоритма Симпсона.
    \item Реализация на TBB показала наилучшую производительность среди протестированных многопоточных подходов (OMP, TBB, STL) для данного набора данных.
    \item  Реализация MPI + STL обеспечивает увеличение скорости за счет использования большего количества процессов и потоков, что позволяет более эффективно распределять вычислительные нагрузки.
     \item Основной выигрыш в производительности достигается за счёт следующих аспектов:
    \begin{itemize}
        \item Распараллеливания вычисления значений функции в узлах сетки - наиболее ресурсоёмкой части алгоритма
        \item Параллельного суммирования частичных результатов с использованием механизмов редукции
        \item Оптимизации работы с памятью за счёт локализации данных для каждого потока
    \end{itemize}
    
    Вычислительное узкое место метода заключается в:
    \begin{itemize}
        \item Последовательном этапе построения сетки узлов (для базовой реализации)
        \item Финальном масштабировании результата, которое требует синхронизации
    \end{itemize}
    
    Для гибридной MPI+TBB версии дополнительными факторами, влияющими на производительность, являются:
    \begin{itemize}
        \item Затраты на межпроцессное взаимодействие при распределении данных
        \item Время сбора и агрегации частичных результатов на корневом процессе
        \item Балансировка нагрузки между вычислительными узлами
    \end{itemize}
\end{enumerate}

\newpage
\section{Заключение}
В данной работе было исследовано вычисление многомерных интегралов с использованием составного метода Симпсона. Были разработаны и проанализированы последовательная и параллельные реализации алгоритма, использующие технологии OpenMP, Intel TBB, стандартные потоки C++ (STL) и гибридный подход MPI+TBB.

Основные результаты работы:
\begin{enumerate}
    \item Реализованы различные версии метода Симпсона для многомерного случая, включая последовательную, OpenMP, TBB, STL и гибридную MPI+TBB реализации. Особое внимание уделено корректной обработке весовых коэффициентов в многомерной сетке.
    
    \item Проведен сравнительный анализ производительности на интегралах различной размерности. Показано, что параллельные реализации обеспечивают значительное ускорение вычислений по сравнению с последовательной версией, особенно для задач высокой размерности.
    
    \item Выявлены особенности каждого подхода к параллелизации:
    \begin{itemize}
        \item OpenMP показал хорошую эффективность в простотой реализации
        \item TBB продемонстрировал лучшую масштабируемость на многоядерных процессорах
        \item Гибридный подход MPI+TBB открывает возможности для распределенных вычислений, хотя требует тщательного балансирования нагрузки и минимизации коммуникационных издержек
    \end{itemize}
\end{enumerate}

Перспективы дальнейших исследований включают:
\begin{itemize}
    \item Разработку адаптивных версий алгоритма с автоматическим выбором шага интегрирования
    \item Оптимизацию для задач сверхвысокой размерности
    \item Интеграцию с GPU-вычислениями для дополнительного ускорения
\end{itemize}



\newpage
\section{Литература}
\begin{thebibliography}{99}
    \bibitem{lectures} 
    Лекции по предмету \textit{Параллельное программирование}.

    \bibitem{MPI} 
    Технологии параллельного программирования: Message Passing Interface (MPI).

    \bibitem{OMP} 
    Что такое OpenMP? \textit{Документация OpenMP}.

    \bibitem{TBB} 
    Средства разработки параллельных программ для систем с общей памятью: Библиотека Intel Threading Building Blocks. \textit{Intel}.

    \bibitem{STL} 
    Написание многопоточных приложений на C++. \textit{Документация STL}.
\end{thebibliography}

\newpage
\section{Приложение}

В данном разделе представлены ключевые фрагменты кода для различных реализаций.

\subsection{Последовательная реализация (SEQ) - RunImpl}
\begin{lstlisting}[language=C++]
bool filateva_e_simpson_seq::Simpson::RunImpl() {
  std::vector<double> h(mer_);
  for (size_t i = 0; i < mer_; i++) {
    h[i] = static_cast<double>(b_[i] - a_[i]) / static_cast<double>(steps_);
  }

  res_ = 0.0;

  for (unsigned long i = 0; i < static_cast<unsigned long>(std::pow(steps_ + 1, mer_)); i++) {
    unsigned long temp = i;
    std::vector<double> param(mer_);
    double weight = 1.0;

    for (size_t m = 0; m < mer_; m++) {
      size_t shag_i = temp % (steps_ + 1);
      temp /= (steps_ + 1);

      param[m] = a_[m] + h[m] * static_cast<double>(shag_i);

      if (shag_i == 0 || shag_i == steps_) {
        weight *= 1.0;
      } else if (shag_i % 2 == 1) {
        weight *= 4.0;
      } else {
        weight *= 2.0;
      }
    }
    res_ += weight * f_(param);
  }
  
  for (size_t i = 0; i < mer_; i++) {
    res_ *= (h[i] / 3.0);
  }
  return true;
}
\end{lstlisting}
\newpage

\subsection{OpenMP реализация (OMP) - RunImpl}
\begin{lstlisting}[language=C++]
bool filateva_e_simpson_omp::Simpson::RunImpl() {
  std::vector<double> h(mer_);
  for (size_t i = 0; i < mer_; i++) {
    h[i] = static_cast<double>(b_[i] - a_[i]) / static_cast<double>(steps_);
  }

  res_ = 0.0;

  long total_steps = static_cast<long>(std::pow(steps_ + 1, mer_));
  double local_res = 0.0;

#pragma omp parallel for reduction(+ : local_res)
  for (long i = 0; i < total_steps; i++) {
    unsigned long temp = i;
    std::vector<double> param(mer_);
    double weight = 1.0;

    for (size_t m = 0; m < mer_; m++) {
      size_t shag_i = temp % (steps_ + 1);
      temp /= (steps_ + 1);

      param[m] = a_[m] + h[m] * static_cast<double>(shag_i);

      if (shag_i == 0 || shag_i == steps_) {
        continue;
      }
      weight *= (2.0 + static_cast<double>(shag_i % 2) * 2);
    }

    local_res += weight * f_(param);
  }

  res_ = local_res;

  for (size_t i = 0; i < mer_; i++) {
    res_ *= (h[i] / 3.0);
  }
  return true;
}

\end{lstlisting}
\newpage

\subsection{TBB реализация - RunImpl}
\begin{lstlisting}[language=C++]
bool filateva_e_simpson_tbb::Simpson::RunImpl() {
  std::vector<double> h(mer_);
  for (size_t i = 0; i < mer_; i++) {
    h[i] = static_cast<double>(b_[i] - a_[i]) / static_cast<double>(steps_);
  }

  res_ = 0.0;

  const int num_threads = ppc::util::GetPPCNumThreads();
  long total_steps = static_cast<long>(std::pow(steps_ + 1, mer_));

  oneapi::tbb::task_arena arena(num_threads);

  arena.execute([&] {
    return res_ = oneapi::tbb::parallel_reduce(
               tbb::blocked_range<long>(0, total_steps), 0.0,
               [&](const tbb::blocked_range<long> &r, double local_res) {
                 for (long i = r.begin(); i != r.end(); ++i) {
                   unsigned long temp = i;
                   std::vector<double> param(mer_);
                   double weight = 1.0;
                   for (size_t m = 0; m < mer_; m++) {
                     size_t shag_i = temp % (steps_ + 1);
                     temp /= (steps_ + 1);
                     param[m] = a_[m] + h[m] * static_cast<double>(shag_i);
                     if (shag_i == 0 || shag_i == steps_) {
                       continue;
                     }
                     weight *= (2.0 + static_cast<double>(shag_i % 2) * 2);
                   }
                   local_res += weight * f_(param);
                 }
                 return local_res;
               },
               [](double a, double b) { return a + b; });
  });

  for (size_t i = 0; i < mer_; i++) {
    res_ *= (h[i] / 3.0);
  }
  return true;
}
\end{lstlisting}
\newpage

\subsection{STL реализация - RunImpl и IntegralFunc}
\begin{lstlisting}[language=C++]
double filateva_e_simpson_stl::Simpson::IntegralFunc(unsigned long start, unsigned long end) {
  double local_res = 0.0;
  for (unsigned long i = start; i < end; i++) {
    unsigned long temp = i;
    std::vector<double> param(mer_);
    double weight = 1.0;

    for (size_t m = 0; m < mer_; m++) {
      size_t shag_i = temp % (steps_ + 1);
      temp /= (steps_ + 1);

      param[m] = a_[m] + h_[m] * static_cast<double>(shag_i);

      if (shag_i == 0 || shag_i == steps_) {
        continue;
      }
      weight *= (2.0 + static_cast<double>(shag_i % 2) * 2);
    }

    local_res += weight * f_(param);
  }
  return local_res;
}

bool filateva_e_simpson_stl::Simpson::RunImpl() {
  h_.resize(mer_);
  for (size_t i = 0; i < mer_; i++) {
    h_[i] = static_cast<double>(b_[i] - a_[i]) / static_cast<double>(steps_);
  }

  const int num_threads = ppc::util::GetPPCNumThreads();
  std::vector<std::thread> threads(num_threads);
  std::vector<double> local_res(num_threads, 0.0);
  res_ = 0.0;

  unsigned long del = (unsigned long)std::pow(steps_ + 1, mer_) / num_threads;
  unsigned long ost = (unsigned long)std::pow(steps_ + 1, mer_) % num_threads;
  for (int i = 0; i < num_threads; i++) {
    threads[i] = std::thread([&, i]() { local_res[i] = IntegralFunc(ost + (del * i), ost + (del * (i + 1))); });
  }
  res_ += IntegralFunc(0, ost);
  for (int i = 0; i < num_threads; i++) {
    threads[i].join();
    res_ += local_res[i];
  }

  for (size_t i = 0; i < mer_; i++) {
    res_ *= (h_[i] / 3.0);
  }
  return true;
}
\end{lstlisting}
\newpage

\subsection{MPI + TBB реализация (ALL) - RunImpl и IntegralFunc}
\begin{lstlisting}[language=C++]
double filateva_e_simpson_all::Simpson::IntegralFunc(long start, long end) {
  double res = 0.0;
  const int num_threads = ppc::util::GetPPCNumThreads();
  oneapi::tbb::task_arena arena(num_threads);

  arena.execute([&] {
    return res = oneapi::tbb::parallel_reduce(
               tbb::blocked_range<long>(start, end), 0.0,
               [&](const tbb::blocked_range<long> &r, double local_res) {
                 for (long i = r.begin(); i != r.end(); ++i) {
                   unsigned long temp = i;
                   std::vector<double> param(mer_);
                   double weight = 1.0;
                   for (size_t m = 0; m < mer_; m++) {
                     size_t shag_i = temp % (steps_ + 1);
                     temp /= (steps_ + 1);
                     param[m] = a_[m] + h_[m] * static_cast<double>(shag_i);
                     if (shag_i == 0 || shag_i == steps_) {
                       continue;
                     }
                     weight *= (2.0 + static_cast<double>(shag_i % 2) * 2);
                   }
                   local_res += weight * f_(param);
                 }
                 return local_res;
               },
               [](double a, double b) { return a + b; });
  });

  return res;
}

bool filateva_e_simpson_all::Simpson::RunImpl() {
  boost::mpi::broadcast(world_, mer_, 0);
  boost::mpi::broadcast(world_, steps_, 0);
  boost::mpi::broadcast(world_, a_, 0);
  const int num_proc = world_.size();
  long del = (long)std::pow(steps_ + 1, mer_) / num_proc;
  long ost = num_proc - ((long)std::pow(steps_ + 1, mer_) % num_proc);

  if (world_.rank() == 0) {
    h_.resize(mer_);
    for (size_t i = 0; i < mer_; i++) {
      h_[i] = static_cast<double>(b_[i] - a_[i]) / static_cast<double>(steps_);
    }
    res_ = 0.0;
  }
  boost::mpi::broadcast(world_, h_, 0);

  long start = (world_.rank() < ost) ? del * world_.rank() : (del * ost) + ((del + 1) * (world_.rank() - ost));
  long end = start + del + (world_.rank() < ost ? 0 : 1);
  double local_res = IntegralFunc(start, end);
  reduce(world_, local_res, res_, std::plus<>(), 0);

  if (world_.rank() == 0) {
    for (size_t i = 0; i < mer_; i++) {
      res_ *= (h_[i] / 3.0);
    }
  }
  return true;
}
\end{lstlisting}

\end{document}