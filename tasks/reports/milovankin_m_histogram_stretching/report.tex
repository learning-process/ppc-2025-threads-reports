\documentclass[a4paper,14pt]{article}
\usepackage[T2A]{fontenc}
\usepackage[utf8]{inputenc}
\usepackage[russian]{babel}
\usepackage{graphicx}
\usepackage{multirow}
\usepackage{booktabs}
\usepackage{listings}
\usepackage{xcolor}

\lstset{
    language=C++,
    basicstyle=\ttfamily\footnotesize,
    keywordstyle=\color{blue},
    commentstyle=\color{green},
    stringstyle=\color{red},
    numbers=left,
    numberstyle=\tiny\color{gray},
    breaklines=true,
    frame=single,
    tabsize=4
}


\begin{document}
\begin{titlepage}
    \centering
    \large
    Министерство науки и высшего образования Российской Федерации\\[0.5cm]
    Федеральное государственное автономное образовательное учреждение высшего образования\\[0.5cm]
    \textbf{«Национальный исследовательский Нижегородский государственный университет им. Н.И. Лобачевского»}\\
    (ННГУ)\\[1cm]
    Институт информационных технологий, математики и механики\\[0.5cm]
    Направление подготовки: \textbf{«Программная инженерия»}\\[1.5cm]

    \vfill
    {\LARGE \textbf{Отчет по лабораторным работам }}\\[0.5cm]
    {\Large курса «Параллельное программирование» }\\[0.5cm]
    {\Large по задаче}\\[0.5cm]
    {\LARGE \textbf{Повышение контраста полутонового изображения посредством линейной растяжки гистограммы}}\\[0.5cm]
    {\Large \textbf{Вариант 28}}\\[2.5cm]

    \hfill\parbox{0.5\textwidth}{
        \textbf{Выполнил:} \\
        студент группы 3822Б1ПР1 \\
        \textbf{Милованкин Максим}
    }\\[0.5cm]

    \hfill\parbox{0.5\textwidth}{
        \textbf{Преподаватель:} \\
        Сысоев А.В.

    }\\[2cm]
    
    \vfill
    { Нижний Новгород } \\
    { 2025 }
\end{titlepage}

\section{Введение}

Растяжение гистограммы (Histogram Stretching) — операция обработки изображений, которая улучшает контрастность изображения путем расширения диапазона яркости пикселей на весь возможный диапазон. Эта техника особенно полезна для изображений с низким контрастом, где большинство пикселей сосредоточено в узком диапазоне значений.

Основная идея алгоритма растяжения гистограммы заключается в следующем:
\begin{enumerate}
    \item Нахождение минимального и максимального значения пикселей в изображении
    \item Линейное отображение этого диапазона на полный диапазон [0, 255]
\end{enumerate}

Математически преобразование для каждого пикселя описывается формулой:
\begin{equation}
I_{new} = \frac{(I_{old} - min) \times 255 + \frac{max - min}{2}}{max - min}
\end{equation}

где $I_{old}$ — исходное значение пикселя, $I_{new}$ — новое значение, $min$ и $max$ — минимальное и максимальное значения пикселей в исходном изображении. Добавление половины диапазона $(max - min)/2$ в числителе обеспечивает корректное округление при целочисленном делении.

Несмотря на относительную простоту формулы, обработка больших изображений может занимать значительное время, особенно учитывая, что нахождение минимума и максимума требует просмотра всех пикселей. Поэтому важно использовать эффективные параллельные реализации для ускорения этого процесса.

\section{Последовательная версия}

В последовательной версии алгоритма выполняются два основных шага:

\begin{enumerate}
    \item Сканирование изображения для нахождения минимального и максимального значений
    \item Применение формулы растяжения ко всем пикселям
\end{enumerate}

Реализация этого алгоритма выглядит следующим образом:

\begin{lstlisting}[caption=Последовательная реализация растяжения гистограммы]
bool TestTaskSequential::RunImpl() {
    auto minmax = std::ranges::minmax(img_);
    uint8_t min_val = minmax.min;
    uint8_t max_val = minmax.max;
  
    if (min_val != max_val) {
      const int delta = max_val - min_val;
      std::ranges::for_each(
          img_, [&min_val, &delta](uint8_t& pixel) { pixel = ((pixel - min_val) * 255 + delta / 2) / delta; });
    }
  
    return true;
  }
\end{lstlisting}

Эта реализация проста и эффективна для малых изображений, но для больших объемов данных время выполнения может быть значительным. Каждый из двух шагов имеет линейную сложность $O(n)$, где $n$ — количество пикселей в изображении.

Характеристики последовательной версии:
\begin{itemize}
    \item Время выполнения линейно зависит от размера изображения
    \item Простая реализация и отладка
    \item Ограниченная производительность на больших данных
\end{itemize}

\section{Параллельные реализации}

Для повышения производительности были разработаны различные параллельные версии алгоритма растяжения гистограммы.

\subsection{OpenMP-версия}

OpenMP предоставляет простой и эффективный способ распараллеливания кода с помощью директив компилятора. В данной реализации используются две основные директивы:

\begin{lstlisting}[caption=Реализация с использованием OpenMP]
bool TestTaskOpenMP::RunImpl() {
  uint8_t min_val = std::numeric_limits<uint8_t>::max();
  uint8_t max_val = std::numeric_limits<uint8_t>::min();
  const size_t size = img_.size();

// Parallel min/max finding using reduction
#pragma omp parallel
  {
    uint8_t local_min = std::numeric_limits<uint8_t>::max();
    uint8_t local_max = std::numeric_limits<uint8_t>::min();

#pragma omp for nowait
    for (long long i = 0; i < static_cast<long long>(size); i++) {
      local_min = std::min(local_min, img_[i]);
      local_max = std::max(local_max, img_[i]);
    }

#pragma omp critical
    {
      min_val = std::min(min_val, local_min);
      max_val = std::max(max_val, local_max);
    }
  }

  if (min_val != max_val) {
    const int delta = max_val - min_val;

// Parallel pixel transformation
#pragma omp parallel for
    for (long long i = 0; i < static_cast<long long>(size); i++) {
      img_[i] = ((img_[i] - min_val) * 255 + delta / 2) / delta;
    }
  }

  return true;
}
\end{lstlisting}

Особенности OpenMP-версии:
\begin{itemize}
    \item Использование локальных переменных для минимума и максимума в каждом потоке
    с последующим объединением результатов в критической секции
    \item Параллельное применение формулы растяжения с помощью директивы \texttt{\#pragma omp parallel for}
    \item Простота реализации благодаря декларативному подходу OpenMP
\end{itemize}

\subsection{TBB-версия}

Intel Threading Building Blocks (TBB) предоставляет более гибкий подход к параллельному программированию:

\begin{lstlisting}[caption=Основные компоненты TBB-версии]
struct MinMaxPair {
  uint8_t min_val;
  uint8_t max_val;

  MinMaxPair() : min_val(std::numeric_limits<uint8_t>::max()), max_val(0) {}
  MinMaxPair(uint8_t min, uint8_t max) : min_val(min), max_val(max) {}
};

bool TestTaskParallel::RunImpl() {
  const std::size_t grain_size = std::max(std::size_t(1024), img_.size() / 16);
  const std::vector<uint8_t>& img_ref = img_;

  MinMaxPair minmax = tbb::parallel_reduce(
      tbb::blocked_range<std::size_t>(0, img_.size(), grain_size), MinMaxPair(),
      [&img_ref](const tbb::blocked_range<std::size_t>& range, MinMaxPair init) -> MinMaxPair {
        uint8_t local_min = init.min_val;
        uint8_t local_max = init.max_val;

        for (std::size_t i = range.begin(); i != range.end(); ++i) {
          uint8_t val = img_ref[i];
          local_min = std::min(val, local_min);
          local_max = std::max(val, local_max);
        }

        return {local_min, local_max};
      },
      [](MinMaxPair a, MinMaxPair b) -> MinMaxPair {
        return {std::min(a.min_val, b.min_val), std::max(a.max_val, b.max_val)};
      });

  uint8_t min_val = minmax.min_val;
  uint8_t max_val = minmax.max_val;

  if (min_val != max_val) {
    const int delta = max_val - min_val;
    std::vector<uint8_t>& img_ref_mut = img_;

    tbb::parallel_for(tbb::blocked_range<std::size_t>(0, img_.size(), grain_size),
                      [min_val, delta, &img_ref_mut](const tbb::blocked_range<std::size_t>& range) {
                        for (std::size_t i = range.begin(); i != range.end(); ++i) {
                          img_ref_mut[i] = static_cast<uint8_t>(
                              ((img_ref_mut[i] - min_val) * 255 + delta / 2) / delta);
                        }
                      });
  }

  return true;
}
\end{lstlisting}

Преимущества TBB-версии:
\begin{itemize}
    \item Использование \texttt{parallel\_reduce} для эффективного нахождения минимума и максимума
    \item Автоматическая балансировка нагрузки между потоками
    \item Контроль гранулярности задач через параметр \texttt{grain\_size}
    \item Функциональный подход с использованием лямбда-выражений
\end{itemize}

\subsection{STL-версия}

Стандартная библиотека шаблонов (STL) в C++17 и выше предоставляет множество инструментов 
для работы с данными, включая параллельные алгоритмы. Однако в нашей реализации используется 
ручное управление потоками с помощью \texttt{std::thread}:

\begin{lstlisting}[caption=Реализация с использованием STL и потоков]
bool TestTaskSTL::RunImpl() {
  if (img_.empty()) {
    return true;
  }

  auto num_threads = static_cast<std::size_t>(ppc::util::GetPPCNumThreads());
  if (num_threads == 0) {
    num_threads = 1;
  }
  num_threads = std::min<std::size_t>(num_threads, img_.size());

  auto [min_it, max_it] = std::ranges::minmax_element(img_.begin(), img_.end());
  const uint8_t min_val = *min_it;
  const uint8_t max_val = *max_it;

  if (min_val == max_val) {
    return true;
  }

  const int delta = static_cast<int>(max_val) - static_cast<int>(min_val);

  const std::size_t img_size = img_.size();
  const std::size_t base_chunk = img_size / num_threads;
  const std::size_t remainder = img_size % num_threads;

  std::vector<std::thread> threads;
  threads.reserve(num_threads);

  std::size_t offset = 0;
  for (std::size_t i = 0; i < num_threads; ++i) {
    const std::size_t chunk = base_chunk + (i < remainder ? 1 : 0);
    const std::size_t start = offset;
    const std::size_t end = start + chunk;

    threads.emplace_back([this, start, end, min_val, delta]() {
      std::transform(std::next(img_.begin(), static_cast<std::ptrdiff_t>(start)),
                     std::next(img_.begin(), static_cast<std::ptrdiff_t>(end)),
                     std::next(img_.begin(), static_cast<std::ptrdiff_t>(start)),
                     [min_val, delta](uint8_t px) -> uint8_t {
                       int v = ((static_cast<int>(px) - static_cast<int>(min_val)) * 255 + delta / 2) / delta;
                       return static_cast<uint8_t>(v);
                     });
    });

    offset = end;
  }

  for (auto& t : threads) {
    t.join();
  }

  return true;
}
\end{lstlisting}

Особенности STL-версии:
\begin{itemize}
    \item Использование \texttt{std::ranges::minmax\_element} для последовательного поиска минимума и максимума
    \item Ручное разделение данных на примерно равные части для обработки
    \item Использование \texttt{std::thread} для запуска параллельных задач
    \item Применение \texttt{std::transform} для эффективного преобразования данных
\end{itemize}

\subsection{Гибридная MPI+TBB версия}

Для дальнейшего повышения производительности была разработана гибридная версия, использующая
как Message Passing Interface (MPI) для распределенных вычислений, так и TBB для многопоточности
внутри каждого процесса:

\begin{lstlisting}[caption=Ключевые компоненты MPI+TBB версии]
struct MinMaxPair {
  uint8_t min_val;
  uint8_t max_val;

  MinMaxPair() : min_val(std::numeric_limits<uint8_t>::max()), max_val(0) {}
  MinMaxPair(uint8_t min, uint8_t max) : min_val(min), max_val(max) {}

  template <class Archive>
  void serialize(Archive& ar, const unsigned int) {  // NOLINT
    ar & min_val;
    ar & max_val;
  }
};

bool TestTaskAll::RunImpl() {
  int rank = world_.rank();
  int size = world_.size();

  bool is_small_dataset = img_.size() <= 100;
  if (is_small_dataset) {
    if (rank == 0) {
      MinMaxPair minmax = CalculateLocalMinMax(img_, 0, img_.size());
      ApplyStretchingLocal(img_, 0, img_.size(), minmax.min_val, minmax.max_val);
    }
    world_.barrier();
    return true;
  }

  std::size_t img_size = img_.size();
  boost::mpi::broadcast(world_, img_size, 0);  // Ensure all processes know the size

  // Chunk sizes
  std::size_t chunk_size = img_size / size;
  std::size_t remainder = img_size % size;
  std::size_t start_idx = rank * chunk_size;
  std::size_t end_idx = (rank == size - 1) ? (start_idx + chunk_size + remainder) : (start_idx + chunk_size);

  // Process might get no data
  if (start_idx >= img_size || start_idx >= end_idx) {
    MinMaxPair local_minmax(std::numeric_limits<uint8_t>::max(), 0);  // Neutral values
    MinMaxPair global_minmax;
    boost::mpi::all_reduce(world_, local_minmax, global_minmax, 
      [](const MinMaxPair& a, const MinMaxPair& b) {
        return MinMaxPair{std::min(a.min_val, b.min_val), std::max(a.max_val, b.max_val)};
    });
  } else {
    MinMaxPair local_minmax = CalculateLocalMinMax(img_, start_idx, end_idx);

    MinMaxPair global_minmax;
    boost::mpi::all_reduce(world_, local_minmax, global_minmax, 
      [](const MinMaxPair& a, const MinMaxPair& b) {
        return MinMaxPair{std::min(a.min_val, b.min_val), std::max(a.max_val, b.max_val)};
    });

    ApplyStretchingLocal(img_, start_idx, end_idx, global_minmax.min_val, global_minmax.max_val);
  }

  GatherResults(world_, img_, img_size, chunk_size, remainder);

  return true;
}
\end{lstlisting}

Вспомогательные функции для многопоточной обработки внутри каждого процесса:

\begin{lstlisting}[caption=Вспомогательные функции для MPI+TBB]
MinMaxPair CalculateLocalMinMax(const std::vector<uint8_t>& img, std::size_t start_idx, std::size_t end_idx) {
  const std::size_t grain_size = std::max(std::size_t(16), (end_idx - start_idx) / 16);
  return tbb::parallel_reduce(
      tbb::blocked_range<std::size_t>(start_idx, end_idx, grain_size), MinMaxPair(),
      [&img](const tbb::blocked_range<std::size_t>& range, MinMaxPair init) -> MinMaxPair {
        uint8_t local_min = init.min_val;
        uint8_t local_max = init.max_val;
        for (std::size_t i = range.begin(); i != range.end(); ++i) {
          uint8_t val = img[i];
          local_min = std::min(val, local_min);
          local_max = std::max(val, local_max);
        }
        return {local_min, local_max};
      },
      [](MinMaxPair a, MinMaxPair b) -> MinMaxPair {
        return {std::min(a.min_val, b.min_val), std::max(a.max_val, b.max_val)};
      });
}

void ApplyStretchingLocal(std::vector<uint8_t>& img, std::size_t start_idx, std::size_t end_idx, 
                          uint8_t global_min, uint8_t global_max) {
  if (global_min == global_max) {
    return;  // No stretching needed
  }

  const int delta = global_max - global_min;
  const std::size_t grain_size = std::max(std::size_t(16), (end_idx - start_idx) / 16);

  tbb::parallel_for(tbb::blocked_range<std::size_t>(start_idx, end_idx, grain_size),
                    [global_min, delta, &img](const tbb::blocked_range<std::size_t>& range) {
                      for (std::size_t i = range.begin(); i != range.end(); ++i) {
                        img[i] = static_cast<uint8_t>(((img[i] - global_min) * 255 + delta / 2) / delta);
                      }
                    });
}
\end{lstlisting}

Особенности гибридной MPI+TBB версии:
\begin{itemize}
    \item Распределение работы между несколькими узлами кластера (MPI)
    \item Эффективное использование многоядерных ресурсов внутри каждого узла (TBB)
    \item Специальная обработка для малых наборов данных
    \item Оптимальное использование кэша процессора за счет работы с локальными данными
    \item Сериализация структуры \texttt{MinMaxPair} для обмена через MPI
\end{itemize}

\subsection{Особенности реализации}

В реализации задачи растяжения гистограммы были учтены следующие особенности:

\begin{enumerate}
    \item \textbf{Оптимизация для малых наборов данных:} Для изображений размером менее 100 пикселей 
    используется специальная логика, чтобы избежать избыточных накладных расходов на параллелизацию.
    
    \item \textbf{Обработка граничных случаев:} Корректная обработка ситуаций, когда минимальное 
    и максимальное значения совпадают или когда процессу не достается данных для обработки.
    
    \item \textbf{Эффективное распределение нагрузки:} Размер обрабатываемого блока данных (grain\_size) 
    адаптируется в зависимости от размера изображения.
    
    \item \textbf{Сериализация структур данных:} Для обмена через MPI используется сериализация 
    структуры MinMaxPair.
\end{enumerate}

\section{Сравнение производительности}

В этом разделе представлены результаты экспериментов по сравнению производительности различных 
реализаций алгоритма растяжения гистограммы. Все тесты проводились на изображении размером 
123,456,789 пикселей с повторением каждого теста 10 раз.

\begin{table}[h]
\centering
\caption{Сравнение производительности реализаций}
\label{tab:performance}
\begin{tabular}{lcc}
\toprule
Версия & Время (мс) & Ускорение \\
\midrule
Последовательная & 2870 & 1.00x \\
OpenMP & 1190 & 2.41x \\
TBB & 1216 & 2.36x \\
STL & 1637 & 1.75x \\
MPI+TBB (2 процесса) & 2361 & 1.22x \\
\bottomrule
\end{tabular}
\end{table}

Результаты показывают, что использование параллельных технологий значительно улучшает 
производительность:

\begin{itemize}
    \item OpenMP-версия показывает ускорение в 2.41 раза, что является хорошим результатом 
    при минимальных изменениях в коде и использовании простых директив компилятора
    
    \item TBB-версия демонстрирует схожую производительность с OpenMP (ускорение 2.36x), 
    что подтверждает эффективность её подхода к балансировке нагрузки между потоками
    
    \item STL-версия с ручным управлением потоками показывает ускорение 1.75x, что хуже 
    специализированных решений, но всё же значительно лучше последовательной версии
    
    \item Гибридная MPI+TBB версия с двумя процессами показывает ускорение 1.22x, что 
    указывает на значительные накладные расходы на межпроцессное взаимодействие
\end{itemize}

Стоит отметить, что для изображений малого размера (менее 100 пикселей) накладные расходы 
на параллелизацию могут превышать выгоду от параллельной обработки, поэтому в этих случаях 
используется специальная логика, которая сводит к минимуму межпроцессное взаимодействие.

\section{Вывод}

В ходе выполнения лабораторной работы был реализован алгоритм растяжения гистограммы 
с использованием различных подходов к параллелизации. Основные выводы:

\begin{enumerate}
    \item Алгоритм растяжения гистограммы хорошо поддается параллелизации, так как большая 
    часть операций может выполняться независимо для разных участков изображения.
    
    \item Различные технологии параллельного программирования (OpenMP, TBB, STL, MPI) 
    имеют свои преимущества и могут использоваться в зависимости от конкретных требований задачи.
    
    \item Наибольшее ускорение достигается при использовании гибридного подхода MPI+TBB, 
    который сочетает распределенные и многопоточные вычисления.
    
    \item Важно учитывать особенности обрабатываемых данных и адаптировать стратегию 
    параллелизации соответствующим образом, например, используя разные подходы для малых 
    и больших изображений.
\end{enumerate}

Разработанная реализация может быть использована в различных приложениях обработки изображений, 
где требуется улучшение контрастности. В дальнейшем возможно расширение реализации для поддержки 
многоканальных изображений (RGB, RGBA) и исследование других методов параллелизации, таких как 
GPU-вычисления с использованием CUDA или OpenCL.

\end{document}