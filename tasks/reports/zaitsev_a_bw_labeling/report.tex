\documentclass[a4paper,14pt]{article}
\usepackage[T2A]{fontenc}
\usepackage[utf8]{inputenc}
\usepackage[russian]{babel}
\usepackage{graphicx}
\usepackage{multirow}
\usepackage{booktabs}
\usepackage{listings}
\usepackage{xcolor}

\lstset{
    language=C++,
    basicstyle=\ttfamily\footnotesize,
    keywordstyle=\color{blue},
    commentstyle=\color{green},
    stringstyle=\color{red},
    numbers=left,
    numberstyle=\tiny\color{gray},
    breaklines=true,
    frame=single,
    tabsize=4
}


\begin{document}
\begin{titlepage}
    \centering
    \large
    Министерство науки и высшего образования Российской Федерации\\[0.5cm]
    Федеральное государственное автономное образовательное учреждение высшего образования\\[0.5cm]
    \textbf{«Национальный исследовательский Нижегородский государственный университет им. Н.И. Лобачевского»}\\
    (ННГУ)\\[1cm]
    Институт информационных технологий, математики и механики\\[0.5cm]
    Направление подготовки: \textbf{«Программная инженерия»}\\[1.5cm]

    \vfill
    {\LARGE \textbf{Отчет по лабораторным работам }}\\[0.5cm]
    {\Large курса «Параллельное программирование» }\\[0.5cm]
    {\Large по задаче}\\[0.5cm]
    {\LARGE \textbf{«Маркировка компонент на бинарном изображении»}}\\[0.5cm]
    {\Large \textbf{Вариант №29}}\\[2.5cm]

    \hfill\parbox{0.5\textwidth}{
        \textbf{Выполнил:} \\
        студент группы 3822Б1ПР1 \\
        \textbf{Зайцев Артём}
    }\\[0.5cm]

    \hfill\parbox{0.5\textwidth}{
        \textbf{Преподаватель:} \\
        Сысоев А.В.

    }\\[2cm]
    
    \vfill
    { Нижний Новгород } \\
    { 2025 }
\end{titlepage}

\section{Введение}

Задача маркировки связных компонент (Connected Components Labeling) - фундаментальная операция в обработке изображений.

Базовый алгоритм маркировки:
\begin{enumerate}
    \item Первый проход:
    \begin{itemize}
        \item[-] Сканирование изображения построчно
        \item[-] Назначение временных меток
        \item[-] Запись эквивалентностей
    \end{itemize}
    \item Второй проход:
    \begin{itemize}
        \item[-] Разрешение эквивалентностей
        \item[-] Присвоение окончательных меток
    \end{itemize}
\end{enumerate}

Для правильной обработки эквивалентностей используется Union-Find структура данных - DisjointSet. Она реализует логарифмически-сложный поиск элемента по нескольким множествам и логарифмическое объединение множеств в одно. 
\begin{lstlisting}[caption=Disjoint Set (система непересекающихся множеств)]
template <typename T>
class DisjointSet {
  std::vector<T> rank_, size_, parent_;

 public:
  DisjointSet(T n = 10) {
    rank_.resize(n + 1, 0);
    size_.resize(n + 1, 1);
    for (unsigned long long i = 0; i <= n; i++) {
      parent_.push_back(i);
    }
  }

  T FindParent(T n) {
    if (parent_[n] == n) return n;
    return parent_[n] = FindParent(parent_[n]);
  }

  void UnionRank(T n1, T n2) {
    T ulp_u = FindParent(n1);
    T ulp_v = FindParent(n2);
    if (ulp_u == ulp_v) return;
    
    if (rank_[ulp_u] < rank_[ulp_v]) {
      parent_[ulp_u] = parent_[ulp_v];
    } else {
      parent_[ulp_v] = parent_[ulp_u];
      if (rank_[ulp_u] == rank_[ulp_v]) rank_[ulp_v]++;
    }
  }
};
\end{lstlisting}

\section{ Последовательная версия }
В последовательной версии реализован наивный двухпроходный алгоритм с линейным пересчетом меток. Основная часть алгоритма, требующая внимания - первый растровый проход, представлена ниже:

\begin{lstlisting}[caption=Основная логика последовательной версии]
void LabelingRasterScan(std::map<std::uint16_t, std::set<std::uint16_t>>& eqs,
                       std::uint16_t& current_label) {
  for (uint32_t i = 0; i < image_.size(); i++) {
    if (image_[i] == 0) continue;
    
    std::vector<std::uint16_t> neighbours;
    for (int shift = 0; shift < 4; shift++) {
      long x = ((long)i % width_) + (shift % 3 - 1);
      long y = ((long)i / width_) + (shift / 3 - 1);
      long neighbour_index = x + (y * width_);
      if (x >= 0 && x < (long)width_ && y >= 0) {
        if (labels_[neighbour_index] != 0) {
          neighbours.push_back(labels_[neighbour_index]);
        }
      }
    }

    if (neighbours.empty()) {
      labels_[i] = ++current_label;
      eqs[current_label].insert(current_label);
    } else {
      labels_[i] = *std::min_element(neighbours.begin(), neighbours.end());
      for (auto& first : neighbours) {
        for (auto& second : neighbours) {
          eqs[first].insert(second);
        }
      }
    }
  }
}
\end{lstlisting}

Когда все временные метки посчитаны, алгоритм генерирует DisjointSet из списка эквивалентностей, чтобы разделить метки на множества эквивалентных меток:

\begin{lstlisting}[caption=Основная логика последовательной версии]
void SecondRasterScan() {
  DisjointSet<std::uint16_t> disjoint_labels(current_label + 1);
  for (auto& statement : eqs) {
    for (const auto& equal : statement.second) {
      disjoint_labels.UnionRank(statement.first, equal);
    }
  }

  replacements.resize(current_label + 1);
  std::set<std::uint16_t> unique_labels;

  for (std::uint16_t tmp_label = 1; tmp_label < current_label + 1; tmp_label++) {
    replacements[tmp_label] = disjoint_labels.FindParent(tmp_label);
    unique_labels.insert(replacements[tmp_label]);
  }

  std::uint16_t true_label = 0;
  std::map<std::uint16_t, std::uint16_t> reps;
  for (const auto& it : unique_labels) {
    reps[it] = ++true_label;
  }

  for (uint32_t i = 0; i < replacements.size(); i++) {
    replacements[i] = reps[replacements[i]];
  }

  for (uint32_t i = 0; i < size_; i++) {
    labels_[i] = replacements[labels_[i]];
  }
}
\end{lstlisting}

После проделанных действий, мы получаем flat-массив, соответствующий искомой матрице меток.

Характеристики:
\begin{itemize}
    \item Среднее время выполнения: 1112 мс
    \item Простота реализации и отладки
    \item Линейное потребление памяти
\end{itemize}
Замечание: здесь и далее среднее время выполнения приводится как среднее время однократного исполнения кода для входных данных объемом в 2 млн пикселей среди десяти тысяч повторений запуска. Замеры производились при запуске программы на 16 потоках и 2 процессах (где присутствует MPI)

\section{Параллельные реализации}
Все представленные параллельные реализации выполнены на основе последовательной версии. Повторяющиеся или незначительные участки кода в отчете указаны не будут.
\subsection{OMP-версия}
Ключевое отличие OMP-версии показано в листинге 4:

\begin{lstlisting}[caption=Ключевые особенности OMP версии]
#pragma omp declare reduction(merge:Vec16t : std::transform(omp_out.begin(), omp_out.end(), omp_in.begin(), omp_out.begin(), [](const auto& x, const auto& y){return std::max(x, y);})) initializer(omp_priv = decltype(omp_orig)(omp_orig.size()))
void Labeler::LabelingRasterScan(Equivalency& eqs_list, Vec16t& current_labels_list) {
#ifndef _WIN32
#pragma omp parallel if (static_cast<int>(height_) >= 2 * ppc::util::GetPPCNumThreads()) reduction(merge : lbls)
#else
#pragma omp parallel if (static_cast<int>(height_) >= 2 * ppc::util::GetPPCNumThreads())
#endif
  {
    std::uint8_t id = omp_get_thread_num();
    std::uint32_t i_begin = id * chunk_;
    std::uint32_t i_end = i_begin + chunk_;
    i_end = std::min(i_end, size_);

    <sequential version FirstRasterScan>
    }
    
  labels_ = std::move(lbls);
\end{lstlisting}
Можно заметить, что перед началом маркировки компонент исходное изображение делится на несколько других. После чего, метки параллельно считаются для каждого изображения. Затем они объединяются в единое изображение, величины меток сдвигаются в соответствии с наибольшей меткой предыдущего изображения из разбиения, после чего применяется второе сканирование: эквивалентности преобразуются в непересекающиеся множества эквивалентных меток и унифицируются на конечной матрице.

Особенности:
\begin{itemize}
    \item Разделение изображения на горизонтальные полосы
    \item Локальные хранилища эквивалентностей и меток для каждого потока
    \item Среднее время выполнения: 805 мс
\end{itemize}

\subsection{TBB-версия}
Реализация с использованием Intel Threading Building Blocks с точки зрения алгоритма распараллеливания ничем не отличается от версии OMP. Параллельный участок кода указан ниже.

\begin{lstlisting}[caption=Параллельная обработка в TBB]
oneapi::tbb::task_group tg;
for (int i = 0; i < ppc::util::GetPPCNumThreads(); i++) {
  tg.run([i, &lbls, &eqs, &ordinals, this] {
    FirstScan(std::ref(lbls[i]), std::ref(eqs[i]), std::ref(ordinals[i]), width_);
  });
}
tg.wait();
\end{lstlisting}

Функция FirstScan полный аналог функции LabelingRasterScan из вышерассмотренных версий программы

Преимущества:
\begin{itemize}
    \item Более эффективное планирование по сравнению с OMP
    \item Среднее время выполнения: 835 мс
\end{itemize}

\subsection{STL-версия}
Реализация с использованием средств стандартной библиотеки аналогична TBB и OMP:

\begin{lstlisting}[caption=Параллелизация через std::thread]
std::vector<std::thread> threads_list;
for (int i = 0; i < ppc::util::GetPPCNumThreads(); i++) {
  threads_list.emplace_back(FirstScan, std::ref(lbls[i]), 
                          std::ref(eqs[i]), std::ref(ordinals[i]), width_);
}
std::ranges::for_each(threads_list, [](auto& thread) { thread.join(); });
\end{lstlisting}

Особенности:
\begin{itemize}
    \item Высокие накладные расходы на управление потоками
    \item Среднее время выполнения: 1003 мс
\end{itemize}

\subsection{Гибридная MPI+TBB версия}
Комбинированная реализация отличается лишь двойный разделением: сначала по процессам, затем по потоком. В остальном код соответствует предыдущим программам: 

\begin{lstlisting}[caption=Ключевые компоненты MPI+TBB]
boost::mpi::scatterv(world_, image_.data(), counts_, displs_, 
                    local_image.data(), local_height * width, 0);

oneapi::tbb::parallel_for(HardRange(size_, chunk_), 
                         FirstScan(image_, labels_, ordinals, width_, height_));

boost::mpi::gatherv(world_, labels_.data(), static_cast<int>(labels_.size()),
                   labels.data(), counts_, displs_, 0);
\end{lstlisting}

Особенности:
\begin{itemize}
    \item MPI для распределения работы между процессами
    \item TBB для параллельной обработки внутри потоков
    \item Сложная логика объединения результатов
    \item Среднее время выполнения: 623 мс
\end{itemize}

\section{Сравнение производительности}

\begin{table}[h]
\centering
\caption{Сравнение производительности реализаций}
\label{tab:performance}
\begin{tabular}{lcc}
\toprule
Версия & Время (мс) & Ускорение \\
\midrule
Seq & 1112 & 1.00x \\
OMP & 805 & 1.98x \\
TBB & 835 & 1.33x \\
STL & 1003 & 1.08x \\
MPI+TBB & 623 & 1.78x \\
\bottomrule
\end{tabular}
\end{table}

Ключевые наблюдения:
\begin{itemize}
    \item TBB показывает лучшую производительность на двух узлах
    \item Гибридный подход MPI+TBB дает максимальное ускорение
    \item OMP версия наиболее эффективна за счёт встроенной оптимизации для циклов
\end{itemize}

\section{Вывод}
В ходе работы были реализованы и протестированы различные версии алгоритма маркировки связных компонент. Основные выводы:

\begin{itemize}
    \item Параллельные версии обеспечивают ускорение от 10\% до 40\%
    \item Для задач на двух узлах оптимальна TBB-реализация
\end{itemize}


\end{document}
