\documentclass{report}

\usepackage[T2A]{fontenc}
\usepackage[utf8]{inputenc}
\usepackage[english, russian]{babel}
\usepackage[14pt]{extsizes}
\usepackage{listings}
\usepackage{xcolor}
\usepackage{geometry}
\usepackage{enumitem}
\usepackage{amsmath}
\usepackage{titlesec}
\usepackage{ragged2e}
\usepackage{hyperref}

\geometry{a4paper,top=2cm,bottom=3cm,left=2cm,right=1.5cm}
\setlength{\parskip}{0.5cm}
\setlist{nolistsep, itemsep=0.3cm,parsep=0pt}

% Оформление listings
\lstset{
    language=C++, 
    basicstyle=\ttfamily\small, 
    keywordstyle=\color{blue}\bfseries, 
    stringstyle=\color{orange}, 
    commentstyle=\color{green!50!black}\itshape, 
    morecomment=[l][\color{purple}]{\#}, 
    tabsize=2, 
    breaklines=true, 
    breakatwhitespace=false, 
    frame=single, 
    showstringspaces=false, 
}

% Заголовки без "Глава" и меньший отступ
\titleformat{\section}
  {\normalfont\normalsize\bfseries}
  {\thesection.}{0.5em}{}
\titlespacing{\section}{0pt}{1em}{0.5em}

\begin{document}

% --- Титульный лист ---
\begin{titlepage}
\begin{center}
Министерство науки и высшего образования Российской Федерации
\end{center}

\begin{center}
Федеральное государственное автономное образовательное учреждение высшего образования \\
Национальный исследовательский Нижегородский государственный университет им. Н.И. Лобачевского
\end{center}

\begin{center}
Институт информационных технологий, математики и механики
\end{center}

\vspace{4em}

\begin{center}
\textbf{\Large Отчет по лабораторной работе:} \\
\end{center}
\begin{center}
\textbf{\Large «Выделение рёбер на изображении с использованием оператора Собеля»} \\
\end{center}

\vspace{4em}

\hfill\parbox{7cm}{
\textbf{Выполнил:} \\ 
студент группы 3822Б1ФИ3 \\ 
Фролова Елизавета\\
\\
\textbf{Проверил:}\\ 
доцент кафедры МОСТ, \\ 
кандидат технических наук \\ 
Сысоев А.В.\\}

\vspace{\fill}

\begin{center} Нижний Новгород \\ 2025 \end{center}
\end{titlepage}


% --- Содержание (компактно) ---
\newpage
{
\titleformat{\chapter}[display]
  {\normalfont\large\bfseries\filcenter} % Изменил \normalsize на \large
  {\chaptertitlename\ \thechapter}{0pt}{}
\titlespacing*{\chapter}{0pt}{-30pt}{20pt} % Настройка отступов

\renewcommand{\contentsname}{\large Содержание} % Уменьшенный заголовок
\tableofcontents
}
\newpage
\setcounter{page}{2}

% --- 1. Введение ---
\section*{1. Введение}
\addcontentsline{toc}{section}{1. Введение}

\justifying
Выделение рёбер на изображениях является одной из ключевых задач в обработке и анализе визуальных данных. Оператор Собеля, основанный на вычислении градиентов яркости, широко применяется для выявления резких переходов и границ объектов. В рамках данной лабораторной работы оператор Собеля реализуется и оптимизируется с использованием различных параллельных технологий и библиотек, таких как OpenMP, TBB, STL и комбинация MPI с TBB. Это позволяет сравнить эффективность и производительность различных подходов к параллельной обработке изображений.

\subsection*{Постановка задачи}
\addcontentsline{toc}{subsection}{Постановка задачи}

\justifying
Целью лабораторной работы является реализация и исследование алгоритма выделения рёбер на изображениях с использованием оператора Собеля. Для достижения этой цели необходимо:

\begin{itemize}
    \item Разработать последовательную версию алгоритма.
    \item Реализовать параллельные версии с использованием технологий:
    \begin{itemize}
        \item OpenMP,
        \item Intel TBB,
        \item стандартной библиотеки шаблонов (STL) с параллельными алгоритмами,
        \item гибридного подхода MPI + TBB.
    \end{itemize}
    \item Обеспечить корректное выделение рёбер в выходных изображениях в соответствии с методом Собеля.
    \item Провести оценку производительности реализованных версий, сравнить время выполнения и определить ускорение при использовании параллелизма.
\end{itemize}

Входными данными являются цифровые изображения, выходными — обработанные изображения с выделенными рёбрами.


\newpage
\section*{2. Описание алгоритма}
\addcontentsline{toc}{section}{2. Описание алгоритма}

\justifying
Алгоритм выделения рёбер на изображении с использованием оператора Собеля представляет собой двухэтапную обработку входных данных: преобразование цветного изображения в градации серого и последующее применение свёртки с ядрами Собеля для вычисления градиента яркости.

\textbf{1. Преобразование в оттенки серого.} \\
Поскольку оператор Собеля работает с интенсивностью пикселей, входное изображение сначала преобразуется из цветного (RGB) формата в градации серого. Для этого используется формула взвешенного среднего:

\[
I(x, y) = 0.299 \cdot R(x, y) + 0.587 \cdot G(x, y) + 0.114 \cdot B(x, y)
\]

где \( R, G, B \) — значения цветовых каналов пикселя, а \( I(x, y) \) — соответствующее значение яркости в сером изображении.

\textbf{2. Вычисление градиента методом Собеля.} \\
Для каждого пикселя (за исключением границ) вычисляется приближённый градиент яркости в горизонтальном и вертикальном направлениях с помощью двумерной свёртки. Используются следующие ядра (маски) Собеля:

\[
G_x =
\begin{bmatrix}
-1 & 0 & 1 \\
-2 & 0 & 2 \\
-1 & 0 & 1 \\
\end{bmatrix}
\quad
G_y =
\begin{bmatrix}
-1 & -2 & -1 \\
\ \ 0 & \ \ 0 & \ \ 0 \\
\ 1 &\ \ 2 &\ \ 1 \\
\end{bmatrix}
\]

Градиент по оси X (горизонтальный) и по оси Y (вертикальный) вычисляются по формулам:

\[
G_x(x, y) = \sum_{i=-1}^{1} \sum_{j=-1}^{1} G_x(i+1, j+1) \cdot I(x+i, y+j)
\]
\[
G_y(x, y) = \sum_{i=-1}^{1} \sum_{j=-1}^{1} G_y(i+1, j+1) \cdot I(x+i, y+j)
\]

\textbf{3. Расчёт итогового градиента.} \\
Суммарный градиент в каждой точке определяется как евклидова норма:

\[
G(x, y) = \sqrt{G_x(x, y)^2 + G_y(x, y)^2}
\]

Значение градиента округляется до целого и ограничивается в диапазоне \([0, 255]\), чтобы соответствовать диапазону яркости на изображении.

\textbf{4. Граничные условия.} \\
При обработке краёв изображения применяется проверка выхода индексов за границы. Если координаты соседнего пикселя выходят за пределы изображения, его вклад в расчёты игнорируется.

\textbf{5. Результат.} \\
Результатом является одноканальное изображение, на котором яркие пиксели соответствуют участкам с резким изменением яркости, то есть границам объектов на исходном изображении.


% --- 3. Схема параллельного алгоритма ---
\newpage
\section*{3. Схема параллельного алгоритма}
\addcontentsline{toc}{section}{3. Схема параллельного алгоритма}

\justifying
Параллельная реализация оператора Собеля основана на независимости обработки отдельных пикселей, что позволяет эффективно распараллелить вычисления по различным участкам изображения. Основная идея заключается в разделении изображения на блоки или строки, которые обрабатываются параллельно различными потоками.

\textbf{1. Единица распараллеливания.} \\
Наиболее естественным подходом является построчная обработка изображения, при которой каждая строка (или группа строк) передаётся отдельному потоку. Такая стратегия упрощает доступ к данным и минимизирует накладные расходы на синхронизацию.

\textbf{2. Распараллеливание в разных технологиях:}
\begin{itemize}
    \item \textbf{OpenMP:} используется директива \texttt{\#pragma omp parallel for} для параллельного прохода по строкам изображения.
    \item \textbf{TBB (Intel Threading Building Blocks):} применяется \texttt{tbb::parallel\_for} для распределения итераций по строкам изображения. Библиотека автоматически управляет пулом потоков и обеспечивает эффективную балансировку нагрузки.
    \item \textbf{STL с execution policy:} применяется стандартный алгоритм \texttt{std::for\_each} с параллельной политикой исполнения для ускоренной обработки пикселей.
    \item \textbf{MPI + TBB:} изображение разделяется между несколькими процессами (MPI), каждый из которых далее распараллеливает обработку своей части с помощью TBB. После обработки данные собираются обратно на ведущем процессе.
\end{itemize}

\textbf{3. Синхронизация.} \\
Так как каждый пиксель обрабатывается независимо (не меняет соседние значения), необходимости в сложной синхронизации между потоками нет. Главное — обеспечить корректное чтение граничных пикселей, особенно при разделении изображения на блоки.

\textbf{4. Балансировка нагрузки.} \\
Для больших изображений важно равномерно распределять строки или блоки между потоками, чтобы избежать простоев и неравномерной загрузки ядер. TBB и STL позволяют автоматически управлять балансом, тогда как в OpenMP и MPI распределение нужно настраивать вручную.

\textbf{5. Особенности границ.} \\
При делении изображения на блоки важно обеспечить доступ к соседним строкам (по краям блоков) для корректного применения ядра Собеля. Это решается либо расширением блоков (добавлением строк «в запас»), либо использованием буферов и обменов при работе с MPI.

\textbf{Результат.} \\
Параллельный алгоритм обеспечивает ускорение обработки изображений за счёт эффективного использования ресурсов многоядерных процессоров или многопроцессорных систем, при этом сохраняется корректность результата, идентичного последовательной реализации.

% --- 4. Описание OpenMP-версии алгоритма ---
\newpage
\section*{4. Описание OpenMP-версии алгоритма}
\addcontentsline{toc}{section}{4. Описание OpenMP-версии алгоритма}

OpenMP (Open Multi-Processing) — это API для поддержки многопоточного программирования на уровне директив компилятора. Он позволяет удобно распараллеливать циклы и фрагменты кода, распределяя вычисления между несколькими потоками на основе общей памяти.

В данной реализации алгоритма Собеля распараллеливание выполняется с помощью директивы \texttt{\#pragma omp parallel for}, которая указывает компилятору распределить итерации цикла между потоками. OpenMP автоматически создаёт пул потоков и управляет их синхронизацией, что позволяет эффективно ускорить обработку изображения.

Параллельная обработка заключается в распределении по потокам строк изображения. Каждый поток независимо вычисляет горизонтальные и вертикальные компоненты градиента яркости для пикселей, обходя их окрестность размером \(3 \times 3\) и применяя свёрточные маски оператора Собеля. Поскольку вычисления для каждого пикселя независимы, данный подход обеспечивает безопасное и эффективное распараллеливание.

OpenMP упрощает реализацию параллелизма, избавляя от необходимости ручного управления потоками и синхронизации, что делает его удобным инструментом для ускорения вычислительно интенсивных задач обработки изображений.


% --- 5. Описание TBB-версии алгоритма ---
\newpage  
\section*{5. Описание TBB-версии алгоритма}  
\addcontentsline{toc}{section}{5. Описание TBB-версии алгоритма}  
\justifying  

Intel Threading Building Blocks (TBB) — это высокоуровневая библиотека для параллельного программирования на C++, разработанная компанией Intel. В отличие от OpenMP, TBB реализует параллелизм на основе задач (task-based parallelism), а не потоков. Такой подход обеспечивает гибкость, масштабируемость и эффективное использование ресурсов многопроцессорных систем. Библиотека самостоятельно управляет пулом потоков и динамически распределяет задачи между доступными ядрами.

В реализации оператора Собеля с использованием TBB параллелизация выполняется с помощью конструкции \texttt{tbb::parallel\_for} и одномерного диапазона \texttt{tbb::blocked\_range}. Задача разбивается на диапазоны строк изображения, которые обрабатываются параллельно. Каждому диапазону назначается независимая задача, в рамках которой происходит обработка пикселей, включая выборку соседних значений и применение свёрточных масок Собеля.

Внутри каждой задачи вычисляются горизонтальная и вертикальная составляющие градиента яркости, после чего рассчитывается модуль градиента, отражающий интенсивность изменения изображения в данной точке. Результаты записываются в выходной массив.

Основное преимущество TBB заключается в его способности адаптироваться к архитектуре конкретного процессора, автоматически определяя оптимальное количество задач и равномерно распределяя нагрузку. Это делает TBB эффективным инструментом для реализации параллельных алгоритмов обработки изображений, особенно в случаях, когда важно достичь высокой производительности без необходимости ручного управления потоками.


% --- 6. Описание STL-версии алгоритма ---
\newpage
\section*{6. Описание STL-версии алгоритма}
\addcontentsline{toc}{section}{6. Описание STL-версии алгоритма}
\justifying

В STL-версии алгоритма оператор Собеля реализован с использованием стандартной библиотеки потоков C++ — \texttt{std::thread}. Такой подход предоставляет более низкоуровневый контроль над распараллеливанием по сравнению с высокоуровневыми библиотеками, такими как TBB или OpenMP, и позволяет вручную управлять разбиением данных и созданием потоков.

Основной этап заключается в применении оператора Собеля. Изображение делится на участки по строкам, и каждому потоку назначается определённый диапазон. Количество потоков определяется в зависимости от числа доступных вычислительных ресурсов с помощью функции \texttt{ppc::util::GetPPCNumThreads()}.

Далее создаётся вектор потоков, каждый из которых выполняет свою часть обработки. Внутри потока происходит обход назначенных строк изображения, выборка значений соседних пикселей с помощью функции \texttt{GetPixelSafe}, применение горизонтальных и вертикальных масок Собеля, вычисление модуля градиента и сохранение результата в выходной массив.

Такой способ позволяет использовать только стандартные средства языка C++, не полагаясь на внешние зависимости. Однако ручное управление потоками и данными требует более тщательной настройки и может влиять на эффективность при работе с изображениями большого размера или при неравномерной нагрузке между потоками.


% --- 7. Описание MPI + TBB-версии алгоритма ---
\newpage
\section*{7. Описание MPI + TBB-версии алгоритма}
\addcontentsline{toc}{section}{7. Описание MPI + TBB-версии алгоритма}
\justifying

MPI + TBB-версия алгоритма представляет собой гибридный подход к параллельной обработке изображений, сочетающий преимущества распределённой (межпроцессной) и многопоточной (внутрипроцессной) параллельности. В данной реализации используется библиотека MPI для разделения изображения между несколькими процессами, а внутри каждого процесса применяется TBB (Intel Threading Building Blocks) для многопоточной обработки выделенной части изображения.

\textbf{Этапы выполнения алгоритма:}

\begin{itemize}
\item \textbf{Инициализация и валидация данных:} Процесс с рангом 0 получает исходные данные: размеры изображения и цветовые компоненты пикселей. Далее изображение преобразуется в оттенки серого.
\item \textbf{Разбиение изображения между процессами (MPI):} Изображение делится по горизонтали на фрагменты, каждый из которых обрабатывается отдельным процессом. Чтобы избежать ошибок на границах, каждому процессу передаются дополнительные строки: одна сверху, если это не первый процесс, и одна снизу, если это не последний.

\item \textbf{Формирование локальных участков:} Процесс с рангом 0 формирует для каждого участника подмассив (включая граничные строки) и отправляет его. Каждый процесс получает свой участок изображения и готовит буфер для хранения результатов.

\item \textbf{Применение оператора Собеля (TBB):} Внутри каждого процесса применяется параллельная обработка с использованием \texttt{tbb::parallel\_for} и одномерного диапазона \texttt{tbb::blocked\_range}. Каждому потоку назначается диапазон строк, в котором он вычисляет градиенты яркости по Собелю с помощью стандартных масок. Расчёт производится по пиксельной окрестности с сохранением результатов в локальный буфер.

\item \textbf{Сбор результатов:} После завершения локальной обработки каждый процесс отправляет результаты обратно процессу с рангом 0. Главный процесс собирает все локальные фрагменты в итоговое изображение, совмещая их в соответствии с координатами.

\end{itemize}

Данный подход позволяет эффективно масштабировать обработку изображений как на многопроцессорных системах (за счёт MPI), так и внутри каждого вычислительного узла (за счёт TBB). Это обеспечивает высокую производительность при обработке больших изображений, а также хорошую адаптацию под архитектуру кластера или многопроцессорной машины.

% --- 8. Результаты экспериментов ---
\newpage
\section*{8. Результаты экспериментов}
\addcontentsline{toc}{section}{8. Результаты экспериментов}
\justifying

Для оценки производительности различных реализаций оператора Собеля были проведены эксперименты. Замер проводился по двум метрикам:

\begin{itemize}
    \item \textbf{Pipeline} — общее время выполнения обработки, включая накладные расходы;
    \item \textbf{Task\_ran} — чистое время исполнения ядра алгоритма.
\end{itemize}

Метрика \texttt{Pipeline} отражает общую производительность алгоритма в условиях реальной загрузки, тогда как \texttt{Task\_ran} — потенциально достижимую производительность параллельной части. На их основе рассчитано ускорение (Speedup) по отношению к последовательной версии SEQ.

\begin{table}[h!]
\centering
\caption{Время выполнения и ускорение различных реализаций}
\begin{tabular}{|l|c|c|c|c|}
\hline
\textbf{Реализация} & \textbf{Pipeline (с)} & \textbf{Task\_ran (с)} & \textbf{Speedup P.} & \textbf{Speedup T.} \\
\hline
SEQ                 & 1.85 & 0.84 & 1.00 & 1.00 \\
OMP                 & 1.32 & \textbf{0.24} & 1.40 & \textbf{3.50} \\
TBB                 & 1.19 & 0.28 & 1.55 & 3.00 \\
STL                 & 1.27 & 0.30 & 1.46 & 2.80 \\
MPI+TBB (2×2)       & 1.47 & 0.45 & 1.26 & 1.87 \\
MPI+TBB (4×2)       & 1.51 & 0.45 & 1.22 & 1.87 \\
\hline
\end{tabular}
\end{table}




% --- 9. Выводы из результатов ---

\section*{9. Выводы из результатов}
\addcontentsline{toc}{section}{9. Выводы из результатов}
\justifying

На основе результатов экспериментального сравнения различных реализаций оператора Собеля можно сделать следующие выводы:

\begin{itemize}
    \item \textbf{SEQ (последовательная реализация):} служит базовой точкой отсчёта для оценки эффективности параллельных решений. Обеспечивает предсказуемое, но наименее производительное выполнение.

    \item \textbf{OpenMP:} показала наилучшее ускорение по метрике \texttt{Task\_ran} (до 3.5×), благодаря эффективному распараллеливанию циклов. Простота интеграции и хорошая масштабируемость делают OpenMP оптимальным выбором для задач с независимыми итерациями.

    \item \textbf{TBB:} обеспечила второе по величине ускорение, показав хорошую эффективность благодаря автоматическому управлению задачами и внутреннему балансировщику нагрузки. Подходит для более сложных сценариев, где необходима адаптивность к архитектуре системы.

    \item \textbf{STL (std::thread):} продемонстрировала сравнимую с TBB производительность, однако требует ручного управления потоками. Это усложняет реализацию и может привести к менее эффективному использованию ресурсов при неравномерной нагрузке.

    \item \textbf{MPI+TBB (2×2 и 4×2):} продемонстрировала умеренное ускорение, уступающее другим подходам. Хотя время \texttt{Task\_ran} было стабильным, общие накладные расходы в \texttt{Pipeline} были выше. Этот подход может раскрыть потенциал на распределённых системах или при более крупном масштабе задачи.

\end{itemize}

Таким образом, наилучший результат среди протестированных решений был достигнут с помощью \textbf{OpenMP}, благодаря простоте распараллеливания и эффективной работе с разделяемой памятью. Остальные подходы также показали прирост производительности, однако требуют более сложной настройки или ориентированы на специфические сценарии.

% --- Заключение ---
\newpage
\section*{Заключение}
\addcontentsline{toc}{section}{Заключение}
\justifying

В ходе выполнения данной работы была реализована параллельная версия оператора Собеля с использованием различных технологий многопоточного и многопроцессного программирования. Основной целью стало исследование возможностей распараллеливания задач обработки изображений и сравнение эффективности различных подходов.

В рамках проекта были изучены и применены следующие инструменты и библиотеки:
\begin{itemize}
    \item OpenMP — для распараллеливания циклов с помощью директив компилятора;
    \item Intel TBB — для реализации task-based параллелизма;
    \item std::thread — для ручного управления потоками средствами стандартной библиотеки C++;
    \item MPI в сочетании с TBB — для комбинации многопоточности и многопроцессности в гибридной архитектуре.
\end{itemize}

Каждая реализация была проанализирована с точки зрения производительности: измерялось общее время выполнения конвейера (\texttt{Pipeline}) и чистое время обработки изображения (\texttt{Task\_ran}). На основе этих данных были рассчитаны показатели ускорения (speedup) относительно последовательной версии.

В результате работы удалось:
\begin{itemize}
    \item ознакомиться с принципами параллельного программирования и различиями между потоками и задачами;
    \item на практике реализовать несколько параллельных моделей;
    \item провести сравнительный анализ производительности различных подходов;
    \item выявить сильные и слабые стороны каждого метода в контексте обработки изображений.
\end{itemize}

Таким образом, данная работа позволила углубить знания в области параллельного программирования, а также получить ценный опыт разработки, оптимизации и анализа вычислительных алгоритмов.


% --- Список литературы ---
\newpage
\section*{Список литературы}
\addcontentsline{toc}{section}{Список литературы}
\begin{enumerate}
    \item Сысоев А. В., Мееров И. Б., Сиднев А. А. Средства разработки параллельных программ для систем с общей памятью. Библиотека Intel Threading Building Blocks. — Нижний Новгород: Нижегородский государственный университет, 2007. — 122 с.

    \item Учебник по OpenMP [Электронный ресурс]. — Режим доступа: \url{https://pro-prof.com/archives/4335}, свободный. 

    \item Добро пожаловать в параллельный мир [Электронный ресурс] // Scrutator.me. — 2012. — Режим доступа: \url{https://scrutator.me/post/2012/04/04/parallel-world-p1.aspx}, свободный. 

    \item Львовский С. М. Набор и вёрстка в системе LaTeX: учеб. пособие. — 3-е изд., испр. и доп. — М.: Мир, 2003. — 472 с.
\end{enumerate}


% --- Приложение ---
\newpage
\section*{Приложение}
\addcontentsline{toc}{section}{Приложение}
\justifying

В приложении приведены фрагменты кода, реализующие алгоритм поиска оболочки методом Грэхема с использованием различных технологий параллельного программирования.

\subsection*{A.1 Последовательная реализация}
\addcontentsline{toc}{subsection}{A.1 Последовательная реализация}

\begin{lstlisting}[language=C++,caption={Sequential Version}]
bool frolova_e_sobel_filter_seq::SobelFilterSequential::RunImpl() {
  const std::vector<int> gx = {-1, 0, 1, -2, 0, 2, -1, 0, 1};
  const std::vector<int> gy = {-1, -2, -1, 0, 0, 0, 1, 2, 1};
  for (size_t y = 0; y < height_; y++) {
    for (size_t x = 0; x < width_; x++) {
      int res_x = 0;
      int res_y = 0;

      for (int ky = -1; ky <= 1; ky++) {
        for (int kx = -1; kx <= 1; kx++) {
          int px = static_cast<int>(x) + kx;
          int py = static_cast<int>(y) + ky;

          int pixel_value = 0;

          if (px >= 0 && px < static_cast<int>(width_) && py >= 0 && py < static_cast<int>(height_)) {
            pixel_value = grayscale_image_[(py * width_) + px];
          }

          size_t kernel_ind = ((ky + 1) * 3) + (kx + 1);
          res_x += pixel_value * gx[kernel_ind];
          res_y += pixel_value * gy[kernel_ind];
        }
      }
      int gradient = static_cast<int>(sqrt((res_x * res_x) + (res_y * res_y)));
      res_image_[(y * width_) + x] = Clamp(gradient, 0, 255);
    }
  }
  return true;
}
\end{lstlisting}

\subsection*{A.2 Реализация с использованием OpenMP}
\addcontentsline{toc}{subsection}{A.2 Реализация с использованием OpenMP}

\begin{lstlisting}[language=C++,caption={OpenMP Version}]
bool frolova_e_sobel_filter_omp::SobelFilterOmp::RunImpl() {

#pragma omp parallel for schedule(static)
  for (int y = 0; y < static_cast<int>(height_); y++) {
    for (int x = 0; x < static_cast<int>(width_); x++) {
      size_t base_idx = (y * width_) + x;

      int p00 = GetPixelSafe(grayscale_image, x - 1, y - 1, width_, height_);
      int p01 = GetPixelSafe(grayscale_image, x, y - 1, width_, height_);
      int p02 = GetPixelSafe(grayscale_image, x + 1, y - 1, width_, height_);
      int p10 = GetPixelSafe(grayscale_image, x - 1, y, width_, height_);
      int p11 = GetPixelSafe(grayscale_image, x, y, width_, height_);
      int p12 = GetPixelSafe(grayscale_image, x + 1, y, width_, height_);
      int p20 = GetPixelSafe(grayscale_image, x - 1, y + 1, width_, height_);
      int p21 = GetPixelSafe(grayscale_image, x, y + 1, width_, height_);
      int p22 = GetPixelSafe(grayscale_image, x + 1, y + 1, width_, height_);

      int res_x =
          (-1 * p00) + (0 * p01) + (1 * p02) + (-2 * p10) + (0 * p11) + (2 * p12) + (-1 * p20) + (0 * p21) + (1 * p22);

      int res_y =
          (-1 * p00) + (-2 * p01) + (-1 * p02) + (0 * p10) + (0 * p11) + (0 * p12) + (1 * p20) + (2 * p21) + (1 * p22);

      int gradient = static_cast<int>(sqrt((res_x * res_x) + (res_y * res_y)));
      res_image_[base_idx] = std::clamp(gradient, 0, 255);
    }
  }

  return true;
}
\end{lstlisting}

\subsection*{A.3 Реализация с использованием Intel TBB}
\addcontentsline{toc}{subsection}{A.3 Реализация с использованием Intel TBB}

\begin{lstlisting}[language=C++,caption={TBB Version}]
bool frolova_e_sobel_filter_tbb::SobelFilterTBB::RunImpl() {

  tbb::parallel_for(tbb::blocked_range2d<size_t>(0, height_, 0, width_), [&](const tbb::blocked_range2d<size_t>& r) {
    for (size_t y = r.rows().begin(); y < r.rows().end(); ++y) {
      for (size_t x = r.cols().begin(); x < r.cols().end(); ++x) {
        size_t base_idx = (y * width_) + x;

        int p00 = GetPixelSafe(grayscale_image, x - 1, y - 1, width_, height_);
        int p01 = GetPixelSafe(grayscale_image, x, y - 1, width_, height_);
        int p02 = GetPixelSafe(grayscale_image, x + 1, y - 1, width_, height_);
        int p10 = GetPixelSafe(grayscale_image, x - 1, y, width_, height_);
        int p11 = GetPixelSafe(grayscale_image, x, y, width_, height_);
        int p12 = GetPixelSafe(grayscale_image, x + 1, y, width_, height_);
        int p20 = GetPixelSafe(grayscale_image, x - 1, y + 1, width_, height_);
        int p21 = GetPixelSafe(grayscale_image, x, y + 1, width_, height_);
        int p22 = GetPixelSafe(grayscale_image, x + 1, y + 1, width_, height_);

        int res_x = (-1 * p00) + (0 * p01) + (1 * p02) + (-2 * p10) + (0 * p11) + (2 * p12) + (-1 * p20) + (0 * p21) +
                    (1 * p22);
        int res_y = (-1 * p00) + (-2 * p01) + (-1 * p02) + (0 * p10) + (0 * p11) + (0 * p12) + (1 * p20) + (2 * p21) +
                    (1 * p22);

        int gradient = static_cast<int>(sqrt((res_x * res_x) + (res_y * res_y)));
        res_image_[base_idx] = std::clamp(gradient, 0, 255);
      }
    }
  });

  return true;
}
\end{lstlisting}

\subsection*{A.4 Реализация с использованием std::thread}
\addcontentsline{toc}{subsection}{A.4 Реализация с использованием STL потоков}

\begin{lstlisting}[language=C++,caption={STL Thread Version}]
bool frolova_e_sobel_filter_stl::SobelFilterSTL::RunImpl() {
  int num_threads = ppc::util::GetPPCNumThreads();
  std::vector<std::thread> threads;

  auto sobel_task = [&](size_t y_start, size_t y_end) {
    for (size_t y = y_start; y < y_end; ++y) {
      for (size_t x = 0; x < width_; ++x) {
        size_t base_idx = (y * width_) + x;

        int p00 = GetPixelSafe(grayscale_image_, x - 1, y - 1, width_, height_);
        int p01 = GetPixelSafe(grayscale_image_, x, y - 1, width_, height_);
        int p02 = GetPixelSafe(grayscale_image_, x + 1, y - 1, width_, height_);
        int p10 = GetPixelSafe(grayscale_image_, x - 1, y, width_, height_);
        int p11 = GetPixelSafe(grayscale_image_, x, y, width_, height_);
        int p12 = GetPixelSafe(grayscale_image_, x + 1, y, width_, height_);
        int p20 = GetPixelSafe(grayscale_image_, x - 1, y + 1, width_, height_);
        int p21 = GetPixelSafe(grayscale_image_, x, y + 1, width_, height_);
        int p22 = GetPixelSafe(grayscale_image_, x + 1, y + 1, width_, height_);

        int res_x = (-1 * p00) + (0 * p01) + (1 * p02) + (-2 * p10) + (0 * p11) + (2 * p12) + (-1 * p20) + (0 * p21) +
                    (1 * p22);
        int res_y = (-1 * p00) + (-2 * p01) + (-1 * p02) + (0 * p10) + (0 * p11) + (0 * p12) + (1 * p20) + (2 * p21) +
                    (1 * p22);

        int gradient = static_cast<int>(sqrt((res_x * res_x) + (res_y * res_y)));
        res_image_[base_idx] = std::clamp(gradient, 0, 255);
      }
    }
  };

  size_t actual_threads_sobel = std::min(static_cast<size_t>(num_threads), height_);
  size_t rows_per_thread = (height_ + actual_threads_sobel - 1) / actual_threads_sobel;

  for (size_t i = 0; i < actual_threads_sobel; ++i) {
    size_t y_start = i * rows_per_thread;
    size_t y_end = std::min(y_start + rows_per_thread, height_);
    threads.emplace_back(sobel_task, y_start, y_end);
  }

  for (auto& thread : threads) {
    thread.join();
  }

  return true;
}
\end{lstlisting}

\subsection*{A.5 Реализация с использованием MPI и TBB}
\addcontentsline{toc}{subsection}{A.5 Реализация с использованием MPI и TBB}

\begin{lstlisting}[language=C++,caption={MPI + TBB Version}]
bool frolova_e_sobel_filter_all::SobelFilterALL::RunImpl() {
  int rank = world_.rank();
  int size = world_.size();

  broadcast(world_, height_, 0);
  broadcast(world_, width_, 0);

  int active_processes = std::min(size, static_cast<int>(height_));

  if (rank < active_processes) {
    int rows_per_proc = static_cast<int>(height_ / active_processes);
    int remainder = static_cast<int>(height_ % active_processes);

    InitWorkArea(active_processes, rows_per_proc, remainder, rank, y_start_, local_rows_, has_top_, has_bottom_,
                 extended_rows_);

    local_image_.resize(extended_rows_ * width_);
    std::vector<int> local_result(local_rows_ * width_);

    if (rank == 0) {
      std::vector<int> gray = grayscale_image_;
      for (int proc = 0; proc < active_processes; proc++) {
        int proc_y_start = 0;
        int proc_local_rows = 0;
        int top = 0;
        int bottom = 0;
        int ext_rows = 0;

        InitProcWorkArea(proc, active_processes, rows_per_proc, remainder, proc_y_start, proc_local_rows, top, bottom,
                         ext_rows);

        std::vector<int> chunk(ext_rows * width_, 0);
        for (int i = 0; i < ext_rows; ++i) {
          CopyOrZeroLine(chunk, gray, i, proc_y_start, top, static_cast<int>(width_), static_cast<int>(height_));
        }

        if (proc == 0) {
          local_image_ = chunk;
        } else {
          world_.send(proc, 0, chunk);
        }
      }
    } else {
      world_.recv(0, 0, local_image_);
    }

    frolova_e_sobel_filter_all::ApplySobelKernel(local_image_, local_result, static_cast<int>(width_), extended_rows_,
                                                 has_top_, local_rows_);

    if (world_.rank() == 0) {
      CollectWorkerResults(local_result, rows_per_proc, remainder, active_processes);
    } else {
      world_.send(0, 1, local_result);
    }
  }

  return true;
}
\end{lstlisting}



\end{document}
