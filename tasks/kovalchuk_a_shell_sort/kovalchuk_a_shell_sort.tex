\documentclass[12pt]{article}
\usepackage[russian]{babel}
\usepackage{graphicx}
\usepackage{geometry}
\usepackage{amsmath}
\usepackage{listings}
\usepackage{xcolor}
\usepackage{appendix}

\geometry{a4paper, left=25mm, right=15mm, top=20mm, bottom=20mm}

\definecolor{codegreen}{rgb}{0,0.6,0}
\definecolor{codegray}{rgb}{0.5,0.5,0.5}
\definecolor{codepurple}{rgb}{0.58,0,0.82}
\lstdefinestyle{mystyle}{
basicstyle=\ttfamily\footnotesize,
commentstyle=\color{codegreen},
keywordstyle=\color{magenta},
numberstyle=\tiny\color{codegray},
stringstyle=\color{codepurple},
breakatwhitespace=false,
breaklines=true,
keepspaces=true,
numbers=left,
numbersep=5pt,
showspaces=false,
showstringspaces=false,
showtabs=false,
tabsize=4
}
\lstset{style=mystyle}

\title{Отчет по практическим работам по курсу параллельного программирования на тему:\\[20pt]
«Параллельная реализация алгоритма сортировки Шелла с использованием технологий MPI и TBB»}
\author{Ковальчук А.Д., студент группы 3822Б1ПР4}
\date{2025}

\begin{document}
\sloppy

\begin{titlepage}
    \centering
    \textbf{Министерство образования и науки Российской Федерации\\[5pt]
    Федеральное государственное автономное образовательное\\
    учреждение высшего образования\\
    Национальный исследовательский Нижегородский государственный университет им. Н.И. Лобачевского\\[10pt]
    Институт информационных технологий, математики и механики}\\[30pt]

    \vfill

    \textbf{\Large Отчет по практическим работам по курсу параллельного программирования на тему:\\[20pt]
    «Параллельная реализация алгоритма сортировки Шелла с использованием технологий MPI и TBB»}\\[70pt]

    \hfill\parbox{0.35\textwidth}{
        \textbf{Выполнил:}\\
        студент группы 3822Б1ПР4,\\
        Ковальчук А.Д.\\[10pt]
        \textbf{Проверил:}\\
        доцент кафедры ВВиСП,\\
        к.т.н., Сысоев А.В.\\
    }

    \vfill

    Нижний Новгород\\
    2025
\end{titlepage}

\tableofcontents
\newpage

\section{Введение}
Сортировка Шелла — это алгоритм сортировки, который является обобщением сортировки вставками. Он позволяет сортировать элементы массива, расположенные далеко друг от друга, что уменьшает количество перестановок. В данной работе рассматривается параллельная реализация сортировки Шелла с использованием технологии MPI и TBB для ускорения вычислений. Основная цель работы — исследование эффективности параллельной реализации на различных типах данных и размерах массивов.

\section{Постановка задачи}
Требуется реализовать параллельную версию сортировки Шелла с простым слиянием, используя следующие технологии:
\begin{itemize}
\item MPI для распределения данных между процессами
\item TBB для параллельного выполнения сортировки внутри каждого процесса
\item Простое слияние для объединения отсортированных частей массива
\end{itemize}

Цели работы:
\begin{itemize}
\item Разработка алгоритма сортировки Шелла с простым слиянием
\item Реализация параллельной версии с использованием MPI и TBB
\item Проведение функционального тестирования на различных типах данных
\item Оценка производительности на массивах большого размера
\end{itemize}

\section{Структура данных}
Алгоритм работает с одномерным массивом целых чисел. Основные структуры данных:
\begin{itemize}
\item \texttt{std::vector<int> input\_}: исходный массив для сортировки (хранится только в процессе с рангом 0)
\item \texttt{std::vector<int> counts\_}: количество элементов для каждого процесса
\item \texttt{std::vector<int> result\_}: итоговый отсортированный массив (хранится только в процессе с рангом 0)
\end{itemize}

\section{Реализация}
\subsection{Общий алгоритм}
\begin{enumerate}
\item Распределение данных между процессами с помощью \texttt{MPI_Scatterv}
\item Параллельная сортировка Шелла внутри каждого процесса с использованием TBB
\item Сбор отсортированных частей с помощью \texttt{MPI_Gatherv}
\item Простое слияние отсортированных частей в процессе с рангом 0
\end{enumerate}

\subsection{Особенности реализации}
\subsubsection{MPI}
Использование коммуникаторов для разделения процессов:
\begin{lstlisting}[language=C++, caption=Фрагмент кода MPI]
boost::mpi::communicator world_, group_;
if (rank >= total_size) {
world_.split(1);
return true;
}
group_ = world_.split(0);
\end{lstlisting}

\subsubsection{TBB}
Параллельная сортировка Шелла с использованием \texttt{tbb::parallel_for}:
\begin{lstlisting}[language=C++, caption=Фрагмент кода TBB]
for (int gap = n / 2; gap > 0; gap /= 2) {
tbb::parallel_for(0, gap, [&](int k) {
for (int i = k + gap; i < n; i += gap) {
int temp = buffer[i];
int j = i;
while (j >= gap && buffer[j - gap] > temp) {
buffer[j] = buffer[j - gap];
j -= gap;
}
buffer[j] = temp;
}
});
}
\end{lstlisting}

\subsubsection{Простое слияние}
Использование приоритетной очереди для слияния отсортированных частей:
\begin{lstlisting}[language=C++, caption=Фрагмент кода простого слияния]
std::priority_queue<Element, std::vector<Element>, decltype(comp)> min_heap(comp);
for (int i = 0; i < num_procs; ++i) {
if (counts_[i] > 0) {
min_heap.emplace(gathered[displs[i]], i, 0);
}
}
\end{lstlisting}

\section{Тестирование}
\subsection{Функциональные тесты}
\begin{itemize}
\item Пустой массив: проверка обработки пустого ввода
\item Уже отсортированный массив: проверка сохранения порядка
\item Обратно отсортированный массив: проверка корректности сортировки
\item Массив с дубликатами: проверка устойчивости алгоритма
\item Массив с отрицательными числами: проверка работы с отрицательными значениями
\item Массив с экстремальными значениями: проверка обработки INT32_MIN и INT32_MAX
\item Массив со случайным порядком: проверка общего случая
\item Массив из одного элемента: проверка граничного условия
\end{itemize}

\subsection{Производительность}
Результаты тестирования на массиве из 1 000 000 элементов (среднее время 10 запусков):
\begin{table}[h]
\centering
\begin{tabular}{|l|c|}
\hline
\textbf{Версия} & \textbf{Время (сек)} \\ \hline
Последовательная & 2.45 \\ \hline
MPI (4 процесса) & 0.78 \\ \hline
MPI + TBB (4 процесса, 8 потоков) & 0.42 \\ \hline
\end{tabular}
\caption{Сравнение производительности}
\label{tab:performance}
\end{table}

\section{Выводы}
\begin{itemize}
\item Параллельная реализация сортировки Шелла с использованием MPI и TBB показывает значительное ускорение по сравнению с последовательной версией.
\item Гибридный подход (MPI + TBB) обеспечивает наилучшую производительность за счет комбинации распределенных и параллельных вычислений.
\item Алгоритм корректно обрабатывает различные типы данных, включая массивы с дубликатами и экстремальными значениями.
\item Простое слияние с использованием приоритетной очереди эффективно объединяет отсортированные части массива.
\end{itemize}

\newpage
\begin{appendices}
\section{Исходные коды}
\subsection{Основная реализация}
\begin{lstlisting}[language=C++, caption=Фрагмент кода из ops_all.cpp]
bool ShellSortAll::RunImpl() {
const int rank = world_.rank();
int total_size = rank == 0 ? static_cast<int>(input_.size()) : 0;
boost::mpi::broadcast(world_, total_size, 0);

if (rank >= total_size) {
    world_.split(1);
    return true;
}

group_ = world_.split(0);
int num_procs = group_.size();
counts_ = std::vector<int>(num_procs, total_size / num_procs);
std::vector<int> displs(num_procs, 0);
for (int i = 0; i < total_size % num_procs; ++i) {
    counts_[i]++;
}
for (int i = 1; i < num_procs; ++i) {
    displs[i] = displs[i - 1] + counts_[i - 1];
}

const int n = counts_[rank];
std::vector<int> buffer(n);
boost::mpi::scatterv(group_, input_.data(), counts_, displs, buffer.data(), n, 0);

for (int gap = n / 2; gap > 0; gap /= 2) {
    tbb::parallel_for(0, gap, [&](int k) {
        for (int i = k + gap; i < n; i += gap) {
            int temp = buffer[i];
            int j = i;
            while (j >= gap && buffer[j - gap] > temp) {
                buffer[j] = buffer[j - gap];
                j -= gap;
            }
            buffer[j] = temp;
        }
    });
}
input_ = buffer;

std::vector<int> gathered;
if (group_.rank() == 0) {
    gathered.resize(task_data->inputs_count[0]);
}

boost::mpi::gatherv(group_, input_, gathered.data(), counts_, 0);

if (group_.rank() == 0) {
    SimpleMerge(num_procs, gathered, displs, total_size);
}

return true;
}
\end{lstlisting}

\subsection{Функциональные тесты}
\begin{lstlisting}[language=C++, caption=Фрагмент кода из func_all.cpp]
TEST(kovalchuk_a_shell_sort_func, Test_ReverseSorted) {
std::vector<int> input = {10, 7, 5, 3, 1};
std::vector<int> output(input.size());
boost::mpi::communicator world;

auto task_data = std::make_shared<ppc::core::TaskData>();
if (world.rank() == 0) {
    task_data->inputs.emplace_back(reinterpret_cast<uint8_t*>(input.data()));
    task_data->inputs_count.emplace_back(input.size());
    task_data->outputs.emplace_back(reinterpret_cast<uint8_t*>(output.data()));
    task_data->outputs_count.emplace_back(output.size());
}

auto task = std::make_shared<kovalchuk_a_shell_sort_all::ShellSortAll>(task_data);
ASSERT_TRUE(task->Validation());
task->PreProcessing();
task->Run();
task->PostProcessing();

if (world.rank() == 0) {
    std::vector<int> expected = {1, 3, 5, 7, 10};
    EXPECT_EQ(expected, output);
}
}
\end{lstlisting}
\end{appendices}

\end{document}
