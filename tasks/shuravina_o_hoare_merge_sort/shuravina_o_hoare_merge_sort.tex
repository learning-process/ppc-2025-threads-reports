\documentclass[a4paper,12pt]{article}
\usepackage[utf8]{inputenc}
\usepackage[russian]{babel}
\usepackage{amsmath}
\usepackage{graphicx}
\usepackage{listings}
\usepackage{xcolor}
\usepackage{hyperref}

\title{Параллельная реализация сортировки Хоара с простым слиянием}
\author{Шуравина Оксана}
\date{\today}

\begin{document}

\maketitle

\section{Введение}
В данном отчете рассматривается реализация алгоритма быстрой сортировки (сортировки Хоара) с последующим простым слиянием. Алгоритм был реализован как в последовательной, так и в параллельной версиях с использованием технологии OpenMP.

\section{Постановка задачи}
Требовалось реализовать:
\begin{itemize}
    \item Последовательную версию алгоритма быстрой сортировки
    \item Простое слияние отсортированных частей
    \item Параллельную версию с использованием OpenMP
    \item Тестирование на различных типах входных данных
    \item Сравнение производительности последовательной и параллельной версий
\end{itemize}

\section{Алгоритм}
\subsection{Сортировка Хоара}
Основные этапы алгоритма:
\begin{enumerate}
    \item Выбор опорного элемента (pivot)
    \item Разделение массива на две части: элементы меньше pivot и больше pivot
    \item Рекурсивная сортировка обеих частей
\end{enumerate}

\subsection{Простое слияние}
После сортировки частей:
\begin{enumerate}
    \item Объединение двух отсортированных подмассивов
    \item Поэлементное сравнение и слияние в один массив
\end{enumerate}

\section{Реализация}
Ключевые аспекты реализации:
\begin{itemize}
    \item Рекурсивная реализация QuickSort
    \item Итеративное слияние отсортированных частей
    \item Использование std::vector для хранения данных
    \item Поддержка различных типов входных данных (случайные, отсортированные, обратно отсортированные)
\end{itemize}

\section{Тестирование}
Были реализованы следующие тесты:
\begin{itemize}
    \item Сортировка случайного массива (1000 элементов)
    \item Сортировка большого массива (10007 элементов)
    \item Сортировка и слияние фиксированного массива
    \item Обработка пустого массива
    \item Обработка массива из одного элемента
    \item Сортировка уже отсортированного массива
    \item Сортировка массива с отрицательными числами
\end{itemize}

\section{Производительность}
Результаты замеров времени выполнения:

\begin{tabular}{|c|c|c|}
\hline
Размер массива & Последовательная (мс) & Параллельная (мс) \\
\hline
10 000 & 12.4 & 4.2 \\
50 000 & 68.5 & 19.3 \\
100 000 & 145.2 & 42.1 \\
\hline
\end{tabular}

Ускорение составило примерно 3.4x на 4-ядерном процессоре.

\section{Выводы}
\begin{itemize}
    \item Алгоритм корректно работает на всех типах входных данных
    \item Параллельная реализация дает значительное ускорение
    \item Наибольшая эффективность достигается на больших массивах
    \item Простое слияние хорошо сочетается с параллельной сортировкой
\end{itemize}

\end{document}