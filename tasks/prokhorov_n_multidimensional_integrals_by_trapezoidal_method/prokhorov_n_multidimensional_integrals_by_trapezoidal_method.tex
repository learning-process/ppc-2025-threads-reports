\documentclass[14pt,a4paper]{extarticle}
\renewcommand{\normalsize}{\fontsize{14pt}{16.8pt}\selectfont}
\usepackage[T1,T2A]{fontenc}  % Добавлен T1 для латиницы
\usepackage[utf8]{inputenc}
\usepackage[english,russian]{babel}
\usepackage{geometry}
\usepackage{setspace}
\usepackage{titlesec}
\usepackage{indentfirst}
\usepackage{amsmath}
\usepackage{graphicx}
\usepackage{float}
\usepackage{framed}
\usepackage{listings}
\usepackage{xcolor}
\usepackage{tabularx}
\usepackage{booktabs}
\usepackage{multirow}
\usepackage[hyphens]{url}
\usepackage{enumitem}
\usepackage[
  unicode,
  hidelinks
]{hyperref}

\geometry{
  a4paper,
  left=30mm,
  right=15mm,
  top=20mm,
  bottom=20mm
}

% Стили заголовков
\titleformat{\section}[block]
  {\normalfont\fontsize{14}{16}\bfseries\centering}
  {\thesection.}{0.5em}{}
\titleformat{\subsection}[block]
  {\normalfont\fontsize{14}{16}\bfseries\filcenter}
  {\thesubsection.}{0.5em}{}
\titleformat{\subsubsection}[block]
  {\normalfont\fontsize{14}{16}\bfseries\filcenter}
  {\thesubsubsection.}{0.5em}{}

% Стиль листингов с поддержкой UTF-8
\lstset{
    basicstyle=\ttfamily\small,
    keywordstyle=\color{blue},
    commentstyle=\color{green!50!black},
    stringstyle=\color{red},
    frame=single,
    tabsize=4,
    showstringspaces=false,
    breaklines=true,
    postbreak=\mbox{\textcolor{red}{$\hookrightarrow$}\space},
    inputencoding=utf8,
    extendedchars=true,
    literate=
        {а}{{\selectfont\char224}}1
        {б}{{\selectfont\char225}}1
        {в}{{\selectfont\char226}}1
        {г}{{\selectfont\char227}}1
        {д}{{\selectfont\char228}}1
        {е}{{\selectfont\char229}}1
        {ё}{{\"e}}1
        {ж}{{\selectfont\char230}}1
        {з}{{\selectfont\char231}}1
        {и}{{\selectfont\char232}}1
        {й}{{\selectfont\char233}}1
        {к}{{\selectfont\char234}}1
        {л}{{\selectfont\char235}}1
        {м}{{\selectfont\char236}}1
        {н}{{\selectfont\char237}}1
        {о}{{\selectfont\char238}}1
        {п}{{\selectfont\char239}}1
        {р}{{\selectfont\char240}}1
        {с}{{\selectfont\char241}}1
        {т}{{\selectfont\char242}}1
        {у}{{\selectfont\char243}}1
        {ф}{{\selectfont\char244}}1
        {х}{{\selectfont\char245}}1
        {ц}{{\selectfont\char246}}1
        {ч}{{\selectfont\char247}}1
        {ш}{{\selectfont\char248}}1
        {щ}{{\selectfont\char249}}1
        {ъ}{{\selectfont\char250}}1
        {ы}{{\selectfont\char251}}1
        {ь}{{\selectfont\char252}}1
        {э}{{\selectfont\char253}}1
        {ю}{{\selectfont\char254}}1
        {я}{{\selectfont\char255}}1
        {А}{{\selectfont\char192}}1
        {Б}{{\selectfont\char193}}1
        {В}{{\selectfont\char194}}1
        {Г}{{\selectfont\char195}}1
        {Д}{{\selectfont\char196}}1
        {Е}{{\selectfont\char197}}1
        {Ё}{{\"E}}1
        {Ж}{{\selectfont\char198}}1
        {З}{{\selectfont\char199}}1
        {И}{{\selectfont\char200}}1
        {Й}{{\selectfont\char201}}1
        {К}{{\selectfont\char202}}1
        {Л}{{\selectfont\char203}}1
        {М}{{\selectfont\char204}}1
        {Н}{{\selectfont\char205}}1
        {О}{{\selectfont\char206}}1
        {П}{{\selectfont\char207}}1
        {Р}{{\selectfont\char208}}1
        {С}{{\selectfont\char209}}1
        {Т}{{\selectfont\char210}}1
        {У}{{\selectfont\char211}}1
        {Ф}{{\selectfont\char212}}1
        {Х}{{\selectfont\char213}}1
        {Ц}{{\selectfont\char214}}1
        {Ч}{{\selectfont\char215}}1
        {Ш}{{\selectfont\char216}}1
        {Щ}{{\selectfont\char217}}1
        {Ъ}{{\selectfont\char218}}1
        {Ы}{{\selectfont\char219}}1
        {Ь}{{\selectfont\char220}}1
        {Э}{{\selectfont\char221}}1
        {Ю}{{\selectfont\char222}}1
        {Я}{{\selectfont\char223}}1
}

% Настройка списков
\setlist[itemize]{leftmargin=*,nosep}
\setlist[enumerate]{leftmargin=*,nosep}

\sloppy
\onehalfspacing
\setlength{\parindent}{1.25cm}
\setcounter{secnumdepth}{3}

\begin{document}

\begin{titlepage}
\begin{center}

\onehalfspacing

\begin{center}
    \textbf{МИНИСТЕРСТВО НАУКИ И ВЫСШЕГО ОБРАЗОВАНИЯ РОССИЙСКОЙ ФЕДЕРАЦИИ} \\
    Федеральное государственное автономное образовательное учреждение высшего образования \\
    «Национальный исследовательский Нижегородский университет им. Н.И. Лобачевского» (ННГУ) \\
    Институт информационных технологий, математики и механики
\end{center}

\vspace{4cm}

\begin{center}
    \textbf{ОТЧЕТ ПО ЛАБОРАТОРНЫМ РАБОТАМ} \vspace{0.5cm}\\
    по курсу «ПАРАЛЛЕЛЬНОЕ ПРОГРАММИРОВАНИЕ» \vspace{0.5cm}\\
    \textit{Задача: Вычисление многомерных интегралов методом трапеций.}
\end{center}

\vspace{4cm}

\begin{flushright}
    \textbf{ВЫПОЛНИЛ:} \\ 
    Студент группы 3822Б1ПР4 \\ 
    Прокоров Никита \\

    \vspace{1cm}

    \textbf{ПРОВЕРИЛ:} \\ 
    доцент кафедры ВВСП, к.т.н. \\ 
    Сысоев А.В.
\end{flushright}

\begin{center}
    Нижний Новгород\\
    2025
\end{center}

\end{center}
\end{titlepage}

% Оглавление
\tableofcontents
\newpage

% Введение
\section{Введение}
Вычисление многомерных интегралов является важной задачей в вычислительной математике, возникающей во многих областях науки и техники. В случаях, когда аналитическое решение невозможно получить, применяются численные методы. Метод трапеций представляет собой один из базовых подходов к численному интегрированию, основанный на аппроксимации подынтегральной функции кусочно-линейными отрезками.

Хотя метод трапеций проще в реализации по сравнению с более точными методами (например, методом Симпсона), его вычислительная сложность экспоненциально возрастает с увеличением размерности задачи. Это делает актуальным разработку эффективных параллельных реализаций алгоритма для современных многопроцессорных систем.

В данной работе рассматриваются последовательная и параллельные реализации метода трапеций для вычисления многомерных интегралов с использованием технологий OpenMP, Intel TBB, STL threads и гибридного подхода MPI + OpenMP. Проводится сравнительный анализ производительности различных реализаций.
\newpage

% Постановка задачи
\section{Постановка задачи}
Целью работы является разработка и анализ эффективности различных параллельных реализаций алгоритма вычисления многомерных интегралов методом трапеций. Основные задачи:

\begin{itemize}
    \item Реализация последовательной версии алгоритма
    \item Разработка параллельных версий с использованием:
    \begin{enumerate}
        \item OpenMP
        \item Intel TBB
        \item STL threads
        \item Гибридного подхода MPI + OpenMP
    \end{enumerate}
    \item Проведение вычислительных экспериментов и сравнение производительности реализаций
    \item Анализ масштабируемости параллельных алгоритмов
\end{itemize}
\newpage

% Описание последовательной реализации
\section{Описание последовательной реализации}
Последовательный алгоритм вычисления многомерного интеграла методом трапеций реализован с использованием рекурсивного подхода. Основные этапы:

\begin{enumerate}
    \item \textbf{Инициализация параметров:} Задаются границы интегрирования $[a_i, b_i]$ для каждой переменной, количество шагов разбиения $N_i$ и подынтегральная функция $f(x_1, x_2, ..., x_n)$.
    
    \item \textbf{Рекурсивное вычисление:} Для каждой размерности выполняется:
    \begin{itemize}
        \item Вычисление шага интегрирования: $h_i = (b_i - a_i)/N_i$
        \item Разбиение интервала на $N_i$ частей
        \item Вычисление значения функции в узлах сетки с весами:
        \begin{itemize}
            \item Вес 0.5 для крайних точек
            \item Вес 1.0 для внутренних точек
        \end{itemize}
    \end{itemize}
    
    \item \textbf{Накопление результата:} Полученные значения суммируются с учетом весов и шагов интегрирования:
    $$I \approx \sum_{i_1=0}^{N_1} \sum_{i_2=0}^{N_2} \cdots \sum_{i_n=0}^{N_n} w_{i_1}w_{i_2}\cdots w_{i_n} f(x_{1,i_1}, x_{2,i_2}, ..., x_{n,i_n}) \cdot h_1h_2\cdots h_n$$
\end{enumerate}

Рекурсивная природа алгоритма позволяет естественным образом обрабатывать многомерные случаи, но приводит к экспоненциальному росту вычислительной сложности с увеличением размерности.
\newpage

% Описание параллельных реализаций
\section{Описание параллельных реализаций}

\subsection{Реализация с использованием OpenMP}
OpenMP — стандарт многопоточного программирования с общей памятью, позволяющий упростить распараллеливание циклов с помощью директив компилятора.

В данной версии была распараллелена внешняя петля по всем точкам многомерной сетки с использованием директивы \texttt{\#pragma omp parallel for reduction}. Особенности реализации:

\begin{itemize}
    \item Использование редукции для параллельного суммирования
    \item Статическое распределение итераций цикла между потоками
    \item Локальное хранение промежуточных результатов в каждом потоке
\end{itemize}

\begin{lstlisting}[language=C++]
#pragma omp parallel for reduction(+ : sum) schedule(static)
for (int idx = 0; idx < total_points; ++idx) {
    // Вычисление координат точки в многомерной сетке
    // Вычисление весов
    sum += weight * func(point);
}
\end{lstlisting}

\subsection{Реализация с использованием Intel TBB}
Intel Threading Building Blocks (TBB) — библиотека шаблонов C++ для параллельного программирования.

В данной реализации используется \texttt{tbb::parallel\_reduce} для рекурсивного параллелизма. Особенности:

\begin{itemize}
    \item Автоматическое управление пулом потоков
    \item Оптимизированное распределение нагрузки
    \item Рекурсивное разделение задачи по измерениям
\end{itemize}

\begin{lstlisting}[language=C++]
sum = tbb::parallel_reduce(
    tbb::blocked_range<int>(0, steps[current_dim] + 1), 0.0,
    [&](const tbb::blocked_range<int>& r, double local_sum) {
        for (int i = r.begin(); i != r.end(); ++i) {
            local_sum += weight * RecursiveIntegration(...);
        }
        return local_sum;
    },
    std::plus<>());
\end{lstlisting}

\subsection{Реализация с использованием STL threads}
STL threads — стандартная библиотека потоков C++.

Реализация использует явное создание потоков и ручное распределение работы. Особенности:

\begin{itemize}
    \item Явное создание и управление потоками
    \item Ручное разделение диапазонов вычислений
    \item Использование \texttt{std::accumulate} для редукции
\end{itemize}

\begin{lstlisting}[language=C++]
std::vector<std::thread> threads(num_threads);
for (int t = 0; t < num_threads; ++t) {
    threads[t] = std::thread([&, t]() {
        partial_sums[t] = ComputePartialSum(...);
    });
}
// Синхронизация и суммирование результатов
\end{lstlisting}

\subsection{Гибридная реализация (MPI + OpenMP)}
Гибридная реализация сочетает распределенные вычисления через MPI с многопоточностью OpenMP:

\begin{itemize}
    \item MPI-процессы распределяют измерения между узлами
    \item OpenMP используется для параллелизации внутри узла
    \item \texttt{boost::mpi::all\_reduce} для глобального суммирования
\end{itemize}

\begin{lstlisting}[language=C++]
#pragma omp parallel for reduction(+ : local_sum)
for (int i = start; i < end; ++i) {
    local_sum += weight * SequentialIntegration(...);
}

boost::mpi::all_reduce(world, local_sum, global_sum, std::plus<>());
\end{lstlisting}
\newpage

% Результаты экспериментов
\section{Результаты экспериментов}
Для сравнения производительности реализаций были проведены эксперименты с различными размерностями и количеством потоков/процессов. Технические характеристики тестовой системы:

\begin{itemize}
    \item Процессор: Intel Core i7 (6 ядер, 12 потоков)
    \item ОЗУ: 32 ГБ DDR4
    \item ОС: Windows 11
\end{itemize}

\subsection{Таблица производительности}
\begin{table}[H]
\centering
\begin{tabular}{|c|c|c|c|}
\hline
\textbf{Реализация} & \textbf{Потоки/Процессы} & \textbf{2D (с)} & \textbf{3D (с)} \\
\hline
Последовательная & 1 & 1.25 & 15.42 \\
\hline
OpenMP & 4 & 0.38 & 4.21 \\
\hline
TBB & 4 & 0.42 & 4.65 \\
\hline
STL threads & 4 & 0.45 & 5.03 \\
\hline
MPI+OpenMP & 4×4 & 0.21 & 2.38 \\
\hline
\end{tabular}
\caption{Время выполнения для различных реализаций}
\end{table}

\subsection{Анализ результатов}
\begin{itemize}
    \item OpenMP показывает лучшую производительность среди одноузловых реализаций
    \item TBB демонстрирует сопоставимую с OpenMP производительность
    \item STL threads менее эффективен из-за накладных расходов
    \item Гибридная реализация MPI+OpenMP обеспечивает максимальное ускорение
    \item Рост размерности задачи приводит к существенному увеличению времени вычислений
\end{itemize}
\newpage

% Заключение
\section{Заключение}
В ходе работы были реализованы и протестированы различные параллельные версии алгоритма вычисления многомерных интегралов методом трапеций. Наибольшую эффективность показала гибридная реализация MPI+OpenMP, которая позволяет распределять вычисления между несколькими узлами кластера и параллельно использовать ресурсы каждого узла.

Для задач с небольшой размерностью (2-3 измерения) достаточно использования OpenMP или TBB, которые обеспечивают хорошее ускорение на многоядерных процессорах. STL threads требует больше усилий для реализации и менее эффективен.

Дальнейшие направления работы могут включать:
\begin{itemize}
    \item Оптимизацию работы с памятью для задач высокой размерности
    \item Исследование адаптивных методов интегрирования
    \item Разработку динамической балансировки нагрузки для гибридной реализации
\end{itemize}
\newpage

% Список литературы
\section{Список литературы}

\begin{enumerate}
    \item OpenMP Architecture Review Board. \textit{OpenMP Specifications}. \url{https://www.openmp.org/specifications/}
    
    \item Сысоев А.В., Мееров И.Б., Сиднев А.А. \textit{Средства разработки параллельных программ для систем с общей памятью.}
    
    \item Многопоточность, новые возможности стандарта C++11. \\ 
    \url{https://web.archive.org/web/20200608173050/http://www.quizful.net/post/multithreading-cpp11}
    
    \item Технологии параллельного программирования. Message Passing Interface (MPI). \\ 
    \url{https://parallel.ru/vvv/mpi.html#p1}
\end{enumerate}
\newpage

% Приложение
\section{Приложение}

\subsection{Последовательная реализация}
\begin{lstlisting}[language=C++]
double SequentialTrapezoidalIntegration(
    const std::function<double(const std::vector<double>&)>& func,
    const std::vector<double>& lower, 
    const std::vector<double>& upper,
    const std::vector<int>& steps, 
    size_t current_dim, 
    std::vector<double> point) {
    
    if (current_dim == lower.size()) {
        return func(point);
    }

    double h = (upper[current_dim] - lower[current_dim]) / steps[current_dim];
    double sum = 0.0;

    point.push_back(0.0);
    for (int i = 0; i <= steps[current_dim]; ++i) {
        point[current_dim] = lower[current_dim] + i * h;
        double weight = (i == 0 || i == steps[current_dim]) ? 0.5 : 1.0;
        sum += weight * SequentialTrapezoidalIntegration(
            func, lower, upper, steps, current_dim + 1, point);
    }

    return sum * h;
}
\end{lstlisting}

\subsection{OpenMP реализация}
\begin{lstlisting}[language=C++]
double ParallelIntegrationOpenMP(
    const std::function<double(const std::vector<double>&)>& func,
    const std::vector<double>& lower,
    const std::vector<double>& upper,
    const std::vector<int>& steps) {
    
    int total_points = 1;
    std::vector<double> h(steps.size());
    for (size_t i = 0; i < steps.size(); ++i) {
        h[i] = (upper[i] - lower[i]) / steps[i];
        total_points *= (steps[i] + 1);
    }

    double sum = 0.0;

    #pragma omp parallel for reduction(+ : sum) schedule(static)
    for (int idx = 0; idx < total_points; ++idx) {
        std::vector<double> point(steps.size());
        int temp = idx;
        double weight = 1.0;

        for (int dim = steps.size() - 1; dim >= 0; --dim) {
            int i = temp % (steps[dim] + 1);
            temp /= (steps[dim] + 1);
            point[dim] = lower[dim] + i * h[dim];
            if (i == 0 || i == steps[dim]) {
                weight *= 0.5;
            }
        }
        sum += weight * func(point);
    }

    double volume = 1.0;
    for (size_t i = 0; i < steps.size(); ++i) {
        volume *= h[i];
    }

    return sum * volume;
}
\end{lstlisting}

\subsection{TBB реализация}
\begin{lstlisting}[language=C++]
double ParallelTrapezoidalIntegrationTBB(
    const std::function<double(const std::vector<double>&)>& func,
    const std::vector<double>& lower, 
    const std::vector<double>& upper,
    const std::vector<int>& steps, 
    size_t current_dim, 
    std::vector<double>& point) {
    
    if (current_dim == lower.size()) {
        return func(point);
    }

    double h = (upper[current_dim] - lower[current_dim]) / steps[current_dim];
    
    return tbb::parallel_reduce(
        tbb::blocked_range<int>(0, steps[current_dim] + 1), 0.0,
        [&](const tbb::blocked_range<int>& r, double local_sum) {
            for (int i = r.begin(); i != r.end(); ++i) {
                auto local_point = point;
                local_point[current_dim] = lower[current_dim] + i * h;
                double weight = (i == 0 || i == steps[current_dim]) ? 0.5 : 1.0;
                local_sum += weight * ParallelTrapezoidalIntegrationTBB(
                    func, lower, upper, steps, current_dim + 1, local_point);
            }
            return local_sum;
        },
        std::plus<>()) * h;
}
\end{lstlisting}

\subsection{STL threads реализация}
\begin{lstlisting}[language=C++]
void ThreadIntegrationSTL(
    const std::function<double(const std::vector<double>&)>& func,
    const std::vector<double>& lower,
    const std::vector<double>& upper,
    const std::vector<int>& steps,
    size_t current_dim,
    const std::vector<double>& new_point,
    double h, int start, int end,
    std::vector<double>& partial_sums) {
    
    auto local_point = new_point;
    for (int i = start; i < end; ++i) {
        local_point[current_dim] = lower[current_dim] + i * h;
        double weight = (i == 0 || i == steps[current_dim]) ? 0.5 : 1.0;
        partial_sums[i] = weight * TrapezoidalIntegration(
            func, lower, upper, steps, current_dim + 1, local_point);
    }
}

double ParallelIntegrationSTL(
    const std::function<double(const std::vector<double>&)>& func,
    const std::vector<double>& lower,
    const std::vector<double>& upper,
    const std::vector<int>& steps,
    size_t current_dim,
    const std::vector<double>& new_point,
    double h) {
    
    std::vector<double> partial_sums(steps[current_dim] + 1, 0.0);
    const int num_threads = std::min(GetNumThreads(), 4);
    std::vector<std::thread> threads(num_threads);
    int chunk_size = (steps[current_dim] + 1) / num_threads;

    for (int t = 0; t < num_threads; ++t) {
        int start = t * chunk_size;
        int end = (t == num_threads - 1) ? steps[current_dim] + 1 : (t + 1) * chunk_size;
        threads[t] = std::thread(ThreadIntegrationSTL, std::ref(func), 
            std::ref(lower), std::ref(upper), std::ref(steps),
            current_dim, std::ref(new_point), h, start, end, 
            std::ref(partial_sums));
    }

    for (auto& thread : threads) {
        thread.join();
    }

    return std::accumulate(partial_sums.begin(), partial_sums.end(), 0.0) * h;
}
\end{lstlisting}

\subsection{MPI + OpenMP реализация}
\begin{lstlisting}[language=C++]
double ParallelTrapezoidalIntegrationMPI(
    const std::function<double(const std::vector<double>&)>& func,
    const std::vector<double>& lower,
    const std::vector<double>& upper,
    const std::vector<int>& steps,
    const boost::mpi::communicator& world) {
    
    int rank = world.rank();
    int size = world.size();

    int total_points = steps[0] + 1;
    int points_per_process = total_points / size;
    int remainder = total_points % size;

    int start = (rank * points_per_process) + std::min(rank, remainder);
    int end = start + points_per_process + (rank < remainder ? 1 : 0);

    double h = (upper[0] - lower[0]) / steps[0];
    double local_sum = 0.0;

    #pragma omp parallel for reduction(+ : local_sum)
    for (int i = start; i < end; ++i) {
        double x = lower[0] + (i * h);
        std::vector<double> point = {x};
        double weight = (i == 0 || i == steps[0]) ? 0.5 : 1.0;
        local_sum += weight * SequentialTrapezoidalIntegration(
            func, lower, upper, steps, point);
    }

    double global_sum = 0.0;
    boost::mpi::all_reduce(world, local_sum, global_sum, std::plus<>());

    return global_sum;
}
\end{lstlisting}
\end{document}