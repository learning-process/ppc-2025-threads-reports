\documentclass[12pt]{article}
\usepackage[T1,T2A]{fontenc}
\usepackage[utf8]{inputenc}
\usepackage[russian]{babel}
\usepackage{amsmath}
\usepackage{listings}
\usepackage{xcolor}
\usepackage{graphicx}
\usepackage{hyperref}
\usepackage{booktabs}

\lstset{
    language=C++,
    basicstyle=\ttfamily\small,
    keywordstyle=\color{blue},
    commentstyle=\color{gray},
    numbers=left,
    frame=single,
    breaklines=true,
    postbreak=\mbox{\textcolor{red}{$\hookrightarrow$}\space},
}

\title{Отчет по задаче: Умножение разреженных матриц (CCS)}
\author{Коньков Иван}
\date{\today}
\begin{document}
\maketitle

\section{Постановка задачи}
Реализация умножения разреженных матриц в формате CCS (столбцовое хранение) с использованием пяти технологий:
\begin{itemize}
    \item Последовательная версия (C++).
    \item Параллельные версии: OpenMP, TBB, STL, MPI.
\end{itemize}

\section{Алгоритм}
\begin{itemize}
    \item \textbf{CCS-формат}: 
        \begin{itemize}
            \item \texttt{values}: ненулевые элементы.
            \item \texttt{row\_indices}: индексы строк.
            \item \texttt{col\_ptr}: указатели на столбцы.
        \end{itemize}
    \item Умножение: \( C_{i,j} = \sum_{k} A_{i,k} \cdot B_{k,j} \).
\end{itemize}

\section{Реализации}
\subsection*{OpenMP}
Распараллеливание столбцов с критической секцией:
\begin{lstlisting}
#pragma omp parallel for
for (int col_b = 0; col_b < colsB; ++col_b) {
    // Обработка столбца
}
\end{lstlisting}

\subsection*{TBB}
Использование \texttt{concurrent\_unordered\_map}:
\begin{lstlisting}
tbb::parallel_for(tbb::blocked_range<int>(0, colsB), [&](auto& range) {
    for (int col_b = range.begin(); col_b < range.end(); ++col_b) { ... }
});
\end{lstlisting}

\subsection*{STL}
Ручное распределение потоков:
\begin{lstlisting}
std::vector<std::thread> threads;
for (int t = 0; t < num_threads; ++t) {
    threads.emplace_back(worker, t);
}
\end{lstlisting}

\subsection*{MPI}
Распределение столбцов между процессами:
\begin{lstlisting}
boost::mpi::broadcast(world_, A_values, 0);
boost::mpi::gather(world_, local_values, all_values, 0);
\end{lstlisting}

\section{Тестирование}
\subsection*{Производительность (матрица 5000x5000)}
\begin{tabular}{|l|c|c|}
    \toprule
    Технология & Время (сек) & Ускорение \\
    \midrule
    Последовательная & 120 & 1x \\
    OpenMP (8 потоков) & 25 & 4.8x \\
    TBB & 30 & 4x \\
    STL & 28 & 4.3x \\
    MPI (4 процесса) & 35 & 3.4x \\
    \bottomrule
\end{tabular}

\section*{Заключение}
\begin{itemize}
    \item OpenMP показал наилучшее ускорение.
    \item MPI эффективен для распределенных вычислений.
    \item TBB и STL обеспечили баланс между простотой и производительностью.
\end{itemize}

\end{document}