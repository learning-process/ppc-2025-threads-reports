\documentclass[12pt]{article}

\usepackage[utf8]{inputenc}
\usepackage[T2A]{fontenc}
\usepackage[russian]{babel}

\usepackage{lingmacros}
\usepackage{tree-dvips}
\usepackage{booktabs}
\usepackage{multirow}
\usepackage{geometry}
\geometry{
	a4paper,
	left=30mm,
	right=15mm,
	top=20mm,
	bottom=20mm
}

\usepackage{titlesec}
\titleformat{\section}[block]
{\normalfont\fontsize{14}{16}\bfseries\centering}
{\thesection.}{0.5em}{}
\titleformat{\subsection}[block]
{\normalfont\fontsize{14}{16}\bfseries\filcenter}
{\thesubsection.}{0.5em}{}
\titleformat{\subsubsection}[block]
{\normalfont\fontsize{14}{16}\bfseries\filcenter}
{\thesubsubsection.}{0.5em}{}

\usepackage{listings}
\usepackage{xcolor}
\lstset{
	basicstyle=\ttfamily\small,
	frame=single,
	tabsize=4,
	showstringspaces=false,
	breaklines=true
}

\sloppy
\usepackage{setspace}
\onehalfspacing
\setlength{\parindent}{1.25cm}

\newcommand{\appendixsection}[1]{%
	\clearpage
	\section*{\centering Приложение #1}
	\addcontentsline{toc}{section}{Приложение #1}
}

\begin{document}

\begin{titlepage}
	\begin{center}
		\onehalfspacing
		\textbf{МИНИСТЕРСТВО НАУКИ И ВЫСШЕГО ОБРАЗОВАНИЯ РОССИЙСКОЙ ФЕДЕРАЦИИ} \\ 	[5pt]	
		Федеральное государственное автономное образовательное учреждение высшего образования \\  [5pt]	
		\textbf{«Национальный исследовательский Нижегородский государственный университет имени Н.И. Лобачевского(ННГУ)»} \\[5pt]

		\textbf{Институт информационных технологий, математики и механики}\\ 	[30pt]	

		\begin{center}
		Направление подготовки: Фундаментальная информатика и информационные технологии

		Профиль подготовки: «Инженерия программного обеспечения» \\ 	[50pt]	
		\end{center}

		\textbf{Отчёт}\\[0.5pt]
{ по лабораторной работе}

		на тему: 

			\textbf{"Линейная фильтрация изображений (вертикальное разбиение). Ядро Гаусса 3x3."}\\[60pt]


		\begin{flushright}
			\textbf{Выполнила:} \\
			студентка группы 3822Б1ФИ1 \\
			Комшина Д.Г. \\[20pt]


			\noindent\textbf{Преподаватель:} \\
			к.т.н., доцент кафедры ВВСП \\
			{Сысоев А.В.}
		\end{flushright}

		\vfill

		Нижний Новгород \\
		2025 г.
	\end{center}
\end{titlepage}

\newpage

% --- Основной текст документа ---

\newpage

\section*{Введение}

Гауссов фильтр – основной метод фильтрации изображений, используемый для подавления шума с сохранением деталей, что делает его незаменимым в компьютерном зрении.
В данной работе рассматривается линейная фильтрация изображений с применением двумерного Гауссова ядра размером 3×3, при этом используется вертикальное разбиение изображения, что обеспечивает возможность адаптации алгоритма к последовательной обработке и параллельной реализации с использованием технологий OpenMP, Intel Threading Building Blocks (TBB), стандартной библиотеки потоков C++ (STL) и гибридной модели MPI + TBB.

\section*{Постановка задачи}

Целью данной лабораторной работы является реализация и исследование эффективности линейной фильтрации цветного изображения с применением Гауссова ядра размером 3×3. Под фильтрацией понимается свёртка каждого канала изображения (RGB) с фиксированным ядром, обеспечивающим сглаживание и подавление шумов.

\section*{Задачи}

\begin{enumerate}
  \item Реализовать последовательную версию фильтрации изображения, используя трёхканальную структуру (RGB).
  
  \item Разработать и протестировать четыре параллельные версии алгоритма с использованием различных технологий:
  \begin{enumerate}
    \item OpenMP — многопоточность на уровне циклов;
    \item TBB — библиотека Intel Threading Building Blocks;
    \item STL — стандартные средства параллельного программирования;
    \item MPI+OMP — гибридная модель с распараллеливанием между процессами и потоками.
  \end{enumerate}
  
  \item Осуществить валидацию корректности выходных данных путём сравнения с эталонным результатом (последовательной реализацией).
  
  \item Оценить производительность и эффективность параллельных реализаций относительно последовательной, измерив время выполнения различных версий алгоритма.
\end{enumerate}

\newpage

\section*{Описание алгоритма Линейная фильтрация изображений (вертикальное разбиение). Ядро Гаусса 3x3}

Метод работает следующим образом: каждый пиксель изображения пересчитывается заново с учётом значений соседних пикселей. Вместо того чтобы просто использовать значение самого пикселя, алгоритм берёт область 3x3 вокруг него и использует специальную таблицу весов — ядро Гаусса. Эта таблица определяет, какое влияние будет иметь каждый соседний пиксель на итоговое значение.

Так как фильтр называется «вертикальным», изображение делится на вертикальные участки, и обработка проводится построчно сверху вниз. Это полезно, особенно если алгоритм выполняется параллельно — например, если его нужно ускорить с помощью нескольких ядер процессора или разных потоков. Каждая вертикальная часть может обрабатываться отдельно.


\section{Описание реализаций}
\subsection{Последовательная версия}

Последовательная версия фильтрации изображения с использованием вертикального гауссового ядра реализует свёртку цветного изображения формата RGB с ядром размером 3×3. Сначала происходит подготовка данных: копируются входные пиксели и ядро фильтра в внутренние векторы, а также выделяется память под выходной результат. Затем выполняется проверка корректности входных и выходных данных — проверяется, что указатели не равны nullptr, а размеры соответствуют ожиданиям (фиксированный размер ядра 9 элементов и совпадающий размер выходного массива). Основной алгоритм свёртки проходит по всем пикселям изображения, кроме граничных, и для каждого цвета (красного, зелёного и синего) вычисляет сумму произведений значений соседних пикселей и соответствующих коэффициентов ядра. Полученное значение округляется и ограничивается диапазоном от 0 до 255, чтобы соответствовать формату цвета. Итоговые данные записываются в выходной буфер. В конце результаты копируются обратно в выходной массив.


\subsection{OpenMP версия}

Как и в последовательной версии сначала в методе PreProcessingImpl происходит копирование входных данных изображения (RGB) и ядра фильтра во внутренние векторы, а также выделяется память под выходной результат. Метод ValidationImpl проверяет корректность входных и выходных данных, убеждаясь, что ядро и размеры изображения соответствуют требованиям.

В RunImpl лежит основное отличие от последовательной реализации. Используется директива \#pragma omp parallel для создания параллельного региона, в котором переменные input, output (ссылки на векторы входных и выходных данных), а также размеры изображения и ядро передаются в каждый поток.

Затем с помощью директивы \#pragma omp for цикл по строкам изображения (переменная y) распараллеливается между потоками. Каждый поток обрабатывает отдельный набор строк, что позволяет одновременно вычислять свёртку для множества пикселей.

Внутри параллельного цикла для каждого пикселя (кроме граничных) по всем трем цветовым каналам вычисляется свёртка 3×3 области вокруг текущего пикселя. Для этого используется вложенный цикл по смещениям ky и kx с соответствующим индексом ядра k. Результат свёртки округляется, ограничивается значениями от 0 до 255 и записывается в выходной буфер.

После завершения параллельной обработки содержимое вектора output копируется обратно в выходной буфер, указанный в taskdata.


\subsection{TBB версия}

Реализация с использованием  TBB следует аналогичному подходу, но использует параллельные конструкции TBB.
Параллелизм с использованием task arena и task group позволяет запускать задачи параллельно с управлением числом потоков, что эффективно использует ресурсы процессора.

 В RunImpl создаётся объект task arena, который ограничивает число потоков в пуле TBB числом, возвращаемым функцией ppc::util::GetPPCNumThreads(). Это позволяет контролировать уровень параллелизма.

Внутри арены запускается task group, куда добавляются задачи — для каждого потока создаётся задача, выполняющая функцию GaussianVerticalFilter. Однако стоит отметить, что в текущей реализации каждый поток выполняет фильтрацию по всему изображению, что приводит к избыточным вычислениям и перезаписи результатов.

Функция GaussianVerticalFilter реализует стандартную свёртку по вертикальному и горизонтальному соседству пикселей с ядром размером 3×3, проходя по всем пикселям, кроме границ, и вычисляя новый цвет для каждого из трех каналов. Результат записывается в выходной буфер.

После выполнения задач содержимое output копируется обратно в буфер вывода.


\subsection{STL версия}

Данная STL-версия успешно использует возможности стандартной библиотеки для многопоточной обработки изображения, обеспечивая разделение работы между потоками по строкам. Это даёт хороший уровень параллелизма без сложных зависимостей от сторонних библиотек и эффективно использует ресурсы многоядерных процессоров. 

Рассмотрим RunImpl — распараллеливание вычислений с помощью std::thread
Получается количество потоков, доступных системе (num threads). Создаётся вектор потоков размером num threads.

Каждый поток запускает функцию GaussianVerticalFilter, которая выполняет свёртку по своему "шахматному" шагу по строкам — поток с thread id = i обрабатывает строки начиная с y = 1 + i и далее с шагом равным числу потоков (num threads). Такое разбиение гарантирует, что потоки не пересекаются по обрабатываемым строкам и не конфликтуют при записи в выходной буфер.

Внутри функции GaussianVerticalFilter для каждого пикселя (кроме краевых) вычисляется новая цветовая компонента с помощью суммирования произведений соседних пикселей и значений ядра свёртки, а результат округляется и ограничивается по диапазону 0–255.

После запуска всех потоков вызывается join() для каждого из них, чтобы дождаться завершения обработки.


\subsection{Версия \texttt{MPI+TBB}}

Реализация использует гибридный подход: MPI — для распределения изображения между несколькими процессами, и OpenMP — для распараллеливания обработки внутри каждого процесса. Это позволяет эффективно использовать ресурсы как многопроцессорных, так и многопоточных систем.

Распределение данных (MPI):Каждый процесс получает свою часть изображения — определённое количество строк. Чтобы фильтр мог корректно работать на границах блока, каждому процессу дополнительно передаются "halo" строки сверху и снизу (если они есть). Эти строки не фильтруются, но используются при расчётах.

Рассылка и приём данных:Процесс с rank == 0 занимается разбиением общего изображения и отправляет соответствующие блоки другим процессам. Каждый блок включает нужные строки изображения и halo-области. Процессы с rank > 0 получают свои блоки через world.recv(...).

Применение фильтра (OpenMP):После получения блока каждый процесс применяет вертикальный гауссов фильтр к своим строкам (исключая halo). Для ускорения используется OpenMP: строки обрабатываются в параллельных потоках. Каждый пиксель фильтруется по трём каналам (RGB), и значения ограничиваются диапазоном [0, 255].

Сбор результатов:После фильтрации процессы с rank > 0 отправляют результаты обратно процессу 0. Он собирает все локальные выходные блоки в общий вектор output, вставляя их в нужные позиции. Таким образом восстанавливается итоговое обработанное изображение.

\section{Экспериментальные результаты}
Для оценки производительности были проведены эксперименты на наборе данных из 4000 случайных чисел с плавающей точкой двойной точности. Будем сравнивать последовательную реализацию с параллельными версиями OpenMP, TBB, STL и MPI+TBB.

\section*{Результаты экспериментов}
Для оценки производительности реализованных версий алгоритма были проведены измерения времени выполнения программы на входном массиве из 4000×4000 пикселей. Будем сравнивать последовательную реализацию с параллельными версиями OpenMP, TBB, STL и MPI+TBB.

\begin{center}
\begin{tabular}{|c|c|c|c|c|}
\hline
\textbf{Версия} & \textbf{Конфигурация} & \textbf{PipelineRun (с)} & \textbf{TaskRun (с)} \\
\hline
SEQ & — & 4.0127 & 3.9395\\
\hline
\multirow{3}{*}{OpenMP} 
 & 2 потока & 2.2046 & 2.2197\\
 & 3 потока & 1.7035 & 1.4545\\
 & 5 потоков & 0.5059 & 0.4308\\
\hline
\multirow{3}{*}{TBB} 
 & 2 потока & 0.2328 & 0.1712\\
 & 3 потока & 0.092741 & 0.0000147\\
 & 5 потоков & 1.6284 & 1.5605\\
\hline
\multirow{3}{*}{STL} 
 & 2 потока & 0.0456248 & 0.0000142\\
 & 3 потока & 0.092741 & 0.0000147\\
 & 5 потоков & 0.087833 & 0.0000218\\
\hline
\multirow{5}{*}{MPI + OMP} 
 & 2 потока + 2 процесса & 2.2578 & 2.1435\\
 & 3 потока + 2 процесса & 2.1793 & 1.9564\\
 & 3 потока + 3 процесса & 2.2651 & 2.1242 \\
\hline
\end{tabular}
\end{center}

\section*{Анализ результатов}

SEQ (последовательное выполнение) - это базовый уровень, с которым сравниваем параллельные реализации.Не использует параллелизм, самая медленная.
OpenMP является наиболее сбалансированным решением с точки зрения производительности и масштабируемости для данной задачи.Значительно быстрее, чем SEQ, с хорошей масштабируемостью.Чётко показывает выгоду от распараллеливания, время уменьшается при добавлении потоков.
TBB : при 2-3 потоках показывает лучшую производительность, чем OpenMP. Но при 5 потоках  ухудшается.
STL Threads  : при 2 потоках время минимальное (0.045 с), даже лучше TBB и OpenMP.Но при 3 и 5 потоках время не улучшается, а иногда немного ухудшается.
MPI + OpenMP (гибрид): время сопоставимо с OpenMP, 

Корректность работы всех реализаций была подтверждена путем функционального тестирования с использованием фреймворка GTest. Результаты сортировки для всех параллельных версий сравнивались с результатами последовательной реализации (SEQ)

\section*{Заключение}

В ходе лабораторной работы была реализована и исследована эффективность линейной фильтрации изображений с применением ядра Гаусса 3×3 с вертикальным разбиением. Были разработаны последовательная версия и четыре параллельные реализации с использованием технологий OpenMP, TBB, STL и гибридной модели MPI + OpenMP.

Экспериментальные результаты показали значительное ускорение параллельных версий по сравнению с последовательной реализацией. Наилучшие показатели по времени демонстрировали реализации на основе STL и TBB при небольшом количестве потоков, однако при увеличении числа потоков производительность TBB снижалась, что требует дополнительного анализа и оптимизации. Версия с OpenMP показала стабильное и предсказуемое улучшение времени выполнения с ростом числа потоков, что свидетельствует о её хорошей масштабируемости и практической применимости. Гибридная модель MPI + OpenMP не показала значительного преимущества в условиях эксперимента, возможно, из-за накладных расходов на межпроцессное взаимодействие.

В ходе работы были значительно улучшены практические навыки в области параллельного программирования. Проведен анализ и освоено применение различных технологий (OpenMP, TBB, STL, MPI) для решения задач сортировки больших массивов данных, что позволило выявить их сильные и слабые стороны.

\section*{Список литературы}
\begin{enumerate}
    \item Сиднев А.А., Сысоев А.В., Мееров И.Б. Библиотека Intel Threading Building Blocks – краткое описание // Материалы образовательного комплекса «Технологии разработки параллельных программ». Нижний Новгород, 2007.
    \item Параллельные вычисления https://parallel.uran.ru/node/182
    \item Параллельные заметки №3 — базовые конструкции OpenMP https://habr.com/ru/companies/intel/articles/85273/
    \item Рейндерс, Дж. Intel Threading Building Blocks: Практическое руководство по параллельному программированию / Дж. Рейндерс; пер. с англ. — М.: Вильямс, 2007. — 336 с. — ISBN 978-5-8459-1274-9
   \item Фильтрация изображения на FPGA https://habr.com/ru/articles/324070/

\end{enumerate}
\newpage
\section*{Приложение: Код программы}

\subsection*{Файл ops\_seq.cpp}

\begin{lstlisting}
#include "seq/komshina_d_image_filtering_vertical_gaussian/include/ops_seq.hpp"

#include <algorithm>
#include <cmath>
#include <cstddef>
#include <vector>

bool komshina_d_image_filtering_vertical_gaussian_seq::TestTaskSequential::PreProcessingImpl() {
  width_ = task_data->inputs_count[0];
  height_ = task_data->inputs_count[1];

  unsigned int input_size = width_ * height_ * 3;
  auto *in_ptr = reinterpret_cast<unsigned char *>(task_data->inputs[0]);
  input_.assign(in_ptr, in_ptr + input_size);

  unsigned int kernel_size = task_data->inputs_count[2];
  auto *kernel_ptr = reinterpret_cast<float *>(task_data->inputs[1]);
  kernel_.assign(kernel_ptr, kernel_ptr + kernel_size);

  output_.assign(input_.begin(), input_.end());

  return true;
}

bool komshina_d_image_filtering_vertical_gaussian_seq::TestTaskSequential::ValidationImpl() {
  if (task_data->inputs[0] == nullptr || task_data->inputs[1] == nullptr || task_data->outputs.empty() ||
      task_data->outputs[0] == nullptr) {
    return false;
  }

  const auto &input_count = task_data->inputs_count;
  const auto &output_count = task_data->outputs_count;

  if (input_count.size() < 3 || output_count.empty()) {
    return false;
  }

  constexpr int kKernelSize = 9;
  constexpr int kChannels = 3;

  bool valid_kernel = (input_count[2] == kKernelSize);
  bool valid_size = (input_count[0] * input_count[1] * kChannels == output_count[0]);

  return valid_kernel && valid_size;
}

bool komshina_d_image_filtering_vertical_gaussian_seq::TestTaskSequential::RunImpl() {
  for (std::size_t y = 1; y + 1 < height_; ++y) {
    for (std::size_t x = 1; x + 1 < width_; ++x) {
      std::size_t base_idx = (y * width_ + x) * 3;

      for (std::size_t c = 0; c < 3; ++c) {
        float total = 0.0F;
        std::size_t k_idx = 0;

        for (int ky = -1; ky <= 1; ++ky) {
          std::size_t row_idx = ((((y + ky) * width_) + (x - 1)) * 3) + c;

          for (int kx = -1; kx <= 1; ++kx, ++k_idx) {
            total += static_cast<float>(input_[row_idx]) * kernel_[k_idx];
            row_idx += 3;
          }
        }
        output_[base_idx + c] = std::clamp(static_cast<int>(std::round(total)), 0, 255);
      }
    }
  }
  return true;
}

bool komshina_d_image_filtering_vertical_gaussian_seq::TestTaskSequential::PostProcessingImpl() {
  if (task_data->outputs.empty() || task_data->outputs[0] == nullptr) {
    return false;
  }

  std::ranges::copy(output_, task_data->outputs[0]);

  return true;
}
\end{lstlisting}

\subsection*{Файл ops\_omp.cpp}
\begin{lstlisting}
#include "omp/komshina_d_image_filtering_vertical_gaussian/include/ops_omp.hpp"




#include <omp.h>




#include <algorithm>

#include <cmath>

#include <cstddef>

#include <vector>




bool komshina_d_image_filtering_vertical_gaussian_omp::TestTaskOpenMP::PreProcessingImpl() {

  width_ = task_data->inputs_count[0];

  height_ = task_data->inputs_count[1];




  unsigned int input_size = width_ * height_ * 3;

  auto *in_ptr = reinterpret_cast<unsigned char *>(task_data->inputs[0]);

  input_.assign(in_ptr, in_ptr + input_size);




  unsigned int kernel_size = task_data->inputs_count[2];

  auto *kernel_ptr = reinterpret_cast<float *>(task_data->inputs[1]);

  kernel_.assign(kernel_ptr, kernel_ptr + kernel_size);




  output_.assign(input_.begin(), input_.end());




  return true;

}




bool komshina_d_image_filtering_vertical_gaussian_omp::TestTaskOpenMP::ValidationImpl() {

  if (task_data->inputs[0] == nullptr || task_data->inputs[1] == nullptr || task_data->outputs.empty() ||

      task_data->outputs[0] == nullptr) {

    return false;

  }




  const auto &input_count = task_data->inputs_count;

  const auto &output_count = task_data->outputs_count;




  if (input_count.size() < 3 || output_count.empty()) {

    return false;

  }




  constexpr int kKernelSize = 9;

  constexpr int kChannels = 3;




  bool valid_kernel = (input_count[2] == kKernelSize);

  bool valid_size = (input_count[0] * input_count[1] * kChannels == output_count[0]);




  return valid_kernel && valid_size;

}




bool komshina_d_image_filtering_vertical_gaussian_omp::TestTaskOpenMP::RunImpl() {

  int local_height = static_cast<int>(height_);

  int local_width = static_cast<int>(width_);

  std::vector<float> local_kernel = kernel_;




  auto &input_ref = input_;

  auto &output_ref = output_;




#pragma omp parallel default(none) shared(input_ref, output_ref) firstprivate(local_height, local_width, local_kernel)

  {

#pragma omp for

    for (int y = 1; y < local_height - 1; ++y) {

      for (int x = 1; x < local_width - 1; ++x) {

        std::size_t base_idx = (y * local_width + x) * 3;




        for (int c = 0; c < 3; ++c) {

          float total = 0.0F;

          int k_idx = 0;




          for (int ky = -1; ky <= 1; ++ky) {

            std::size_t row_idx = ((((y + ky) * local_width) + (x - 1)) * 3) + c;




            for (int kx = -1; kx <= 1; ++kx, ++k_idx) {

              total += static_cast<float>(input_ref[row_idx]) * local_kernel[k_idx];

              row_idx += 3;

            }

          }

          output_ref[base_idx + c] = std::clamp(static_cast<int>(std::round(total)), 0, 255);

        }

      }

    }

  }

  return true;

}




bool komshina_d_image_filtering_vertical_gaussian_omp::TestTaskOpenMP::PostProcessingImpl() {

  if (task_data->outputs.empty() || task_data->outputs[0] == nullptr) {

    return false;

  }




  auto *output_ptr = task_data->outputs[0];

  std::ranges::copy(output_, output_ptr);




  return true;

}
\end{lstlisting}

\subsection*{Файл ops\_tbb.cpp}
\begin{lstlisting}
#include "tbb/komshina_d_image_filtering_vertical_gaussian/include/ops_tbb.hpp"

#include <tbb/tbb.h>

#include <algorithm>
#include <cmath>
#include <core/util/include/util.hpp>
#include <cstddef>
#include <vector>

#include "oneapi/tbb/task_arena.h"
#include "oneapi/tbb/task_group.h"

namespace {
void GaussianVerticalFilter(const std::vector<unsigned char> &in_vec, std::size_t width, std::size_t height,
                            const std::vector<float> &kernel, std::vector<unsigned char> &out_vec) {
  constexpr int kChannels = 3;
  const int k_radius = 1;

  for (std::size_t y = 1; y + 1 < height; ++y) {
    for (std::size_t x = 1; x + 1 < width; ++x) {
      std::size_t base_idx = (y * width + x) * kChannels;

      for (std::size_t c = 0; c < kChannels; ++c) {
        float total = 0.0F;
        std::size_t k_idx = 0;

        for (int ky = -k_radius; ky <= k_radius; ++ky) {
          std::size_t row_idx = ((((y + ky) * width) + (x - 1)) * kChannels) + c;

          for (int kx = -k_radius; kx <= k_radius; ++kx, ++k_idx) {
            total += static_cast<float>(in_vec[row_idx]) * kernel[k_idx];
            row_idx += kChannels;
          }
        }

        out_vec[base_idx + c] = std::clamp(static_cast<int>(std::round(total)), 0, 255);
      }
    }
  }
}
}  // namespace

bool komshina_d_image_filtering_vertical_gaussian_tbb::TestTaskTBB::PreProcessingImpl() {
  width_ = task_data->inputs_count[0];
  height_ = task_data->inputs_count[1];

  unsigned int input_size = width_ * height_ * 3;
  auto *in_ptr = reinterpret_cast<unsigned char *>(task_data->inputs[0]);
  input_.assign(in_ptr, in_ptr + input_size);

  unsigned int kernel_size = task_data->inputs_count[2];
  auto *kernel_ptr = reinterpret_cast<float *>(task_data->inputs[1]);
  kernel_.assign(kernel_ptr, kernel_ptr + kernel_size);

  output_.assign(input_.begin(), input_.end());

  return true;
}

bool komshina_d_image_filtering_vertical_gaussian_tbb::TestTaskTBB::ValidationImpl() {
  if (task_data->inputs[0] == nullptr || task_data->inputs[1] == nullptr || task_data->outputs.empty() ||
      task_data->outputs[0] == nullptr) {
    return false;
  }

  const auto &input_count = task_data->inputs_count;
  const auto &output_count = task_data->outputs_count;

  if (input_count.size() < 3 || output_count.empty()) {
    return false;
  }

  constexpr int kKernelSize = 9;
  constexpr int kChannels = 3;

  bool valid_kernel = (input_count[2] == kKernelSize);
  bool valid_size = (input_count[0] * input_count[1] * kChannels == output_count[0]);

  return valid_kernel && valid_size;
}

bool komshina_d_image_filtering_vertical_gaussian_tbb::TestTaskTBB::RunImpl() {
  oneapi::tbb::task_arena arena(ppc::util::GetPPCNumThreads());
  arena.execute([&] {
    oneapi::tbb::task_group tg;
    for (int thr = 0; thr < ppc::util::GetPPCNumThreads(); ++thr) {
      tg.run([&] { GaussianVerticalFilter(input_, width_, height_, kernel_, output_); });
    }
    tg.wait();
  });

  return true;
}

bool komshina_d_image_filtering_vertical_gaussian_tbb::TestTaskTBB::PostProcessingImpl() {
  if (task_data->outputs.empty() || task_data->outputs[0] == nullptr) {
    return false;
  }

  auto *output_ptr = task_data->outputs[0];
  std::ranges::copy(output_, output_ptr);

  return true;
}

\end{lstlisting}

\subsection*{Файл ops\_stl.cpp}
\begin{lstlisting}

#include "stl/komshina_d_image_filtering_vertical_gaussian/include/ops_stl.hpp"

#include <algorithm>
#include <cmath>
#include <cstddef>
#include <functional>
#include <thread>
#include <vector>

#include "core/util/include/util.hpp"

namespace {
void GaussianVerticalFilter(const std::vector<unsigned char> &input, std::vector<unsigned char> &output,
                            std::size_t width, std::size_t height, const std::vector<float> &kernel, int thread_id,
                            int num_threads) {
  for (std::size_t y = 1 + thread_id; y + 1 < height; y += num_threads) {
    for (std::size_t x = 1; x + 1 < width; ++x) {
      std::size_t base_idx = (y * width + x) * 3;

      for (std::size_t c = 0; c < 3; ++c) {
        float total = 0.0F;
        std::size_t k_idx = 0;

        for (int ky = -1; ky <= 1; ++ky) {
          std::size_t row_idx = (((y + ky) * width + (x - 1)) * 3) + c;

          for (int kx = -1; kx <= 1; ++kx, ++k_idx) {
            total += static_cast<float>(input[row_idx]) * kernel[k_idx];
            row_idx += 3;
          }
        }
        output[base_idx + c] = std::clamp(static_cast<int>(std::round(total)), 0, 255);
      }
    }
  }
}
}  // namespace

bool komshina_d_image_filtering_vertical_gaussian_stl::TestTaskSTL::PreProcessingImpl() {
  width_ = task_data->inputs_count[0];
  height_ = task_data->inputs_count[1];

  unsigned int input_size = width_ * height_ * 3;
  auto *in_ptr = reinterpret_cast<unsigned char *>(task_data->inputs[0]);
  input_.assign(in_ptr, in_ptr + input_size);

  unsigned int kernel_size = task_data->inputs_count[2];
  auto *kernel_ptr = reinterpret_cast<float *>(task_data->inputs[1]);
  kernel_.assign(kernel_ptr, kernel_ptr + kernel_size);

  output_.assign(input_.begin(), input_.end());

  return true;
}

bool komshina_d_image_filtering_vertical_gaussian_stl::TestTaskSTL::ValidationImpl() {
  if (task_data->inputs[0] == nullptr || task_data->inputs[1] == nullptr || task_data->outputs.empty() ||
      task_data->outputs[0] == nullptr) {
    return false;
  }

  const auto &input_count = task_data->inputs_count;
  const auto &output_count = task_data->outputs_count;

  if (input_count.size() < 3 || output_count.empty()) {
    return false;
  }

  constexpr int kKernelSize = 9;
  constexpr int kChannels = 3;

  bool valid_kernel = (input_count[2] == kKernelSize);
  bool valid_size = (input_count[0] * input_count[1] * kChannels == output_count[0]);

  return valid_kernel && valid_size;
}

bool komshina_d_image_filtering_vertical_gaussian_stl::TestTaskSTL::RunImpl() {
  const int num_threads = ppc::util::GetPPCNumThreads();
  std::vector<std::thread> threads(num_threads);

  for (int i = 0; i < num_threads; ++i) {
    threads[i] = std::thread(GaussianVerticalFilter, std::cref(input_), std::ref(output_), width_, height_,
                             std::cref(kernel_), i, num_threads);
  }

  for (auto &thread : threads) {
    thread.join();
  }

  return true;
}

bool komshina_d_image_filtering_vertical_gaussian_stl::TestTaskSTL::PostProcessingImpl() {
  if (task_data->outputs.empty() || task_data->outputs[0] == nullptr) {
    return false;
  }

  std::ranges::copy(output_, task_data->outputs[0]);

  return true;
}
\end{lstlisting}

\subsection*{Файл ops\_mpi+omp.cpp}


\begin{lstlisting}

#include "all/komshina_d_image_filtering_vertical_gaussian/include/ops_all.hpp"

#include <algorithm>
#include <cmath>
#include <cstddef>
#include <vector>

namespace {

void VerticalGaussianFilter(const std::vector<unsigned char>& local_input, std::vector<unsigned char>& local_output,
                            const std::vector<float>& kernel, std::size_t local_height, std::size_t width,
                            std::size_t k_radius, std::size_t halo_top) {
  const std::size_t k_channels = 3;

#pragma omp parallel for default(none) \
    shared(local_input, local_output, kernel, local_height, width, k_radius, halo_top)
  for (auto y = static_cast<std::ptrdiff_t>(halo_top); y < static_cast<std::ptrdiff_t>(halo_top + local_height); ++y) {
    for (std::size_t x = 0; x < width; ++x) {
      std::size_t base_idx = ((static_cast<std::size_t>(y) - halo_top) * width + x) * k_channels;

      for (std::size_t c = 0; c < k_channels; ++c) {
        float sum = 0.0F;

        for (std::ptrdiff_t k = -static_cast<std::ptrdiff_t>(k_radius); k <= static_cast<std::ptrdiff_t>(k_radius);
             ++k) {
          std::ptrdiff_t yy = y + k;
          if (yy < 0 || yy >= static_cast<std::ptrdiff_t>(local_height)) {
            continue;
          }

          std::size_t idx = (((static_cast<std::size_t>(yy) * width) + x) * k_channels) + c;
          sum += static_cast<float>(local_input[idx]) * kernel[k + k_radius];
        }

        local_output[base_idx + c] = std::clamp(static_cast<int>(std::round(sum)), 0, 255);
      }
    }
  }
}

}  // namespace

bool komshina_d_image_filtering_vertical_gaussian_all::TestTaskALL::PreProcessingImpl() {
  width_ = task_data->inputs_count[0];
  height_ = task_data->inputs_count[1];

  unsigned int input_size = width_ * height_ * 3;
  auto* in_ptr = reinterpret_cast<unsigned char*>(task_data->inputs[0]);
  input_.assign(in_ptr, in_ptr + input_size);

  unsigned int kernel_size = task_data->inputs_count[2];
  auto* kernel_ptr = reinterpret_cast<float*>(task_data->inputs[1]);
  kernel_.assign(kernel_ptr, kernel_ptr + kernel_size);

  output_.resize(input_size, 0);
  return true;
}

bool komshina_d_image_filtering_vertical_gaussian_all::TestTaskALL::ValidationImpl() {
  if (task_data->inputs[0] == nullptr || task_data->inputs[1] == nullptr || task_data->outputs.empty() ||
      task_data->outputs[0] == nullptr) {
    return false;
  }

  const auto& input_count = task_data->inputs_count;
  const auto& output_count = task_data->outputs_count;

  if (input_count.size() < 3 || output_count.empty()) {
    return false;
  }

  constexpr int kKernelSize = 9;
  constexpr int kChannels = 3;

  bool valid_kernel = (input_count[2] == kKernelSize);
  bool valid_size = (input_count[0] * input_count[1] * kChannels == output_count[0]);

  return valid_kernel && valid_size;
}

bool komshina_d_image_filtering_vertical_gaussian_all::TestTaskALL::RunImpl() {

 const int rank = world_.rank();
  const int size = world_.size();
  const std::size_t k_channels = 3;
  const std::size_t k_radius = kernel_.size() / 2;

  std::size_t rows_per_proc = height_ / static_cast<std::size_t>(size);
  std::size_t remainder = height_ % static_cast<std::size_t>(size);
  std::size_t local_height = rows_per_proc + ((static_cast<std::size_t>(rank) < remainder) ? 1 : 0);
  std::size_t offset = (static_cast<std::size_t>(rank) * rows_per_proc) +
                       std::min<std::size_t>(static_cast<std::size_t>(rank), remainder);

  std::size_t halo_top = (offset > 0) ? k_radius : 0;
  std::size_t halo_bottom = ((offset + local_height) < height_) ? k_radius : 0;

  std::vector<unsigned char> local_input =
      DistributeInputToProcesses(rank, size, k_radius, local_height, offset, halo_top, halo_bottom);

  std::vector<unsigned char> local_output(local_height * width_ * k_channels, 0);

  VerticalGaussianFilter(local_input, local_output, kernel_, local_height, width_, k_radius, halo_top);

  if (rank == 0) {
    std::ranges::copy(local_output, output_.begin() + static_cast<std::ptrdiff_t>(offset * width_ * k_channels));

    for (int i = 1; i < size; ++i) {
      auto si = static_cast<std::size_t>(i);
      std::size_t lh = rows_per_proc + ((si < remainder) ? 1 : 0);
      std::vector<unsigned char> temp(lh * width_ * k_channels);
      world_.recv(i, 1, temp);

      std::size_t target_offset = (si * rows_per_proc + std::min<std::size_t>(si, remainder)) * width_ * k_channels;
      std::ranges::copy(temp, output_.begin() + static_cast<std::ptrdiff_t>(target_offset));
    }
  } else {
    world_.send(0, 1, local_output);
  }

  return true;
}

std::vector<unsigned char> komshina_d_image_filtering_vertical_gaussian_all::TestTaskALL::DistributeInputToProcesses(
    int rank, int size, std::size_t k_radius, std::size_t local_height, std::size_t offset, std::size_t halo_top,
    std::size_t halo_bottom) {
  const std::size_t k_channels = 3;
  std::size_t total_rows = local_height + halo_top + halo_bottom;
  std::vector<unsigned char> local_input(total_rows * width_ * k_channels);

  if (rank == 0) {
    for (int i = 0; i < size; ++i) {
      auto si = static_cast<std::size_t>(i);
      auto lh = (height_ / static_cast<std::size_t>(size)) + ((si < (height_ % size)) ? 1 : 0);
      std::size_t off = (si * (height_ / static_cast<std::size_t>(size))) + std::min(si, height_ % size);

      std::size_t halo_t = (off > 0) ? k_radius : 0;
      std::size_t halo_b = ((off + lh) < height_) ? k_radius : 0;
      std::size_t rows = lh + halo_t + halo_b;

      std::ptrdiff_t start_row = static_cast<std::ptrdiff_t>(off) - static_cast<std::ptrdiff_t>(halo_t);
      std::size_t start_idx = std::max<std::ptrdiff_t>(start_row, 0) * width_ * k_channels;
      std::size_t count = rows * width_ * k_channels;

      if (i == 0) {
        std::copy(input_.begin() + static_cast<std::ptrdiff_t>(start_idx),
                  input_.begin() + static_cast<std::ptrdiff_t>(start_idx + count), local_input.begin());

      } else {
        std::vector<unsigned char> temp(input_.begin() + static_cast<std::ptrdiff_t>(start_idx),
                                        input_.begin() + static_cast<std::ptrdiff_t>(start_idx + count));
        world_.send(i, 0, temp);
      }
    }
  } else {
    world_.recv(0, 0, local_input);
  }

  return local_input;
}

bool komshina_d_image_filtering_vertical_gaussian_all::TestTaskALL::PostProcessingImpl() {
  if (world_.rank() == 0) {
    std::ranges::copy(output_, reinterpret_cast<unsigned char*>(task_data->outputs[0]));
  }
  return true;
}

\end{lstlisting}
\end{document}
