\documentclass[12pt]{article}
\usepackage[utf8]{inputenc}
\usepackage[russian]{babel}
\usepackage{amsmath}
\usepackage{geometry}
\usepackage{titlesec}
\usepackage{hyperref}
\usepackage{tocloft}
\usepackage{xcolor} 
\usepackage{listings}
\geometry{a4paper, left=25mm, right=15mm, top=20mm, bottom=20mm}

\hypersetup{
    colorlinks=true,
    linkcolor=black,
    urlcolor=black,
    pdftitle={Отчет по проекту},
    pdfauthor={Коровин Никита},
    pdfsubject={Сортировка Хоара с четно-нечетным слиянием Бэтчера}
}

\lstset{
  language=C++,
  basicstyle=\ttfamily\small,
  keywordstyle=\bfseries\color{blue},
  commentstyle=\itshape\color{green!50!black},
  stringstyle=\color{red!60!black},
  numbers=left,
  numberstyle=\tiny,
  stepnumber=1,
  numbersep=8pt,
  backgroundcolor=\color{gray!10},
  showspaces=false,
  showstringspaces=false,
  breaklines=true,
  frame=single,
  tabsize=2,
  captionpos=b
}

\renewcommand{\cftsecaftersnum}{.}
\renewcommand{\cftsecleader}{\cftdotfill{\cftdotsep}}
\titleformat{\section}{\large\bfseries}{\thesection.}{1em}{}

\begin{document}
\begin{titlepage}
    \centering
    \large
    Министерство образования и науки Российской Федерации\\
    Федеральное государственное автономное образовательное учреждение высшего образования\\
    \textbf{«Национальный исследовательский Нижегородский государственный университет им. Н.И. Лобачевского»}\\
    (ННГУ)\\[1cm]
    Институт информационных технологий, математики и механики\\
    Направление подготовки: «Программная инженерия»\\[2cm]
    
    {\Large \textbf{ОТЧЕТ}}\\[0.5cm]
    {\Large по задаче}\\[0.5cm]
    {\Large \textbf{«Сортировка Хоара с четно-нечетным слиянием Бэтчера»}}\\[0.5cm]
    {\Large Вариант №14}\\[4cm]
    
    \hfill\parbox{0.5\textwidth}{
        Выполнил: студент группы 3822Б1ПР2\\
        Коровин Никита\\[1.5em]
        Преподаватель:\\
        доцент, кандидат технических наук\\
        Сысоев Александр Владимирович
    }\\[3cm]
    
    \vspace*{\fill}
    Нижний Новгород\\
    2025
\end{titlepage}

\thispagestyle{empty}
\clearpage
\pagenumbering{arabic} 
\setcounter{page}{2} 
\tableofcontents
\clearpage
\setcounter{page}{3} 

\section{Введение}
\hspace*{1.25em}Сортировка Хоара (QuickSort) является одним из наиболее эффективных алгоритмов сортировки на практике благодаря среднему временному поведению $O(n\log n)$ и малому объёму дополнительной памяти. Однако при обработке больших массивов и желании ускорить выполнение на современных многоядерных системах необходимы параллельные подходы. 

\hspace*{1.25em}Четно-нечетное слияние (Odd–Even Merge) по Бэтчеру представляет собой схему параллельного объединения уже отсортированных фрагментов, позволяющую сравнительно просто организовать слияние блоков без глобальных операций, лишь местных сравнений и обменов. Сочетание QuickSort для локальной сортировки и четно-нечетного слияния для объединения блоков образует гибридный алгоритм, способный эффективно использовать ресурсы многоядерных и распределённых систем. 

\hspace*{1.25em}Цель данного отчёта — реализовать и сравнить пять версий гибридного алгоритма:
\begin{itemize}
    \item Последовательная реализация (Seq).
    \item Параллельная реализация с использованием OpenMP.
    \item Параллельная реализация с использованием Intel TBB.
    \item Параллельная реализация с использованием стандартных потоков C++ (\texttt{std::thread}).
    \item Гибридная версия с использованием Boost.MPI для распределения данных и OpenMP внутри каждого процесса.
\end{itemize}

\section{Постановка задачи}
\hspace*{1.25em}Необходимо разработать пять версий алгоритма сортировки Хоара с четно-нечетным слиянием Бэтчера для массивов целых чисел:
\begin{enumerate}
    \item Последовательная реализация для базовой проверки корректности и замера опорного времени.
    \item OpenMP-реализация, использующая директивы \texttt{\#pragma omp} для параллельной сортировки фрагментов и слияния блоков.
    \item Intel TBB-реализация, использующая \texttt{tbb::parallel\_invoke} и \texttt{tbb::parallel\_for} вместе с \texttt{tbb::enumerable\_thread\_specific}.
    \item STL-реализация на \texttt{std::thread} для создания задач параллельной сортировки и слияния.
    \item Комбинированная MPI+OpenMP-версия, в которой Boost.MPI распределяет исходный массив по процессам, а внутри каждого процесса используется OpenMP.
\end{enumerate}

\hspace*{1.25em}Каждую версию необходимо протестировать на одном и том же наборе данных, замерить время выполнения двух тестов:
\begin{itemize}
    \item \texttt{test\_pipeline\_run}
    \item \texttt{test\_task\_run}
\end{itemize}

\section{Общее описание алгоритма}
\hspace*{1.25em}Алгоритм состоит из двух ключевых этапов:
\begin{enumerate}
    \item \textbf{Локальная сортировка (QuickSort)}: Рекурсивный алгоритм Хоара, в котором опорный элемент выбирается случайным образом из текущего диапазона, чтобы снизить вероятность худшего случая $O(n^2)$.
    \item \textbf{Итеративное объединение фрагментов (Odd–Even Merge)}: После того, как каждый блок отсортирован независимо, соседние блоки сливаются через чередующиеся итерации «чётных» и «нечётных» пар с использованием функции \texttt{InPlaceMerge}.
\end{enumerate}

\hspace*{1.25em}В неспециализированной последовательной версии оба этапа выполняются последовательно. В параллельных версиях:
\begin{itemize}
    \item Этап QuickSort распараллеливается либо разделением рекурсии (OpenMP, TBB), либо распределением блоков (STL-потоки, MPI+OpenMP).
    \item Этап слияния Odd–Even Merge распараллеливается по парам блоков либо через \texttt{\#pragma omp parallel for}, \texttt{tbb::parallel\_for} или запуск отдельных потоков.
\end{itemize}

\clearpage
\section{Схемы распараллеливания}

\subsection{Последовательная версия (Seq)}
\hspace*{1.25em}В последовательной реализации отсутствует распараллеливание. Алгоритм выполняет:
\begin{enumerate}
    \item Чтение входных данных в вектор \texttt{input\_}.
    \item Вызов \texttt{QuickSort(input\_, 0, n-1)} при $n>1$.
    \item Копирование отсортированного результата в выходной буфер.
\end{enumerate}

\noindent\textbf{Ключевые детали реализации:}
\begin{itemize}
    \item \texttt{GetRandomIndex(int low, int high)} — генерация случайного индекса опорного элемента.
    \item \texttt{QuickSort(std::vector<int>\& arr, int low, int high)}:
    \begin{lstlisting}
if (low >= high) return;
int pivot_index = GetRandomIndex(low, high);
int pivot = arr[pivot_index];
auto part = std::partition(arr.begin()+low, arr.begin()+high+1,
                           [pivot](int x){ return x <= pivot; });
auto mid = std::partition(arr.begin()+low, part,
                          [pivot](int x){ return x < pivot; });
QuickSort(arr, low, mid_idx-1);
QuickSort(arr, mid_idx+1, high);
    \end{lstlisting}
    \item \texttt{InPlaceMerge} и \texttt{OddEvenMerge} выполняются последовательно:
    \begin{lstlisting}
for (int iter = 0; iter < max_iters; ++iter) {
  for (int b = iter%2; b < p; b+=2) {
    InPlaceMerge(blocks[b], blocks[b+1], buffer);
  }
  if (!changed) break;
}
    \end{lstlisting}
\end{itemize}

\subsection{OpenMP-версия (OMP)}
\hspace*{1.25em}В OpenMP-версии параллелизм организован двумя способами:
\begin{enumerate}
    \item \textbf{Локальная QuickSort с параллельными секциями}:
    \begin{itemize}
        \item Вычисляется \verb|max_depth = log2(omp_get_num_threads()) + 1|.
        \item Если \verb|depth < max_depth|, рекурсивные вызовы выполняются в \verb|#pragma omp parallel sections|.
    \end{itemize}
    \item \textbf{Объединение блоков через #pragma omp parallel for}:
    \begin{itemize}
        \item Итеративное четно-нечетное слияние распараллеливается по индексам блоков.
        \item Используется \verb|reduction(||:changed)| для объединения флагов изменений.
    \end{itemize}
\end{enumerate}

\noindent\textbf{Описание кода:}
\lstset{language=C++}
\begin{lstlisting}
// Пример фрагмента QuickSort для OpenMP
void QuickSort(It low, It high, int depth) {
  if (std::distance(low, high) <= 1) return;
  int pivot = *std::next(low, GetRandomIndex(0, n-1));
  auto part = std::partition(low, high, [pivot](int x){return x<=pivot;});
  auto mid = std::partition(low, part, [pivot](int x){return x<pivot;});
  int max_depth = log2(omp_get_max_threads()) + 1;
  if (depth < max_depth) {
    #pragma omp parallel sections
    {
      #pragma omp section
      QuickSort(low, mid, depth+1);
      #pragma omp section
      QuickSort(part, high, depth+1);
    }
  } else {
    QuickSort(low, mid, depth+1);
    QuickSort(part, high, depth+1);
  }
}

// Итеративное слияние
for (int iter = 0; iter < 2*p; ++iter) {
  bool changed = false;
  #pragma omp parallel for schedule(static) reduction(||:changed)
  for (int b = iter%2; b < p; b+=2) {
    InPlaceMerge(blocks[b], blocks[b+1], buffer);
  }
  if (!changed) break;
}
\end{lstlisting}

\subsection{Intel TBB-версия (TBB)}
\hspace*{1.25em}В TBB-реализации используются:
\begin{itemize}
    \item \texttt{tbb::parallel\_invoke} для распараллеливания рекурсии QuickSort до глубины max\_depth = log2(numThreads) + 1.
    \item \texttt{tbb::parallel\_for} совместно с \texttt{tbb::enumerable\_thread\_specific} для параллельного четно-нечетного слияния блоков.
\end{itemize}

\noindent\textbf{Фрагменты кода:}
\lstset{language=C++}
\begin{lstlisting}
// QuickSort с TBB
void QuickSort(It low, It high, int depth) {
  if (distance(low, high) <= 1) return;
  int pivot = *(low + GetRandomIndex(0, n-1));
  auto part = partition(low, high, [pivot](int x){ return x<=pivot; });
  auto mid  = partition(low, part,  [pivot](int x){ return x<pivot; });
  int max_depth = log2(ppc::util::GetPPCNumThreads()) + 1;
  if (depth < max_depth) {
    tbb::parallel_invoke(
      [&]{ QuickSort(low,  mid,   depth+1); },
      [&]{ QuickSort(part, high,  depth+1); }
    );
  } else {
    QuickSort(low,  mid,   depth+1);
    QuickSort(part, high,  depth+1);
  }
}

// Odd-Even Merge с TBB
tbb::enumerable_thread_specific<vector<int>> buffers(
    [=]{ return vector<int>(buffer_size); });
for (int iter=0; iter<2*p; ++iter) {
  atomic<bool> changed(false);
  tbb::parallel_for(
    tbb::blocked_range<int>(0, p/2),
    [&](auto &range){
      auto &buf = buffers.local();
      for (int idx = range.begin(); idx < range.end(); ++idx) {
        int i = ((iter + idx*2)%2) + idx*2;
        if (i+1<p) {
          if (InPlaceMerge(blocks[i], blocks[i+1], buf))
            changed.store(true, memory_order_relaxed);
        }
      }
    }
  );
  if (!changed.load()) break;
}
\end{lstlisting}

\subsection{STL-потоки (std::thread) версия (STL)}
\hspace*{1.25em}В данной реализации:
\begin{itemize}
    \item Разбиение массива на $p = \min(\text{numThreads},\, n/256)$ блоков.
    \item Для каждого блока создаётся поток \texttt{std::thread}, в котором вызывается \texttt{QuickSort(..., 0)} без ограничения глубины (последовательная рекурсия внутри блока).
    \item Объединение блоков через создание набора потоков на каждой итерации четно-нечетного слияния.
\end{itemize}

\noindent\textbf{Ключевые фрагменты:}
\lstset{language=C++}
\begin{lstlisting}
// Запуск сортировки блоков
int tasks = min(numThreads, max(1, n/256));
vector<thread> threads;
auto blocks = PartitionBlocks(input, tasks);
for (int i = 0; i < tasks; ++i) {
  threads.emplace_back([&, i]{
    QuickSort(blocks[i].low, blocks[i].high, 0);
  });
}
for (auto &t : threads) t.join();

// Odd-Even Merge через потоки
for (int iter=0; iter<2*tasks; ++iter) {
  atomic<bool> changed(false);
  vector<thread> mergeThreads;
  for (int b = iter%2; b < tasks; b+=2) {
    mergeThreads.emplace_back([&, b]{
      static thread_local vector<int> buf;
      buf.resize(buffer_size);
      if (InPlaceMerge(blocks[b], blocks[b+1], buf))
        changed.store(true, memory_order_relaxed);
    });
  }
  for (auto &t : mergeThreads) t.join();
  if (!changed.load()) break;
}
\end{lstlisting}

\subsection{Гибридная версия MPI+OpenMP (ALL)}
\hspace*{1.25em}В комбинированной версии:
\begin{enumerate}
    \item \textbf{Распределение данных с Boost.MPI:}
    \begin{itemize}
        \item \texttt{scatterv} разбивает исходный массив по числу MPI-процессов.
        \item Каждый процесс получает фрагмент \texttt{local}.
    \end{itemize}
    \item \textbf{Локальная обработка:}
    \begin{itemize}
        \item Фрагмент \texttt{local} разбивается на $p = \max(\text{omp\_max\_threads}, 1)$ блоков.
        \item \texttt{\#pragma omp parallel for} запускает QuickSort с ограничением глубины в каждой секции.
        \item \texttt{OddEvenMerge} выполняется параллельно теми же директивами.
    \end{itemize}
    \item \textbf{Сбор результата:}
    \begin{itemize}
        \item \texttt{gatherv} собирает отсортированные фрагменты на корневой процесс.
        \item Дополнительная финальная фаза OpenMP-сортировки + слияния на корневом процессе.
    \end{itemize}
\end{enumerate}

\noindent\textbf{Код MPI+OpenMP:}
\lstset{language=C++}
\begin{lstlisting}
// Распределение посредством Boost.MPI
vector<int> counts(size), displs(size);
for (int i=0; i<size; ++i) {
  counts[i] = chunk + (i<rem?1:0);
  displs[i] = i==0 ? 0 : displs[i-1]+counts[i-1];
}
vector<int> local(counts[rank]);
scatterv(world, root_ptr, counts, displs, local.data(), counts[rank], 0);

// Локальная параллельная сортировка
int p = max(omp_get_max_threads(),1);
auto blocks = PartitionBlocks(local, p);
#pragma omp parallel for schedule(static)
for (int i=0; i<p; ++i)
  QuickSort(blocks[i].low, blocks[i].high, 0);

// Слияние пар блоков
#pragma omp parallel for schedule(static) reduction(||:changed)
for (int b=iter%2; b<p; b+=2)
  InPlaceMerge(blocks[b], blocks[b+1], buffer);

// Сбор результатов
gatherv(world, local.data(), counts[rank], input.data(), counts, displs, 0);
if (rank==0) {
  // Финальная параллельная фаза на root
  auto blocks2 = PartitionBlocks(input, omp_get_max_threads());
  #pragma omp parallel for schedule(static)
  for (int i=0; i<blocks2.size(); ++i)
    QuickSort(blocks2[i].low, blocks2[i].high, 0);
  OddEvenMerge(blocks2);
}
\end{lstlisting}

\clearpage
\section{Результаты экспериментов}
\hspace*{1.25em}Все замеры проводились на машине:
\begin{itemize}
  \item Операционная система: Windows 11.
  \item Процессор: 12th Gen Intel(R) Core(TM) i7-12700H, 14 ядер, 20 логических процессоров.
  \item Оперативная память: 16 ГБ.
\end{itemize}

\begin{table}[htbp]
\centering
\caption{Сравнительные результаты пяти версий алгоритма}
\begin{tabular}{|l|r|r|r|r|r|}
\hline
\textbf{Версия} & \textbf{pipeline (мс)} & \textbf{task (мс)} & \textbf{всего (мс)} & \textbf{~Ускор. pipe~} & \textbf{~Ускор. task~} \\
\hline
Seq        & 2634 & 1088 & 3722 & 1.00  & 1.00  \\
OMP        & 2593 &  975 & 3569 & 1.02  & 1.12  \\
TBB        & 1844 &  683 & 2528 & 1.43  & 1.59  \\
STL threads& 1508 &  625 & 2134 & 1.75  & 1.74  \\
MPI+OMP    & 2458 & 1477 & 3924 & 1.07  & 0.74  \\
\hline
\end{tabular}
\label{tab:results}
\end{table}

\section{Выводы из результатов}
\hspace*{1.25em}Как видно из таблицы \ref{tab:results}, наиболее высокое ускорение по сравнению с последовательной версией достигнуто в реализации на базовых потоках STL и в TBB-версии. Это объясняется минимальными накладными расходами на создание потоков и эффективной балансировкой нагрузки:

\begin{itemize}
  \item STL-потоки: хорошая масштабируемость при $n/256$ блоках, низкая сложность синхронизации при слиянии.
  \item TBB-версия использует динамическое планирование задач, а для локальных буферов при слиянии применяется \texttt{tbb::enumerable\_thread\_specific}.
  \item OpenMP: практически не даёт значительного выигрыша из-за накладных расходов на создание параллельных секций.
  \item MPI+OpenMP: распределение данных по процессам добавляет сетевые задержки, что ухудшает общую производительность.
\end{itemize}

\section{Заключение}
\hspace*{1.25em}В рамках работы реализовано пять версий гибридного алгоритма сортировки Хоара с четно-нечетным слиянием Бэтчера. Проведён сравнительный анализ производительности, показавший:

\begin{itemize}
  \item Эффективность TBB и STL-потоков в локальных многопоточн. средах.
  \item Ограниченную пользу от OpenMP для данного алгоритма.
  \item Перегрузку из-за коммуникаций в распределённой MPI+OpenMP версии.
\end{itemize}

\hspace*{1.25em}Полученные результаты подчёркивают важность выбора подходящей параллельной модели в зависимости от архитектуры системы и характера задачи.

\begin{thebibliography}{9}
\bibitem{Hoare1962}
  C.A.R. Hoare, \emph{QuickSort}, The Computer Journal, vol.5, no.1, pp.10–15, 1962.
\bibitem{Batcher1968}
  K. E. Batcher, \emph{Sorting networks and their applications}, Proc. AFIPS Spring Joint Computer Conference, 1968.
\bibitem{OpenMP}
  OpenMP Architecture Review Board, \emph{OpenMP Application Program Interface}, Version 4.5, 2015.
\bibitem{OneTBB}
  Intel, \emph{oneAPI Threading Building Blocks}, Version 2021.1.
\end{thebibliography}

\appendix
\section*{Приложение}
\addcontentsline{toc}{section}{Приложение}
\hspace*{1.25em}В приложении приведены ключевые фрагменты кода всех пяти версий алгоритма:

\subsection*{Seq: TestTaskSequential::QuickSort}
\begin{lstlisting}
// seq/ops_seq.cpp
void TestTaskSequential::QuickSort(vector<int>& arr, int low, int high) {
  if (low >= high) return;
  int pi = GetRandomIndex(low, high);
  int pv = arr[pi];
  auto part = partition(arr.begin()+low, arr.begin()+high+1,
                        [pv](int x){return x<=pv;});
  auto mid = partition(arr.begin()+low, part,
                       [pv](int x){return x<pv;});
  int i = distance(arr.begin(), mid);
  int j = distance(arr.begin(), part) - 1;
  if (low < i-1) QuickSort(arr, low, i-1);
  if (j+1 < high) QuickSort(arr, j+1, high);
}
\end{lstlisting}

\subsection*{OMP: локальная сортировка и слияние}
\begin{lstlisting}
// omp/ops_omp.cpp
#pragma omp parallel sections
{
  #pragma omp section
    QuickSort(low, mid, depth+1);
  #pragma omp section
    QuickSort(part, high, depth+1);
}
#pragma omp parallel for schedule(static) reduction(||:changed)
for (int b = iter%2; b < p; b+=2)
  InPlaceMerge(blocks[b], blocks[b+1], buffer);
\end{lstlisting}

\subsection*{TBB}
\begin{lstlisting}
// tbb/ops_tbb.cpp
tbb::parallel_invoke(
  [&]{ QuickSort(low, mid, depth+1); },
  [&]{ QuickSort(part, high, depth+1); }
);
tbb::parallel_for(tbb::blocked_range<int>(0,p/2), [&](auto &r){
  // порядок слияния
});
\end{lstlisting}

\subsection*{STL: создание потоков}
\begin{lstlisting}
// stl/ops_stl.cpp
vector<thread> threads;
for (int i = 0; i < tasks; ++i)
  threads.emplace_back([&,i]{ QuickSort(blocks[i].low, blocks[i].high,0); });
for (auto &t: threads) t.join();
\end{lstlisting}

\subsection*{MPI+OpenMP: scatterv, parallel for, gatherv}
\begin{lstlisting}
// all/ops_all.cpp
boost::mpi::scatterv(world, root_ptr, counts, displs,
                     local.data(), counts[rank], 0);
#pragma omp parallel for schedule(static)
for (int i=0; i<p; ++i)
  QuickSort(blocks[i].low, blocks[i].high, 0);
boost::mpi::gatherv(world, local.data(), counts[rank],
                   input.data(), counts, displs, 0);
\end{lstlisting}

\end{document}
